
\usepackage{fontspec}
\setmainfont{Linux Libertine O}
\usepackage{verbatim}
\usepackage{longtable}
\usepackage{booktabs}
\usepackage{pgf}
\usepgflibrary{decorations.text}

\PassOptionsToPackage{prefix}{glossaries-extra}
\usepackage
 [
    novref,
    fnsymleft,
    jdroutline,
%   debug=showwrgloss
 ]{texjavahelp}

\hypersetup{colorlinks,linkcolor=blue}

\newcommand{\FlowframTk}{FlowframTk}
\newcommand{\tool}[1]{\gls{menu.tools.#1} tool}
\newcommand{\texttool}{\tool{textarea}}
\newcommand{\mathstool}{\tool{math}}
\newcommand{\textmode}{\glslink{menu.tools.textarea}{text mode}}
\newcommand{\mathsmode}{%
 \glslink{menu.tools.math}{\manmsg{maths} mode}%
 \glsadd{index.maths-mode}%
}
\newcommand{\selectmode}{\glslink{menu.tools.select}{select mode}}
\newcommand{\select}{\hyperref[sec:selectobjects]{select}}
\newcommand{\selects}{\hyperref[sec:selectobjects]{selects}}
\newcommand{\selected}{\hyperref[sec:selectobjects]{selected}}
\newcommand{\selection}{\hyperref[sec:selectobjects]{selection}}
\newcommand{\gridlock}{\hyperref[mi:gridlock]{grid lock}}
\newcommand{\editpathmode}{\hyperref[sec:editpath]{edit path mode}}
\newcommand{\textattr}[1]{\gls{index.text-attribute.#1}}
\newcommand{\textattrpl}[1]{\glspl{index.text-attribute.#1}}
\newcommand{\pathattr}[1]{\gls{index.path-attribute.#1}}
\newcommand{\pathattrpl}[1]{\glspl{index.path-attribute.#1}}
\newcommand{\affinetrans}[1]{\gls{affine-transformation.#1}}

\newcommand{\filefn}[1]{\menuitem{file.#1}}
\newcommand{\editfn}[1]{\menuitem{edit.#1}}
\newcommand{\transformfn}[1]{\menuitem{transform.#1}}

% Needs to be fairly simple for the HelpSet HTML files:
\newcommand{\degrees}[1]{#1\textdegree}

\title{User Manual for \FlowframTk}
\author{Nicola L.C. Talbot\\\href{https://www.dickimaw-books.com/}{\nolinkurl{dickimaw-books.com}}}

\subtitle{\bigskip\includeimg{title}\bigskip}

\InputIfFileExists{version}{}{}

\GlsXtrLoadResources[src={flowframtk,shared,\langbibsrcs},
 \TeXJavaHelpSymbolResourceOptions
]

\GlsXtrLoadResources[src={flowframtk,shared,\langbibsrcs},
 \TeXJavaHelpGlsResourceOptions
]

\begin{document}
\maketitle

\FlowframTk\ is licensed under the terms of the GNU General Public License
(\url{http://gnu.org/licenses/gpl.html"}).
See \sectionref{sec:licence}.

DOCUMENTATION IS PROVIDED \qt{AS IS} AND ALL EXPRESS OR IMPLIED
CONDITIONS, REPRESENTATIONS AND WARRANTIES, INCLUDING ANY
IMPLIED WARRANTY OF MERCHANTABILITY, FITNESS FOR A PARTICULAR
PURPOSE OR NON-INFRINGEMENT, ARE DISCLAIMED, EXCEPT TO THE EXTENT THAT
SUCH DISCLAIMERS ARE HELD TO BE LEGALLY INVALID.

This document is a user manual for \FlowframTk. For information about
JDRView or jdrutils, see \filefmt{jdrview.pdf} or
\filefmt{jdrutils.pdf}, respectively.

\IfTeXParserLib
  {%
    This manual is also available as a PDF document, \filefmt{flowframtk.pdf}. 
  }
  {}

The latest version of \FlowframTk\ can be downloaded from
\url{https://ctan.org/pkg/flowframtk}.
Older versions can be downloaded from
\url{http://www.dickimaw-books.com/software/flowframtk/}.
The source code is in the GitHub repository
\url{https://github.com/nlct/flowframtk}.

Occasionally the canvas doesn't get redrawn correctly. To force a
redraw, use \accelerator{menu.debug.revalidate}.

\frontmatter
\tableofcontents
%\listoffigures

\mainmatter
\chapter{Introduction}
\label{sec:introduction}

\FlowframTk\ (formerly \gls{JpgfDraw})
is a \gls{vectorgraphics} application written in \gls{Java}, with a 
\gls{gui}.  The main purpose of \FlowframTk\ is to generate \LaTeX\ packages or 
classes that use the \sty{flowfram} package, and to generate
\sty{pgf} picture drawing code. As a side-effect, it can also be
used to generate \gls{parshape} and (\sty{shapepar} package) \gls{shapepar} or 
\gls{Shapepar} specifications. In order to run the application you 
must have the \gls{jre} installed.

In \FlowframTk, you can:

\begin{itemize}
  \item construct shapes using line, move and cubic Bézier segments
(\sectionref{sec:newobjects});

  \item edit paths by modifying the defining \glspl{controlpt}
  (\sectionref{sec:editpath});

  \item transforming paths with \glsxtrfull{cag} (\sectionref{sec:cag}).

  \item incorporate text (\sectionref{sec:newtext}) and 
  bitmap images (\sectionref{sec:insertbitmap})
  for annotating and background effects;

  \item combine text and a path to form a text-along-path
  effect (\sectionref{sec:textpath});

  \item apply replicas to a shape to form patterns
  (\sectionref{sec:patterns});

  \item extract the parameter's for \TeX's \gls{parshape} command and for 
  the \gls{shapepar} (or \gls{Shapepar}) command defined in the
  \sty{shapepar} package;

  \item construct frames for use with the \sty{flowfram} package
  (\sectionref{sec:flowframe}).
\end{itemize}

Pictures can be saved as (\sectionref{sec:saveimage}) or loaded
(\sectionref{sec:loadimage}) from \FlowframTk's native \gls{JDR}
(binary) or \gls{AJR} (ASCII) file formats. Additionally, images can be
exported (\sectionref{sec:exportimage}) as:

\begin{itemize}
    \item a \LaTeX\ file containing a \env{pgfpicture} environment
    for inclusion in a \LaTeX\ document;

    \item a single-paged complete \LaTeX\ document containing the image
    (either just encapsulating the image or with the page set to the
    same size as the \gls{canvas});

    \item a \LaTeX\ package or class that loads the \sty{flowfram} package;

    \item a PNG file;

    \item a PostScript file;

    \item \pgls{svg} file.
\end{itemize}

Note that the export to PS/PDF/SVG functions use external
processes, such as \app{latex} and \app{dvips}.
\strong{You can't import back from the files you can export to.}

The low-level \sty{pgf} basic layer commands are used during
exports to \LaTeX\ files. These commands aren't particularly easy to
read but are faster to compile than the higher-level commands that
require extra processing to parse the syntax and perform additional
calculations. The primary purpose of \FlowframTk\ is to provide a
graphical interface that can generate complicated code that's hard
to write manually.

\FlowframTk\ was tested with version 3.0.0 of the \sty{pgf}
package, version 1.16 of the \sty{flowfram} package and version
2.2 of the \sty{shapepar} package. Files created by \FlowframTk\
may not work with earlier versions of those packages. Note that some
DVI viewers may not understand PGF specials. I strongly recommend
that you read the user manuals for those packages.


Notation: a \gls{primaryclick} is a single click with the
\gls{primary-mouse-button}. This is typically the left button for a
right-handed mouse (or left side of a trackpad), but may be the
right button for a left-handed mouse. A \gls{menuclick} is a click
with the \gls{context-menu} mouse button. This button depends on your
configuration, but is typically the right button for a right-handed
two-button mouse (and vice-versa for a left-handed two-button mouse). 
If the button isn't mentioned, a~\gls{click} can
be assumed to mean a~\gls{primaryclick}.

\chapter{Installation}
\label{sec:install}

Ensure that you have the \gls{jre} installed.
This can be downloaded from \url{http://java.sun.com/j2se/}.
You must ensure that you use at least Java~8, as \FlowframTk\
does not work with earlier versions.

To install, download the installer
\metafilefmt{flowframtk-}{appversion}{-installer.jar}
(where \meta{appversion} is the application version) and run it. 
This can be done from a terminal or command prompt using:

\begin{terminal}
java -jar flowframtk-\meta{appversion}-installer.jar
\end{terminal}

Depending on the setup of your operating system, you may also be
able to \gls{doubleclick} on the \ext{jar} file to run it.


\chapter{Accessibility}
\label{sec:access}

Most of \FlowframTk's mouse functions can be emulated using the
keyboard. Keyboard accelerators and
their menu mnemonic equivalents are listed in \tableref{tab:access}.

\begin{information}
Some functions will also try to move the pointer (such as
\menu{navigate.find}). This is done via the \emph{\gls{robot}}.
However note that some platforms or environments do not allow applications
to move the pointer, so keyboard functions that move the pointer are
not guaranteed to work on every system
\end{information}

Many of the accelerators may be changed using the \dialog{accelerators}
tab in the \dialog{configui} window (see
\sectionref{sec:configureuidialog}).  \accelerator{menubarfocus}
switches the focus to the menu bar. The return \keys{\keyref{return}}
key will usually be equivalent to the \widget{okay} button
except when the focus is with a component that interprets
\keys{\keyref{return}} for some other purpose (such as
a~\gls{dropdown}), in which case you need to use \accelerator{okay}.

Within editable text fields, you can use \keys{\keyref{ctrl}+A} to
select all the text, or shift \keys{\keyref{shift}} followed by
the left \keys{\keyref{left}} or right \keys{\keyref{right}} arrow
key to select a portion of the text. If some of the text has been
selected, you can use \keys{\keyref{ctrl}+C} or
\keys{\keyref{ctrl}+X} to copy or cut the text onto the clipboard,
and you can use \keys{\keyref{ctrl}+V} to paste text from the
clipboard into the text field.

\begin{warning}
Note that a few of the accelerators previously used by \gls{JpgfDraw} have
been changed in \FlowframTk, as they caused a conflict. For example,
\keys{\keyref{escape}} cancels a displayed menu, so if you use it to
dismiss a menu while you had a \gls{path} under construction, the
entire path could be discarded, which is undesirable. Therefore the
abandon path accelerator is now \accelerator{tools.abandon}. The
\accelerator{help.manual} key is now only for the main manual help
button. The help buttons in dialog windows and the preamble editor
are activated with \accelerator{help} and the help button in the
status bar is activated with \accelerator{contexthelp} (otherwise
the main help button, preamble editor help button and status bar
help button would conflict).
\end{warning}

\seealsorefs{sec:configureuidialog,sec:selectobjects,sec:newobjects,sec:accesstutorial}

\begin{longtable}{lp{0.5\textwidth}p{0.2\textwidth}}
\caption{Keyboard Accelerators and Menu Mnemonics\label{tab:access}}\\
\bfseries Accelerator & \bfseries Function & \bfseries Mnemonic \\
\endfirsthead
\caption*{Keyboard Accelerators and Menu Mnemonics (Continued)}\\
\bfseries Accelerator & \bfseries Function & \bfseries Mnemonic \\
\endhead
\multicolumn{3}{l}{\fnsymtext{2}{Functions that move the pointer.}}\\\endfoot
\keys{\keyref{return}} (return) &
Finish current \gls{path}/\gls{textarea} & 
\mnemonictrail{tools.finish}\tabularnewline
& 
\emph{or} select \widget{okay} button in dialog boxes &
\mnemonic{okay}\tabularnewline
%
& 
\emph{or} select \widget{close} button in dialog boxes &
\mnemonic{close}\tabularnewline
%
\accelerator{okay} &
Select \widget{okay} button in dialog boxes &
\mnemonic{okay}\tabularnewline
%
\accelerator{tools.abandon} &
Abandon current \gls{path} &
\mnemonictrail{tools.abandon}\tabularnewline
%
\keys{\keyref{escape}} &
Select \widget{cancel} button in dialog boxes &
\mnemonic{cancel}\tabularnewline
%
 & 
\emph{or} close displayed menu &
\tabularnewline
%
\accelerator{editpath.delete_control} & 
Delete selected \gls{controlpt} &
\accelerator{popup} \mnemonic{editpath.delete_control}
\tabularnewline
%
\accelerator{delete_last} (backspace) &
Delete last segment (path construction mode) &
\tabularnewline
%
\keys{\keyref{insert}} &
Add \gls{controlpt} &
\accelerator{popup} \mnemonic{editpath.add_control}
\tabularnewline
&
\emph{or} display symbol dialog box &
\accelerator{popup} \mnemonic{textarea.insert_symbol}
\tabularnewline
%
\accelerator{focusnext} &
Move focus to next focusable component &
\tabularnewline
%
\accelerator{focusselect} (space) &
Select component with current focus &
\tabularnewline
\midrule
%
\keys{\keyref{pageup}} &
Scroll up by one screen full &
\tabularnewline
%
\keys{\keyref{pagedown}} &
Scroll down by one screen full &
\tabularnewline
%
\keys{\keyref{ctrl}+\keyref{pagedown}} &
If in a tabbed pane: &
\tabularnewline
&
\leftquadpar{Move to the next tab} &
\tabularnewline
&
Otherwise: &
\tabularnewline
&
\leftquadpar{Scroll right by one screen full} &
\tabularnewline
%
\keys{\keyref{ctrl}+\keyref{pageup}} &
If in a tabbed pane: &
\tabularnewline
&
\leftquadpar{Move to the previous tab} &
\tabularnewline
 &
 Otherwise: &
\tabularnewline
&
\leftquadpar{Scroll left by one screen full} &
\tabularnewline
%
Arrow Keys &
If \gls{primary-mouse-button} is pressed: &
\tabularnewline
 &
 \leftquadpar{\fnsym{2}Move mouse by one pixel in given direction} &
\tabularnewline
 &
Otherwise: &
\tabularnewline
 &
\leftquadpar{Scroll by one tick mark in given direction} &
\tabularnewline
%
\keys{\keyref{home}} &
Scroll to the top of the \gls{canvas} &
\tabularnewline
%
\keys{\keyref{end}} &
Scroll to the bottom of the \gls{canvas} &
\tabularnewline
%
\keys{\keyref{ctrl}+\keyref{home}} &
Scroll leftmost &
\tabularnewline
%
\keys{\keyref{ctrl}+\keyref{end}} &
Scroll rightmost &
\tabularnewline
\midrule
%
\accelerator{help.manual} &
Display Handbook &
\mnemonictrail{help.manual}
\tabularnewline
\accelerator{settings.grid.show} &
Show/hide \gls{grid} &
\mnemonictrail{settings.grid.show}
\tabularnewline
\accelerator{popup} &
Show \gls{context-menu} (if available) &
\tabularnewline
%
\accelerator{construct_click} &
Emulate \gls{primaryclick} in \gls{construction} &
\tabularnewline
%
\accelerator{navigate.goto} &
 \fnsym{2}Go to coordinate &
\mnemonictrail{navigate.goto}
\tabularnewline
%
\keys{\actualkey{F6}} &
Select mode: &
\tabularnewline
 &
 \leftquadpar{Deselect the \glsdisp{backobject}{backmost} selected 
  \gls{object}, and select next object in the \gls{stack}} &
\mnemonictrail{navigate.skip}
\tabularnewline
&
Edit mode: &
\tabularnewline
&
\leftquadpar{Select next \gls{controlpt}} &
\accelerator{popup} \mnemonic{editpath.next_control}
\tabularnewline
%
\keys{\actualkey{F7}} &
Select mode: &
\tabularnewline
 &
\leftquadpar{Move selected \gls{object}} &
\mnemonictrail{edit.moveby}
\tabularnewline
 &
 Edit mode: &
\tabularnewline
 &
\leftquadpar{Move selected \gls{controlpt}} &
\accelerator{popup} \mnemonic{editpath.coordinates}
\tabularnewline
%
\accelerator{edit.undo} &
Undo &
\mnemonictrail{edit.undo} \tabularnewline
%
\accelerator{edit.redo} &
Redo &
\mnemonictrail{edit.redo}
\tabularnewline
%
\accelerator{menubarfocus} &
Switch focus to menu bar &
\tabularnewline
%
\accelerator{menu.debug.revalidate} &
Repaint canvas &
\tabularnewline
%
\accelerator{contexthelp} &
Display current tool help (status bar help button) &
\tabularnewline
%
\accelerator{menu.debug.dumpall} &
Saves all images to the \gls{config_dir} &
\tabularnewline
%
\accelerator{help} &
Display subject help (help buttons in dialog windows or preamble
editor) &
\tabularnewline
\accelerator{settings.grid.lock} &
Lock/unlock grid &
\mnemonictrail{settings.grid.lock}
\tabularnewline
%
\accelerator{navigate.select} &
Select next \gls{object} in the \gls{stack} (from the
\gls{front}), and deselect all others &
\mnemonictrail{navigate.select} \tabularnewline
%
\keys{\actualkey{F6}} &
Select mode: &
\tabularnewline
&
\leftquadpar{Add next \gls{object} in the \gls{stack}
     (from the \gls{front}) to selection} &
\mnemonictrail{navigate.add_next}
\tabularnewline
 &
Edit mode: &
\tabularnewline
%
 &
\leftquadpar{Select previous \gls{controlpt}} &
\accelerator{popup} \mnemonic{editpath.prev_control}
\tabularnewline
%
\accelerator{navigate.find} &
\fnsym{2}Find selected \gls{object} &
\mnemonictrail{navigate.find}
\tabularnewline
%
\accelerator{menu.debug.objectinfo} &
Displays debugging information &
\tabularnewline
%
\accelerator{menu.debug.writelog} &
Writes log file in the \gls{config_dir} &
\tabularnewline
\accelerator{close_window} &
Quit &
\mnemonictrail{file.quit}
\tabularnewline
\midrule
%
\accelerator{edit.select_all} &
Select all \gls{object} &
\mnemonictrail{edit.select_all}
\tabularnewline
%
\accelerator{edit.back} &
Move selected \gls{object} to the \gls{back} &
\mnemonictrail{edit.back}
\tabularnewline
%
\accelerator{edit.copy} &
Copy selected \gls{object} to clipboard &
\mnemonictrail{edit.copy}
\tabularnewline
%
\accelerator{transform.convert} &
Convert outline to a \gls{path} &
\mnemonictrail{transform.convert}
\tabularnewline
%
\accelerator{tools.ellipse} &
Switch to ellipse tool &
\mnemonictrail{tools.ellipse}
\tabularnewline
%
\keys{\keyref{ctrl}+\actualkey{F}} &
Select mode: &
\tabularnewline
 &
\leftquadpar{Move selected \gls{object} to the \gls{front}} &
\mnemonictrail{edit.front}
\tabularnewline
 & \TeX\ editor: &
\tabularnewline
 &
\leftquadpar{Find text} &
\mnemonic{texeditor.search.find}
\tabularnewline
\keys{\keyref{ctrl}+\actualkey{G}} &
Select mode: &
\tabularnewline
 &
\leftquadpar{Group selected \gls{object}} &
\mnemonictrail{transform.group}
\tabularnewline
 &
\TeX\ editor: &
\tabularnewline
 &
\leftquadpar{Find again} &
\mnemonic{texeditor.search.find_again}
\tabularnewline
\keys{\keyref{ctrl}+\actualkey{H}} &
Select mode: &
\tabularnewline
 &
\leftquadpar{Shear selected \gls{object}} &
\mnemonictrail{transform.shear}
\tabularnewline
 &
\TeX\ editor: &
\tabularnewline
 &
\leftquadpar{Find and Replace text} &
\mnemonictrail{texeditor.search.replace}
\tabularnewline
%
\accelerator{edit.path.edit} &
Edit selected \gls{path} &
\mnemonictrail{edit.path.edit}
\tabularnewline
%
\accelerator{transform.merge} &
Merge selected \gls{path} &
\mnemonictrail{transform.merge}
\tabularnewline
%
\accelerator{tools.open_curve} &
Switch to open curve tool &
\mnemonictrail{tools.open_curve}
\tabularnewline
%
\accelerator{tools.open_line} &
Switch to open line tool &
\mnemonictrail{tools.open_line}
\tabularnewline
%
\accelerator{tools.gap} &
Gap function &
\mnemonictrail{tools.gap}
\tabularnewline
%
\accelerator{file.new} &
New canvas &
\mnemonictrail{file.new}
\tabularnewline
%
\accelerator{file.open} &
Open \gls{JDR} or \gls{JDR} file &
\mnemonictrail{file.open}
\tabularnewline
%
\accelerator{tools.select} &
Switch to select tool &
\mnemonictrail{tools.select}
\tabularnewline
%
\accelerator{file.quit} &
Quit &
\mnemonictrail{file.quit}
\tabularnewline
%
\accelerator{tools.rectangle} &
Switch to rectangle tool &
\mnemonictrail{tools.rectangle}
\tabularnewline
%
\accelerator{file.save} &
Save current image &
\mnemonictrail{file.save}
\tabularnewline
%
\accelerator{tools.textarea} &
Switch to text tool &
\mnemonictrail{tools.textarea}
\tabularnewline
%
\accelerator{transform.ungroup} &
Ungroup selected \gls{group} &
\mnemonictrail{transform.ungroup}
\tabularnewline
%
\accelerator{edit.paste} &
Paste \gls{object} from clipboard &
\mnemonictrail{edit.paste}
\tabularnewline
%
\accelerator{transform.rotate} &
Rotate selected \gls{object} &
\mnemonictrail{transform.rotate}
\tabularnewline
%
\accelerator{edit.cut} &
Cut selected \gls{object} &
\mnemonictrail{edit.cut}
\tabularnewline
%
\accelerator{edit.path.style.all_styles} &
Edit the selected paths' line styles &
\mnemonictrail{edit.path.style.all_styles}
\tabularnewline
%
\accelerator{transform.scale} &
Scale selected \gls{object} &
\mnemonictrail{transform.scale}
\tabularnewline
%
\accelerator{edit.deselect_all} &
Deselect all &
\mnemonictrail{edit.deselect_all}
\tabularnewline
%
\accelerator{edit.movedown} &
Move selected \gls{object} down the \gls{stack} &
\mnemonictrail{edit.movedown}
\tabularnewline
%
\accelerator{edit.moveup} &
Move selected \gls{object} up the \gls{stack} &
\mnemonictrail{edit.moveup}
\tabularnewline
%
\accelerator{edit.textarea.edit} &
Edit selected text &
\mnemonictrail{edit.textarea.edit}
\tabularnewline
%
\accelerator{tools.closed_curve} &
Switch to closed curve tool &
\mnemonictrail{tools.closed_curve}
\tabularnewline
%
\accelerator{tools.closed_line} &
Switch to closed line tool &
\mnemonictrail{tools.closed_line}
\tabularnewline
%
\accelerator{tools.math} &
Switch to \mathstool &
\mnemonictrail{tools.math}
\tabularnewline
\midrule
%
\keys{\keyref{alt}+\actualkey{1}}\ldots\keys{\keyref{alt}+\actualkey{8}} &
\Glsname{linear-gradient-direction} selectors &
\tabularnewline
%
\keys{\keyref{alt}+\actualkey{1}}\ldots\keys{\keyref{alt}+\actualkey{9}} &
\Glsname{radial-gradient-start-location} selectors &
\end{longtable}


\chapter{Settings}\label{sec:settings}

You can customise the appearance of \FlowframTk's main window either
using the command line arguments (\sectionref{sec:cmdargs}) or
using the settings menu (\sectionref{sec:settingsmenu}).

\section{Command Line Arguments}\label{sec:cmdargs}

\FlowframTk\ can be invoked from a command prompt. 

\appdef{flowframtk}

Depending on your installation, this command may run a bash script
(which will run the application \ext{jar} file) or this command may
be an executable form of the application \ext{jar} file.
The application \ext{jar} file is called \file{flowframtk.jar}
and may also be invoked from the command line explicitly:
\begin{terminal}
java -jar path/to/\file{flowframtk.jar}
\end{terminal}
where \filefmt{path/to/} is the path to the \file{flowframtk.jar}
file.

Note that the options and the filenames may be omitted.
Each filename must be either a \gls{JDR} or an \gls{AJR} file.
The \app{flowframtk} bash script uses the environment variable 
\inlineglsdef{JDR_JVMOPTS} to pass
options to the \gls{jvm}.  For example, if you want to run \FlowframTk\ with
a maximum size of 128Mb for the memory allocation pool, you can set
\gls{JDR_JVMOPTS} to \verb|-Xmx128m|:
\begin{terminal}
setenv \gls{JDR_JVMOPTS} -Xmx128m
\end{terminal}
The \app{flowframtk} script also uses the environment variable
\inlineglsdef{FLOWFRAMTK_OPTS} to pass options to \FlowframTk.

\begin{information}
If you have previously used \gls{JpgfDraw}, the \inlineglsdef{JPGFDRAW_OPTS} 
environment variable will also be recognised, but any settings in 
\gls{FLOWFRAMTK_OPTS} will override those in \gls{JPGFDRAW_OPTS}.
\end{information}

For example, if you always want \FlowframTk\ to start up with the grid showing,
 you can set \gls{FLOWFRAMTK_OPTS} to \switch{show_grid}:
\begin{terminal}
setenv \gls{FLOWFRAMTK_OPTS} \switch{show_grid}
\end{terminal}
(Bear in mind that \gls{JDR} and \gls{AJR} files can have certain
settings saved, which may override the startup settings.)

\begin{important}
Note that these environment variables only have an effect if you
use the \app{flowframtk} script to run the \gls{jre}.
\end{important}

Available command line options are listed below.
Older versions of \FlowframTk\ had switches prefixed with a single
hyphen (such as \shortargfmt{help}). These are still supported for
backward-compatibility but double-hyphen versions are now provided
(such as \longargfmt{help}).

\switchdef{show_grid}
Show the \gls{grid}.

\switchdef{noshow_grid}
Don't show the \gls{grid}.

\switchdef{lock_grid}
Lock the \gls{grid}.

\switchdef{nolock_grid}
Don't lock the \gls{grid}.

\switchdef{toolbar}
Show the \glspl{toolbar}.

\switchdef{notoolbar}
Don't show the \glspl{toolbar}.

\switchdef{statusbar}
Show the \gls{statusbar}.

\switchdef{nostatusbar}
Don't show the \gls{statusbar}.

\switchdef{ruler}
Show the \glspl{ruler}.

\switchdef{noruler}
Don't show the \glspl{ruler}.

\switchdef{paper}
Set the paper \gls{paper.size}. The argument may be a single keyword that
identifies a recognised paper \gls{paper.size} or, for a custom size, the
argument may be the keyword \optfmt{user} followed by the width and
height. For example:
\begin{terminal}
\app{flowframtk} \switch{paper} user 100mm 200mm
\end{terminal}

The custom size must have positive dimensions.
Recognised units: \code{\gls{pt}}, \code{\gls{bp}},
\code{in}, \code{mm}, \code{cm}, \code{pc},
\code{dd} and \code{cc}. If the unit is omitted,
\code{\gls{bp}} is assumed. Examples:
\begin{itemize}
\item \code{\switch{paper} a4r}
\item \code{\switch{paper} user 8.5in 12in}
\item \code{\switch{paper} user 600 1000}
\end{itemize}

\FloatTable
{tab:papersizes}
{%
\begin{tabular}{llll}    
\toprule
\multicolumn{1}{c}{\bfseries Identifier} &
\multicolumn{1}{c}{\bfseries Description} &
\multicolumn{1}{c}{\bfseries Identifier} &
\multicolumn{1}{c}{\bfseries Description}
\tabularnewline 
\midrule
\optfmt{a10}&A10 portrait &    \optfmt{a10r}&A10 landscape\tabularnewline 
\optfmt{a9}&A9 portrait &    \optfmt{a9r}&A9 landscape\tabularnewline
\optfmt{a8}&A8 portrait &    \optfmt{a8r}&A8 landscape\tabularnewline
\optfmt{a7}&A7 portrait &    \optfmt{a7r}&A7 landscape\tabularnewline
\optfmt{a6}&A6 portrait &    \optfmt{a6r}&A6 landscape\tabularnewline
\optfmt{a5}&A5 portrait &    \optfmt{a5r}&A5 landscape\tabularnewline
\optfmt{a4}&A4 portrait &    \optfmt{a4r}&A4 landscape\tabularnewline
\optfmt{a3}&A3 portrait &    \optfmt{a3r}&A3 landscape\tabularnewline
\optfmt{a2}&A2 portrait &    \optfmt{a2r}&A2 landscape\tabularnewline
\optfmt{a1}&A1 portrait &    \optfmt{a1r}&A1 landscape\tabularnewline
\optfmt{a0}&A0 portrait &    \optfmt{a0r}&A0 landscape\tabularnewline
\optfmt{b10}&B10 portrait &    \optfmt{b10r}&B10 landscape\tabularnewline
\optfmt{b9}&B9 portrait &    \optfmt{b9r}&B9 landscape\tabularnewline
\optfmt{b8}&B8 portrait &    \optfmt{b8r}&B8 landscape\tabularnewline
\optfmt{b7}&B7 portrait &    \optfmt{b7r}&B7 landscape\tabularnewline
\optfmt{b6}&B6 portrait &    \optfmt{b6r}&B6 landscape\tabularnewline
\optfmt{b5}&B5 portrait &    \optfmt{b5r}&B5 landscape\tabularnewline
\optfmt{b4}&B4 portrait &    \optfmt{b4r}&B4 landscape\tabularnewline
\optfmt{b3}&B3 portrait &    \optfmt{b3r}&B3 landscape\tabularnewline
\optfmt{b2}&B2 portrait &    \optfmt{b2r}&B2 landscape\tabularnewline
\optfmt{b1}&B1 portrait &    \optfmt{b1r}&B1 landscape\tabularnewline
\optfmt{b0}&B0 portrait &    \optfmt{b0r}&B0 landscape\tabularnewline
\optfmt{c10}&C10 portrait &    \optfmt{c10r}&C10 landscape\tabularnewline
\optfmt{c9}&C9 portrait &    \optfmt{c9r}&C9 landscape\tabularnewline
\optfmt{c8}&C8 portrait &    \optfmt{c8r}&C8 landscape\tabularnewline
\optfmt{c7}&C7 portrait &    \optfmt{c7r}&C7 landscape\tabularnewline
\optfmt{c6}&C6 portrait &    \optfmt{c6r}&C6 landscape\tabularnewline
\optfmt{c5}&C5 portrait &    \optfmt{c5r}&C5 landscape\tabularnewline
\optfmt{c4}&C4 portrait &    \optfmt{c4r}&C4 landscape\tabularnewline
\optfmt{c3}&C3 portrait &    \optfmt{c3r}&C3 landscape\tabularnewline
\optfmt{c2}&C2 portrait &    \optfmt{c2r}&C2 landscape\tabularnewline
\optfmt{c1}&C1 portrait &    \optfmt{c1r}&C1 landscape\tabularnewline
\optfmt{c0}&C0 portrait &    \optfmt{c0r}&C0 landscape\tabularnewline
\optfmt{letter}&Letter portrait &    \optfmt{letterr}&Letter landscape\tabularnewline
\optfmt{legal}&Legal portrait &    \optfmt{legalr}&Legal landscape\tabularnewline
\optfmt{executive}&Executive portrait &    \optfmt{executiver}&Executive landscape\tabularnewline
\bottomrule 
\end{tabular}
}%
[Paper Size Identifiers]
{Paper \gls{paper.size} identifiers for use with \switch{paper} command line switch.}


\switchdef{disable_print}
Disable the request for printer attributes on startup.

\switchdef{nodisable_print}
Don't disable the request for printer attributes on startup (default).

\switchdef{debug}
Enable debugging mode. This will add the \inlineglsdef{menu.debug}
menu to the main menu bar which provides the following functions:
\menu{debug.objectinfo} (displays diagnostic information about the
currently \selected\ objects); \menu{debug.writelog} (writes
diagnostic information for all currently open images to a log file
in the \gls{config_dir}) \menu{debug.revalidate} (revalidates the
image) and \menu{debug.dumpall} (saves all current images to a
subdirectory of the \gls{config_dir}).

\switchdef{nodebug}
Don't enable debugging mode (default).

\switchdef{experimental}
Enable experimental functions for testing purposes. These functions don't
work properly and are not documented.

\switchdef{noexperimental}
Don't enable experimental functions (default).

\switchdef{help}
Display help message and exit.

\switchdef{version}
Display version details and exit.

\section{The Settings Menu}\label{sec:settingsmenu}

\menudef{menu.settings}

While \FlowframTk\ is running, you can change the current settings
using the \menu{settings} menu. Most of the settings will be
remembered next time you use \FlowframTk, but may be overridden
either by \hyperref[sec:cmdargs]{command line arguments} or by
settings specified in any \gls{JDR} or \gls{AJR} file that you load.

If you have selected \widget{initsettings.default} in the
\widget{initsettings.title} tab of the \dialog{config} dialog,
then the \emph{\gls{canvas}} settings will be set to the default
values on startup (unless overridden, as above). Some (but not all)
of the user interface settings may still be remembered from the
previous session.

\subsection{Styles}\label{sec:setcurrentstyles}

\menudef{menu.settings.styles}

The \menu{settings.styles} menu item will open a dialog window
which can be used to set the current \gls{path} and
\gls{textarea} attributes. New paths and text areas will use these
attributes when they are created. The attributes for existing paths
and text areas are changed using the \menu{edit} menu.  These
settings are discussed in more detail in \sectionref{sec:styles}.

\subsection{Show Rulers}\label{sec:showrulers}

\menudef{menu.settings.rulers}

The \menu{settings.rulers} menu item can be used to toggle between showing
and hiding the \glspl{ruler} for the current \gls{canvas}. This
setting will be applied to new \glspl{canvas} that are subsequently
opened, but the setting may be overridden when a new image is
loaded.

\seealsorefs{sec:rulers}

\subsection{Grid}\label{sec:gridmenu}

\menudef{menu.settings.grid}

The \menu{settings.grid} submenu allows you to change
the \gls{grid} settings:

\menudef{menu.settings.grid.show}

The \menu{settings.grid.show} menu item toggles between displaying
the \gls{grid} on the \gls{canvas} and hiding it. If there is
enough memory available, the grid will be stored as a bitmap in
order to improve redraw speed.

\plabel[Locking the Grid]{mi:gridlock}
\menudef{menu.settings.grid.lock}

The \menu{settings.grid.lock} menu item toggles between locking and
unlocking the grid. If the grid is locked, mouse \glspl{click} will
be translated to the nearest tick mark.  This means that if you use
a mouse \gls{click} to set the location of a \gls{controlpt} when
constructing a \gls{path}, the point will be placed at the nearest
tick mark. This also means that when you move a point while in 
\editpathmode, the point will be moved in
intervals of the gap between tick marks. Note that locking the grid
does not affect the keyboard or menu driven functions, such as
\menutrail{menu.navigate.goto} \accelerator{menu.navigate.goto}
or emulate a \gls{primaryclick}
\accelerator{action.construct_click}.


When the grid is locked, the \gls{statusbar} will show the image
\includeimg[alt={Grid lock on}]{key} otherwise it will show
the image \includeimg[alt={Grid lock off}]{keycross}.
You can \gls{doubleclick} on this image to toggle the state.

\begin{warning}
If you lock the grid, you will be unable to use the
mouse to \select\ narrow \glspl{path} that lie between tick marks as
mouse \glspl{click} will be translated to the nearest tick mark, unless you
use the drag rectangle (which may \select\ other \glspl{object} as well).
\end{warning}

Similarly, if the \widget{editpathui.ignorelock}
option is off, you won't be able to select \glspl{controlpt} in
\editpathmode\ with the mouse unless a
tick mark lies inside the \gls{controlpt}['s] bounding box.
You will however be able to select them using the
\menutrail{menu.editpath.next_control} 
or \menutrail{menu.editpath.prev_control} functions.


\subsection{Zoom}\label{sec:zoommenu}

\menudef{menu.settings.zoom}

The \menu{settings.zoom} sub-menu allows you to change
the magnification. You can choose one of the predefined
settings or you can specify an arbitrary setting.

\menudef{menu.settings.zoom.user}

The \menu{settings.zoom.user} menu item will open the
\inlineglsdef{zoom.title} dialog box (see \figureref{fig:zoom}) in
which to specify an arbitrary magnification.  The magnification
value may be entered as either a percentage
(\figureref{fig:zoompercent}) or a decimal value
(\figureref{fig:zoomdecimal}). For example, either 150\% or 1.5 to
zoom by a factor of 1.5.  You can also change the magnification
using the zoom function in the \gls{statusbar}. \Gls{click} on the
decrease button \includeimg{zoomdown} to reduce the magnification or
on the increase button \includeimg{zoomup} to increase the
magnification, according to the list of preset values.
Alternatively, you can \gls{doubleclick} to open the \dialog{zoom}
dialog box or \gls{menuclick} on the percentage value to open
a~popup menu.

\FloatSubFigs{fig:zoom}
 {
   {fig:zoompercent}{\includeimg{setMagnificationPercent}}{Percentage},
   {fig:zoomdecimal}{\includeimg{setMagnificationDecimal}}{Decimal}
 }
 {Setting Custom Magnification}

\subsection{Paper}\label{sec:papermenu}

\menudef{menu.settings.paper}

The \menu{settings.paper} sub-menu allows you to change
the \gls{paper} attributes.

\menudef{menu.settings.paper.margins}

The \menu{settings.paper.margins} item toggles between showing
and hiding the \glspl{printer-margin}, but note that this facility is not
available if you use the \switch{disable_print} command line
argument.

The predefined paper \glspl{paper.size} A0 to A5, letter, legal and executive
can be selected from the \menu{settings.paper} menu.

\menudef{menu.settings.paper.other}

The \menu{settings.paper.other} item allows a custom paper \gls{paper.size} to
be specified.  Select the  \widget{paper.predefined} radio button to
enable a list of additional known paper sizes or select the
\widget{paper.user} radio button to enter a custom size.


\subsection{Configure Image Settings}\label{sec:configuredialog}

\menudef{menu.settings.config}

The \menu{settings.config} menu item opens the \inlineglsdef{config.title}
dialog box, which can be used to change image settings.
(\sectionref{sec:texconfig} covers \TeX\slash\LaTeX\ related settings and
\sectionref{sec:configureuidialog} covers settings related to the user
interface.)


\subsubsection{Control Points}\label{sec:controlsettings}

\widgetdef{controls.title}

\FloatFig
  {fig:configdialog-controlsettings}
  {\includeimg{configureDialog-controls}}
  {Configuration Dialog Box (Control Point Settings)}

The \widget{controls.title} tab
(\figureref{fig:configdialog-controlsettings})
allows you to:
\begin{itemize}
 \item set the size of the \glspl{controlpt};
 \item specify whether the \gls{controlpt} size should be affected by
the zoom setting;
 \item set the \gls{storageunit}.
\end{itemize}

It's best to set the \gls{storageunit} before you start creating an
image. If the selected \gls{canvas} isn't empty, all \glspl{object} will
have their co-ordinates recomputed in terms of the new unit, which
may take a while if there are many \glspl{object} or \glspl{path}
with many segments.  You can also open a dialog window with just the
storage unit panel in it by \glslink{doubleclick}{double-clicking}
the mouse on the storage unit area of the \gls{statusbar}.


\subsubsection{Startup Directory}\label{sec:startdir}

\widgetdef{startdir.title}

\FloatFig
  {fig:configdialog-startdir}
  {\includeimg{configureDialog-startdir}}
  {Configuration Dialog Box (Startup Directory Settings)}

The \widget{startdir.title} tab
(\figureref{fig:configdialog-startdir})
allows you to choose which directory \FlowframTk\ should use as the
current working directory when it starts up. You have a choice of:
\begin{itemize}
\item the current working directory that you were in when you started
up \FlowframTk;
\item the directory you were using when you last used \FlowframTk;
\item a specific directory. In this case, type in the path in the
\widget{startdir.named} box or use the
\widget{startdir.browse} button to select the required
directory.
\end{itemize}


\subsubsection{JDR/AJR Settings}\label{sec:jdrsettings}

\widgetdef{jdr.title}

\FloatFig
  {fig:configdialog-jdrsettings}
  {\includeimg{configureDialog-jdrsettings}}
  {Configuration Dialog Box (JDR/AJR Settings)}

You can use the \widget{jdr.title} tab
(\figureref{fig:configdialog-jdrsettings})
to specify whether or not you want the current
\gls{canvas} settings stored in the \gls{JDR} or \gls{AJR} file when
you save your image. You can also choose whether or not you want to
apply any canvas settings information stored in any \gls{JDR} or
\gls{AJR} file that you load. The canvas settings consist of:
grid show\slash hide, grid locked\slash not locked, rulers
displayed\slash not displayed, the tool in use, the LaTeX normal
size, the \gls{paper} attributes, the grid style, the size of the
\glspl{controlpt} and whether they should be affected by the
magnification setting.


\subsubsection{Startup Settings}\label{sec:initsettings}

\widgetdef{initsettings.title}

\FloatFig
  {fig:configdialog-initsettings}
  {\includeimg{configureDialog-initsettings}}
  {Configuration Dialog Box (Startup Settings)}

You can use the \widget{initsettings.title} tab
(\figureref{fig:configdialog-initsettings})
to choose whether you want \FlowframTk\ to start with its
default settings, or whether to restore the settings from the last
time you used \FlowframTk, or whether to use the settings that are
currently in use.

If you choose the default settings option, the \emph{\gls{canvas}}
settings will be set to the default on startup.  Some (but not all)
of the user interface settings may still be remembered from the
previous session, including the language used by the user interface
and the manual, the paths to the required applications used by the
\filefn{export} functions and the button styles.

Most of the settings are saved in the file \file{flowframtk.conf} in the
\gls{config_dir}.  Mappings,
accelerators, language settings and the recent file list aren't
governed by the startup setting and are stored in separate files in
the configuration directory. When you upgrade to a new version of
\FlowframTk, the accelerators are reset to the default and then the
existing accelerators file is loaded to ensure any new accelerators
are added. Don't modify these files whilst \FlowframTk\ is running.
You may modify them after you have quit \FlowframTk\ using a text
editor, but make sure you save your changes before restarting
\FlowframTk.


\subsubsection{Bitmaps}\label{sec:bitmapconfig}

\widgetdef{bitmapconfig.title}

\FloatFig
  {fig:configdialog-bitmap}
  {\includeimg{configureDialog-bitmap}}
  {Configuration Dialog Box (Bitmap Settings)}

You can use the \widget{bitmapconfig.title} tab
(\figureref{fig:configdialog-bitmap})
to choose whether included \glspl{bitmap} should be
saved using their full path name or a path name relative to the
file being saved. Relative names allow for greater portability, but
if you move the saved file to a different location, you will need to
remember to move the bitmap files relative to the new location or
they won't be found.

You can also use this tab to specify
your preferred default image command, which may be either
\gls{pgfimage} or \gls{includegraphics}.


\subsubsection{Application Paths}\label{sec:processes}

\widgetdef{processes.title}

\FloatFig
  {fig:configdialog-apppaths}
  {\includeimg{configureDialog-apppaths}}
  {Configuration Dialog Box (Application Paths)}

If you \filefn{export} your image to PDF, EPS or SVG
\FlowframTk\ will first save the image as an encapsulated \LaTeX\
document and will then run external applications to create the
desired file format. In order to do this, \FlowframTk\ needs to know
the correct paths to these applications. It will try to find them
from your system's \gls{PATH} environment variable, but if it
can't find them you can use the \widget{processes.title} tab
(\figureref{fig:configdialog-apppaths}) to set their locations.  The
\file{libgs} library is needed by \app{dvisvgm}, so you only
need it if you intend to export to \gls{svg}.

To reduce the chance of zombie processes, each process is run with a
timer that will kill the process if it exceeds the value specified
in the \widget{processes.timeout} box.


\subsection{\TeX\ Settings Dialog}\label{sec:texconfig}

\menudef{menu.settings.configtex}

The \menu{settings.configtex} menu item will open up 
the \inlineglsdef{texconfig.title} dialog box
in which you can adjust the \TeX\slash\LaTeX\ settings.
(\sectionref{sec:configuredialog} covers image settings
and \sectionref{sec:configureuidialog} covers user interface settings.)


\subsubsection{Setting the Document Class and Normal Font Size}
\label{sec:normalsize}

\widgetdef{clssettings.title}

Most of \FlowframTk's \TeX\slash\LaTeX\ related functions (including the
applicable \filefn{export} formats) require a value
corresponding to \gls{normalsize} (the \LaTeX\ command that sets the
\gls{normal-font-size}). A \gls{textarea} needs to know the normal
font size to determine the appropriate font size declaration (see
\sectionref{sec:fontsize}). In addition, both the
\gls{menu.tex.parshape} and \gls{menu.tex.shapepar} functions use
the value of \gls{baselineskip} for the normal font size in order to
determine the location of the scan lines used to compute the
required parameters. This also means that any static or dynamic
frames (\sectionref{sec:flowframe}) that use a non-standard paragraph
shape also require this information.

\FloatFig
 {fig:normalsize}
 {\includeimg{normalsizeandclass}}
 {Setting the Normal Font Size and (optionally) the Class}

\widgetdef{clssettings.normalsize}

The \gls{normal-font-size} can be set using the
\widget{clssettings.title} tab (\figureref{fig:normalsize}).  Select the
required value from the \widget{clssettings.normalsize} \gls{dropdown} and select
\dgls{okay} to set it. Note that you must remember
to use this value in your document. For example, if you set the
normal size as 20, your document will need to use one of the
\sty{extsizes} class files, e.g.\ \cls{extarticle}, and specify
20\gls{pt} as one of the optional arguments:
\begin{codebox}
\cmd{documentclass}[20pt]\marg{extarticle}
\end{codebox}

The largest normal size listed is 25pt (the value is actually
24.88pt, but \FlowframTk\ lists it as 25pt) which is for use with the
\cls{a0poster} class file. Remember that for very large or very
small fonts, you will need to use scalable fonts in your document to
prevent font size substitutions.  Available values, along with the
corresponding value of \gls{baselineskip} and the file in which they
are defined, are listed in \tableref{tab:normalsize}.

\FloatTable
{tab:normalsize}
{%
\begin{tabular}{@{}ccl@{}}
\bfseries Normal size value & \bfseries \gls{baselineskip}
value & \bfseries Relevant File\\
8 & 9.5 & \filefmt{tex/latex/extsizes/size8.clo}\\
9 & 11 & \filefmt{tex/latex/extsizes/size9.clo}\\
10 & 12 & \filefmt{tex/latex/base/size10.clo}\\
11 & 13.6 & \filefmt{tex/latex/base/size11.clo}\\
12 & 14.5 & \filefmt{tex/latex/base/size12.clo}\\
14 & 17 & \filefmt{tex/latex/extsizes/size14.clo}\\
17 & 22 & \filefmt{tex/latex/extsizes/size17.clo}\\
20 & 25 & \filefmt{tex/latex/extsizes/size20.clo}\\
25 & 30 & \filefmt{tex/latex/a0poster/a0poster.sty}
\end{tabular}
}
[Available Values for the Normal Font Size]
{Available values for the normal font size, the corresponding value
and the file in which they are defined (relative to the TEXMF
tree).}

\widgetdef{clssettings.update}

If the \widget{clssettings.update} \gls*{checkbox} is selected,
changing the normal size setting in this dialog will update the
\LaTeX\ font size settings for all the \glspl{textarea} in the
current image.

\widgetdef{clssettings.default_cls}

If you want to export your image to a complete \LaTeX\ document (see
\sectionref{sec:exportdoc}) the \inlineglsdef{document-class} will default to:
\begin{itemize}
\item \cls{article}, if the normal size is
set to 10, 11 or 12;
\item \cls{a0poster}, if the normal size is set to
25;
\item \cls{extarticle}, otherwise.
\end{itemize}

\widgetdef{clssettings.custom_cls}

If you want to use a different \gls{document-class}, select the
\widget{clssettings.custom_cls} radio button, which will enable the
field where you can type the require class name (without the
\filefmt{.cls} extension). You need to ensure the class supports your
specified normal font size (as a standard option, not a key=value
option).

The \gls{document-class} is also used as the base class with the export to
document class function (see \sectionref{sec:exportsty}) and is used
for all \filefn{export} functions that use the export to encapsulated \LaTeX\
document function as an intermediate step.

\widgetdef{clssettings.relative_fontsize}

The \widget{clssettings.relative_fontsize} \gls*{checkbox} governs
the suggested \LaTeX\ font size command when you change the
\hyperref[sec:fontsize]{font size setting} for \glspl{textarea}. If
this \gls*{checkbox} is selected, \FlowframTk\ will attempt to select a
font size declaration, such as \gls{large}, according to the current
normalsize setting. Otherwise, \gls{setfontsize} will be used. If
the font size is significantly larger than the largest available
declaration, \gls{setfontsize} will be used regardless of this
setting.

\widgetdef{clssettings.pdfinfo}

The \widget{clssettings.pdfinfo} \gls*{checkbox} governs whether or
not to add \gls{pdfinfo} to exported \LaTeX\ documents. If the image
has a description (\sectionref{sec:imagedescription}), this will be
added to the \gls{pdfinfo/Title} attribute. The only other attribute
that is set is the \gls{pdfinfo/CreationDate} attribute. This setting is
not used for the export to PostScript or \gls{svg} functions as they
use \app{latex} rather than \app{pdflatex}.

\widgetdef{clssettings.use_typeblock_as_bbox}

When an image is exported to an encapsulated \LaTeX\ document
or complete \LaTeX\ document, by default the image \gls{image.bbox}
is calculated as the minimal sized rectangle that encompasses all
\glspl{object} on the \gls{canvas}. If the image contains an
\glspl*{textarea}, this may result in text being clipped to the left
or right of the image due to the font differences. In this case, or
if you require a different bounding box for some other reason, then
switch on the \widget{clssettings.use_typeblock_as_bbox}
\gls*{checkbox} and if the image has a \gls{image.typeblock} set
(see \sectionref{sec:typeblock}) the \gls{image.typeblock} will be used as the image
\gls{image.bbox}. If the image doesn't have a \gls{image.typeblock} set, then the
default bounding box will be used. See also \figureref{fig:equation}
in \sectionref{sec:exportdoc}.


\subsubsection{Flowframe Settings}\label{sec:texconfigflf}

\widgetdef{flfsettings.title}

\FloatFig
 {fig:flfsettings}
 {\includeimg{flfsettings}}
 {Flowframe Settings Tab}

The \widget{flfsettings.title} tab
(\figureref{fig:flfsettings}) allows you to adjust the
\sty{flowfram} related settings.

The \sty{flowfram} package now allows you to use absolute pages
numbers in the page list rather than using the value of the
\ctr{page} counter. For example, if your document pages are numbered
a, b, i, ii, iii, 1, 2, 3, then with the absolute setting page
number 3 refers to the page \manmsg{labelled} i whereas with the
relative setting page number 3 refers to both the page
\manmsg{labelled} iii and the page \manmsg{labelled} 3. (See the
\sty{flowfram} documentation for further details.) If you want to
use the absolute setting, select the
\widget{flfsettings.pages_opt.absolute} button otherwise select the
\widget{flfsettings.pages_opt.relative} button.

The \sty{shapepar} package now has a command called
\gls{Shapepar} that behaves slightly differently to
\gls{shapepar}. Since \gls{Shapepar} works better than
\gls{shapepar} when used with the \sty{flowfram} package,
\FlowframTk\ defaults to using \gls{Shapepar} for the
\gls{menu.tex.shapepar} function or when a
\hyperref[sec:flowframe]{static or dynamic frame} has the
shape set to \widget{flowframe.shape_shapepar}. However, if
you prefer the original \gls{shapepar} command, selected the
appropriate \widget{flfsettings.shapeparcs} radio button.
If you want to use \gls{Shapepar}, you need to ensure you have
up-to-date versions of the \sty{flowfram} and \sty{shapepar}
packages. This setting applies both to frames with the
\widget{flowframe.shape_shapepar} shape setting and to the
\gls{menu.tex.shapepar} function.

\subsubsection{Text Settings}\label{sec:texconfigtext}

\widgetdef{textconfig.title}

\FloatFig
  {fig:configdialog-texsettings}
  {\includeimg{configureDialog-texsettings}}
  {LaTeX-Related Text Settings}

The \widget{textconfig.title} tab
(\figureref{fig:configdialog-texsettings})
allows you to adjust the text-related settings.

\widgetdef{textconfig.anchor}

If the \widget{textconfig.anchor} \gls{checkbox} is selected,
whenever you \hyperref[sec:alignobjects]{justify} a \gls{group}, any
\glspl{textarea} in that group will have their
\textattrpl{anchor} automatically adjusted.

\widgetdef{textconfig.textualshading}

\FlowframTk\ can't implement \dgls{gradient-paint} for text when
\hyperref[sec:exportpgf]{exporting} to \sty{pgf}. (This includes the
\filefn{export} functions that use \app{latex} or \app{pdflatex}, such as
the export to PDF function.) You can choose how \FlowframTk\ should export
shaded text using the \widget{textconfig.textualshading} dropdown
box. Available options:

\begin{deflist}
\itemtitle
 {\inlineglsdef{textconfig.textualshading.average}}

\begin{itemdesc}
The text will be given the \manmsg{colour} obtained by averaging the
shading's start and end \manmsg{colour}.
\end{itemdesc}

\itemtitle
 {\inlineglsdef{textconfig.textualshading.start}}

\begin{itemdesc}
Just the shading's start \manmsg{colour} will be used.
\end{itemdesc}

\itemtitle
 {\inlineglsdef{textconfig.textualshading.end}}

\begin{itemdesc}
Just the shading's end \manmsg{colour} will be used.
\end{itemdesc}

\itemtitle
 {\inlineglsdef{textconfig.textualshading.path}}

\begin{itemdesc}
The text will be exported as a path rather than text. This means
that the \LaTeX\ alternative text attribute will be ignored. This is
equivalent to applying the 
\hyperref[sec:converttopath]{convert to path}, 
\hyperref[sec:grouping]{ungroup} and 
\hyperref[sec:mergepaths]{merge path}
functions (without actually changing your image).
\end{itemdesc}

\end{deflist}

\widgetdef{textconfig.textpathoutline}

\FlowframTk\ can't implement the text
\hyperref[sec:textoutline]{outline} option for
\glspl{textpath} when exporting to \sty{pgf}. (Again, this
includes the export functions that use \app{latex} or \app{pdflatex}.)
You can choose how \FlowframTk\ should export \gls{textpath}
outlines using the \widget{textconfig.textpathoutline}
dropdown box. Available options:

\begin{deflist}
\itemtitle
 {\inlineglsdef{textconfig.textpathoutline.path}}

\begin{itemdesc}
The \gls{textpath} will be exported as a path rather than a
\gls{textpath}. This means that the \LaTeX\
alternative text attribute will be ignored. This is equivalent to
applying the
\hyperref[sec:converttopath]{convert to path}, 
\hyperref[sec:grouping]{ungroup} and 
\hyperref[sec:mergepaths]{merge path}
functions, without actually changing your image.
(This option will always override the shading options described
above, if the \gls{textpath} has a shading.)
\end{itemdesc}

\itemtitle
 {\inlineglsdef{textconfig.textpathoutline.ignore}}

\begin{itemdesc}
The outline attribute will be ignored.
\end{itemdesc}

\end{deflist}

\iconstartpar{textarea,math}%
\plabel[TeX Mappings]{mi:texmappings}%
\FlowframTk\ has two tools for creating \glspl{textarea}: the
regular text-mode tool (\menu{tools.textarea}) and the
\manmsg{maths}-mode tool (\menu{tools.math}). Each \gls{textarea} has an
associated \LaTeX\ alternative text which, if set, is used
during the \sty{pgf} export operations in place of the
text displayed on the \gls{canvas}. (This includes the export
functions that use \app{latex} or \app{pdflatex}, such as
the export to PDF function.) The \mathstool\ automatically
inserts \glsname{mshift} at the start and end of the
alternative text. In addition, Unicode symbols present in
the text when a new \gls*{textarea} is constructed can be mapped to
a \LaTeX\ command.

The mappings applied depend on which tool has been selected. With
the regular \texttool, all the ten \TeX\ special characters are
mapped to commands. In addition, there are some other characters
that are also mapped by default. These are listed in the table shown
in the \widget{textconfig.textmode} tab (see
\figureref{fig:configdialog-texsettings}). If you don't want any
text mappings applied, deselect the
\widget{textconfig.escape} \gls{checkbox}.

With the \mathstool, the \gls{hash} and \gls{percent} characters are the
only two of the ten \TeX\ special characters that are mapped by
default. However there are other mappings of \manmsg{maths}-related Unicode
characters and these are listed in the table shown in the
\widget{textconfig.mathmode} tab (see
\figureref{fig:configdialog-mathsettings}). If you don't want
\mathsmode\ mappings applied, deselect the
\widget{textconfig.escape.mathchar} \gls*{checkbox}.

\FloatFig
  {fig:configdialog-mathsettings}
  {\includeimg{configureDialog-mathsettings}}
  {LaTeX-Related Text Settings\dash \Manmsg{maths}-Mode Mappings}

The mapping tables can be sorted by \glslink{click}{clicking} on
their column headers. You can add, delete or modify any of these
\textmode\ or \mathsmode\ mappings. Any mapping that ends with a control
word (a backslash followed by one or more letters) should
usually be followed by a space or \verb|{}| to prevent it from
running into any subsequent letter when the mapping is applied. For
\textmode\ mappings, \verb|{}| is better in case the command is
followed by an intended space, although this may depend on the
command definition. For \mathsmode\ mappings it's better
to follow the control word with a space.

One or more rows in a mapping table can be selected with the mouse.
You can use \keys{\keyref{shift}}+\gls{click} or 
\keys{\keyref{ctrl}}+\gls{click} to add to the
current selection.

\widgetdef{textconfig.remove.textmap}

To delete a mapping or mappings, select the appropriate row or rows of the
mapping table and click on the \widget{textconfig.remove.textmap} button
to the side of the table.  To restore all the original mappings,
quit \FlowframTk\ and delete or rename the files
\file{mathmappings.prop} and \file{textmappings.prop} in the
\gls{config_dir}.

To edit an existing mapping, \gls{doubleclick} on the second and
third columns of the mapping table. The first two columns can't be
edited.

\widgetdef{textconfig.add.textmap}
\plabel[New \TeX\ Mappings]{mi:newtexmappings}% HelpSet id
To add a new mapping, \gls{click} on the green plus button to the
side of the relevant table.  This will open the add new mapping
dialog. Enter the character you want mapped in the
\widget{textconfig.symbol} box or type the hexadecimal value in the
\widget{textconfig.codepoint} box and then enter the replacement
command in the \widget{textconfig.command} box.  If the command is
defined in a package, enter the package name (without the \ext{sty}
extension) in the \widget{textconfig.package} box. If the mapping
requires multiple commands from different packages, you can specify
a comma-separated list of packages in the
\widget{textconfig.package} box. If the replacement command is
available in the \LaTeX\ kernel, you can leave this box blank.
(These packages will be automatically added to the
\hyperref[sec:preamble]{image's early-preamble} code, whenever the
mapping is applied.)

The package name may be preceded by optional arguments in square
brackets, for example \code{[weather]ifsym} (note that the package name
isn't enclosed in braces). Set the package name to \qt{none} or
leave it empty if you don't want it automatically added to the early-preamble
code when the mappings are applied.

For example, in \figureref{fig:addTeXMapping} the new mapping dialog
is being used to define a new text mapping from the Unicode dagger
\textdagger\ symbol (U+2020) to the \csfmt{textdagger} command.
Since this command is defined in the \LaTeX\ kernel, the package
field can be left empty. The Unicode symbol can be specified either
by typing the hexadecimal code into the
\widget{textconfig.codepoint} field or by typing the actual
symbol into the \widget{textconfig.symbol} field. If there
is already a mapping for this Unicode symbol, the old mapping will
be overwritten.

\FloatFig
  {fig:addTeXMapping}
  {\includeimg{addTeXMapping}}
  {Adding a New Text Mapping} 

\widgetdef{textconfig.textmappings.import}
You can also import mappings from a tab-separated file (\ext{tsv})
where the first column is the hexadecimal code, the second column is
the command (or commands) and the third column is the package name
or comma-separated list of packages. Any additional columns are
ignored, as are blank lines or lines starting with \gls{hash}. The file
shouldn't contain any delimiters nor may any of
the cells contain a newline or tab character. For example,
the file may look like:
\begin{verbatim}
00260E  \Telephone      [misc]ifsym
0026A1  \Lightning      [weather]ifsym
\end{verbatim}

You can find extra mappings using the Symbol Lookup page at
\url{http://www.dickimaw-books.com/latex/symbol-lookup.php} which
has a web form you can use to look up mappings in the database.  If
the mapping ends with a control word the import function will
automatically append a space for \mathsmode\ or
\verb|{}| for \textmode, unless the Unicode character type is one
of: combining spacing mark, connector punctuation, modifier symbol
or modifier letter, in which case just a space is appended.
(This adjustment isn't done for the edit mapping or
add a single-mapping functions described above. You need to do this
yourself, as I did in \figureref{fig:addTeXMapping}.) If this
adjustment isn't appropriate, you can edit the mapping to remove it.

For example, in \figureref{fig:symbolLookup}, I've selected the
\qt{Latin-1 Supplement} block with the mode set to \qt{text}. If I
then click on the \qt{Search} button at the bottom of the form, this
will list all the mappings listed in the database that match. To
import these values into \FlowframTk, you first need to download
them in the correct format. Set the \qt{Results Format}
\gls{dropdown} to TSV (\emph{not CSV}) and click on the
\qt{Search} button. Depending on your browser, this may
automatically save the file \filefmt{symbol-lookup.tsv} in your
downloads directory\slash folder or it may try to open the file.
Be careful if it tries to open the file in a spreadsheet application
as this may change the format to one that's not compatible with
\FlowframTk.

\FloatFig
  {fig:symbolLookup}
  {\includeimg[width=0.9\textwidth]{symbol-lookup}}
  {Symbol Lookup Script}

If you try this search on the Symbol Lookup page and open the file in your
\manmsg{favourite} text editor, you may notice that some codepoints are listed
multiple times. For example:
\begin{verbatim}
0000A3  \textsterling   none    text    0
0000A3  \pounds         none    both    0
\end{verbatim}
If you simply import the file into \FlowframTk, new mappings will
override the earlier mappings, so in this case the mapping for
U+A3 will be set to \csfmt{pounds} rather than
\csfmt{textsterling} so delete any of the mappings you don't want
before you import the file.

This example also produces mappings that require the \sty{fontenc}
package. For example:
\begin{verbatim}
0000AB  \guillemotleft  fontenc text    0
\end{verbatim}
This indicates commands that are part of the \LaTeX\ kernel but don't
work with the default OT1 font encoding. In my case, I've set my
\hyperref[sec:texconfigpreamble]{default preamble} to include
\sty{fontenc} with the T1 encoding, so I don't want that package
automatically added to my early-preamble code. To fix this, I just need to
use my text-editor's search-and-replace function to replace all instances of
\qt{fontenc} with \qt{none} before I import this file. If you don't
have this package in the default preamble, you'll need to add the appropriate
option.  For example, change the line to:
\begin{verbatim}
0000AB  \guillemotleft  [T1]fontenc     text    0
\end{verbatim}
(Again, you can use your text-editor's search-and-replace function.)
Similarly, if you want to import any of the symbol maps that include
commands from the \sty{mathdesign} package, you'll need to add
the option to set the required font. For example
\begin{verbatim}
002231  \intclockwise   [utopia]mathdesign      math    1
\end{verbatim}

Once all these modifications have been made, the file can then be
imported by clicking on the import button in the appropriate mapping
pane. In this case, I have fetched mappings that are valid in
\textmode, so I need to select the \widget{textconfig.textmode} tab
and then click on the import button to the right of the mapping
table. This will open a file chooser which I can use to select the
TSV file.

\begin{warning}
Note that mappings take up resources and the more mappings you have,
the longer it will take to startup \FlowframTk, so it's best to only
add the mappings that you're likely to need. The symbol lookup form
on the Dickimaw Books website has a maximum limit of 500 for the search
results. (A smaller limit can be set if required. The default value
is 200.)
\end{warning}

\subsubsection{Default Preamble}\label{sec:texconfigpreamble}

\widgetdef{preambleconfig.title}

The \widget{preambleconfig.title} tab
(\figureref{fig:texconfig-preamble}) allows you to specify code that
should always be added to the preamble when exporting images to
complete \LaTeX\ documents or formats that use the encapsulated
\LaTeX\ document function as an intermediate step (see
\sectionref{sec:exportdoc}). This code isn't used for the export to
class (\ext{cls}) or package (\ext{sty}) functions.

\FloatFig
  {fig:texconfig-preamble}
  {\includeimg{texconfig-preamble}}
  {Default Preamble Code}

The default preamble code is stored in a file called
\file{preamble.tex} in \FlowframTk's \gls{config_dir}. If you edit this
file outside of \FlowframTk, you'll need to use the reload button to
refresh this panel. The other buttons are the same as for the \TeX\
editor described in \sectionref{sec:texeditorui}.  You can also
access these button actions through the context menu, which can be
opened using a~\gls{menuclick} on the editor pane.


\subsection{Configure User Interface}\label{sec:configureuidialog}

\menudef{menu.settings.configui}

The menu item \menu{settings.configui} will open up 
the \inlineglsdef{configui.title} dialog box
in which you can adjust the user interface settings.
(\sectionref{sec:configuredialog} covers image settings
and \sectionref{sec:texconfig} covers \TeX\slash\LaTeX\ related
settings.)


\subsubsection{Graphics Settings}\label{sec:graphics}

\widgetdef{graphics.title}

\FloatFig
  {fig:graphics}
  {\includeimg{configureUI-graphics}}
  {Configuration UI Dialog Box (Graphics Settings)}

The \widget{graphics.title} tab (\figureref{fig:graphics})
allows you to make adjustments to the way the image is rendered on
the \gls{canvas}.

\widgetdef{render.anti_alias}

If enabled, use \gls{anti-aliasing} to draw the image.

\widgetdef{render.rendering}

Choose between speed or quality in the \gls{rendering}.

\widgetdef{editpathui.title}

The \widget{editpathui.title} area has the following settings that
relate to the \editpathmode.

\widgetdef{editpathui.canvasclick}

If enabled, this setting will cause a mouse \gls{click} on the
\gls{canvas} outside of any of the \glspl{controlpt} to exit
\editpathmode.

\widgetdef{editpathui.ignorelock}

If enabled, the \gridlock\ will be ignored when using the
mouse to select a \gls{controlpt} in \editpathmode.

\widgetdef{render.control.standard}

Set the \manmsg{colour} for the regular
\glspl*{controlpt} according to whether they are selected or
unselected.

\widgetdef{render.control.symmetry}

Set the \manmsg{colour} for the \gls{line-of-symmetry}
\glspl*{controlpt} according to whether they are selected or
unselected.

\widgetdef{render.control.patternanchor}

Set the \manmsg{colour} for the pattern \gls{pattern.anchor}
\glspl*{controlpt} according to whether they are selected or
unselected.

\widgetdef{render.control.patternadjust}

Set the \manmsg{colour} for the pattern 
\glspl*{pattern.adjustment-controlpt} according to whether they are selected or
unselected.

\plabel[Hotspots]{mi:hotspots}%
\widgetdef{hotspots.title}

\iconstartpar{hotspots.png,nohotspots.png}%
\Inlineglspluraldef{hotspot} are regions around an \gls*{object}['s] 
\gls{selection-bbox} that can be used to perform freehand
transformations.
You can use the \widget{graphics.title} tab to choose between
enabling (left icon) and disabling (right icon) \glspl{hotspot} along the
\glspl{selection-bbox}. If \glspl{hotspot} are enabled, you can scale, rotate or
shear \glspl{object} by dragging the appropriate \gls{hotspot}. (Note that
transforming \glspl{compositeshape} applies the transformation to
the \emph{underlying} shape not to the shape as a whole.)

You may want to disable this option when you want to move small
objects, or you may end up transforming them instead of moving them.
When this option is enabled, the cursor will change
shape when you move it over the edge of the \gls{selection-bbox}.
(The actual cursor appearance depends on the look and
feel of the platform you are using. On some systems the South and
North arrows may look the same, and similarly for the East and West
arrows.)
\figureref{fig:hotspotsa} shows how the \gls{bbox} is displayed
when \glspl{hotspot} are enabled and \figureref{fig:hotspotsb} shows
how the \gls{selection-bbox} is displayed when the \glspl{hotspot} have been
disabled. Each \gls{hotspot} is represented by a small square. Available
functions are listed in \tableref{tab:hotspotfunctions}.

\FloatTable
{tab:hotspotfunctions}
{%
\begin{tabular}{lll}
\bfseries Hotspot & \bfseries Function &
\bfseries Cursor Appearance\\
Bottom left & rotate & hand\\
Bottom \manmsg{centre} & scale vertically & South arrow\\
Bottom right & scale both directions & South-East arrow\\
Middle right & scale horizontally & East arrow\\
Top right & shear vertically & North arrow\\
Top left & shear horizontally & West arrow
\end{tabular}
}
{Hotspot Functions}
 
Note that even if you have more than one \gls{object} \selected, only the
object whose \gls{hotspot} you are dragging will be transformed.
As may be predicted, using \glspl{hotspot} is not as precise as using the
transformation dialog boxes described in
\sectionref{sec:selectobjects}.

\FloatSubFigs{fig:hotspots}
{
 {fig:hotspotsa}{\includeimg{exhotspots}}{Enabled},
 {fig:hotspotsb}{\includeimg{exnohotspots}}{Disabled}
}
{Hotspots}

\seealsorefs{sec:rotateobjects,sec:scaleobjects,sec:shearobjects}

\subsubsection{Annotations Settings}\label{sec:annotations}

\widgetdef{annotations.title}

\FloatFig
  {fig:annotations}
  {\includeimg{configureUI-annotations}}
  {Configuration UI Dialog Box (Annotations Settings)}

The \widget{annotations.title} tab (\figureref{fig:annotations})
allows you to set the font used for annotating frames or draft
\glspl{bitmap} and set the font used on the splash screen during startup.


\subsubsection{Language Settings}\label{sec:languages}

\widgetdef{lang.title}

\FloatFig
  {fig:languages}
  {\includeimg{configureUI-languages}}
  {Configuration UI Dialog Box (Language Settings)}

You can use the \widget{lang.title} tab
(\figureref{fig:graphics})
to set which language to use for the application
resources (menus, messages etc) and which language to use for
the manual. These settings will not be applied until you quit and
restart \FlowframTk.

You can also use the \widget{lang.title} tab to specify the
Unicode blocks to display in the
\hyperref[mi:insertsymbol]{symbol selector}. By default, only
a limited subsection of Unicode characters are available because it
would significantly slow \FlowframTk's startup process to provide
all possible characters. To add a new block, \gls{click} on the
green plus button next to the table. To remove an unwanted block,
select the appropriate row in the table and \gls{click} on the red minus
button. The start and end values don't need to match a complete
Unicode block and may span multiple blocks. For example, suppose you
regularly want to enter musical symbols into your image, then you
can add a block starting from 1D100 and ending at 1D1DD. You need to
quit and restart \FlowframTk\ for the changes to take effect. If you
want to restore the defaults, quit \FlowframTk, open the file
\file{flowframtk.conf} in the \gls{config_dir} and remove
the line starting with \qt{unicode=}.

\seealsorefs{sec:multilingualsupport}

\subsubsection{Accelerator Settings}\label{sec:accelerators}

\widgetdef{accelerators.title}

\FloatFig
  {fig:accelerators}
  {\includeimg{configureUI-accelerators}}
  {Configuration UI Dialog Box (Accelerators)}

You can use the \widget{accelerators.title} tab
(\figureref{fig:accelerators}) to change the default accelerators
(keyboard shortcuts). It's possible to use the same keystroke for
different actions provided the actions are never both enabled at the
same time. (For example, \accelerator{menu.editpath.next_control}
selects the next control in path edit mode or adjust the selection
in \selectmode.)

\begin{information}
Java's accessibility API uses \accelerator{action.menubarfocus} to switch
the focus to the menu bar, so avoid using this key for an
accelerator.
\end{information}

Suppose you want to change the accelerator for the \menu{edit.undo}
function from the default \accelerator{menu.edit.undo} to, say,
\keys{\keyref{ctrl}+\keyref{shift}+U}. Find the appropriate row in
the accelerator table and \gls{doubleclick} on it to open the
\widget{accelerator.set} dialog box
(\figureref{fig:setaccelerator}).

\FloatFig
  {fig:setaccelerator}
  {\includeimg{setaccelerator}}
  {Changing a keyboard shortcut.}

There are two ways to change the
keystroke. If the \widget{accelerator.set.useselector}
button is selected, you can select or deselect the required
modifiers (such as \widget{accelerator.set.shift}) and
use the drop-down box to select the keystroke. So to change the
keystroke to \keys{\keyref{ctrl}+\keyref{shift}+\actualkey{U}}, select the
\widget{accelerator.set.shift} and
\widget{accelerator.set.ctrl} \glspl{checkbox} and set the drop-down
box to \widgetfmt{pressed U}.

Alternatively, if the \widget{accelerator.set.usekeystroke} button
is selected you can type the required keystroke in the
\widget{accelerator.set.usekeystroke.label} field. Care must be
taken with this option. You need to make sure you release the main
key before releasing the modifiers. For example, press down the
\desckeyref{ctrl}, \desckeyref{shift} and \keys{\actualkey{U}} keys
but release the \keys{\actualkey{U}} key before releasing the
\keys{\keyref{ctrl}} and \keys{\keyref{shift}} keys. 

Note that the \desckeyref{tab} key retains its usual
function of moving the focus to the next component in the window,
and so can't be typed in the
\widget{accelerator.set.usekeystroke.label} field. (Neither
can certain other keys that are always intercepted by the operating
system.) The \desckeyref{return} key, on the other hand, will be picked up
by the \widget{accelerator.set.usekeystroke.label} field
if it has the focus, so you need to ensure you move the focus to a
different component if you want to use \keys{\keyref{return}} to activate the
\btn{okay} button.

The accelerator settings are written to a file called
\inlineglsdef{file.accelerators.prop} in the \gls{config_dir} when
\FlowframTk\ quits normally. You may edit this file using your
preferred text editor to change the settings as long as \FlowframTk\
isn't currently running. For example, if you want to change all the
keystrokes that require the control key to be pressed to requiring
the meta key pressed instead, you may find it easier to do a global
search and replace in \file{accelerators.prop} of \code{ctrl}
with \code{meta}. If you do edit this file, make sure you only
edit the values to the right of the \code{=} sign. To restore the
default accelerators, delete or rename this file (when \FlowframTk\
isn't running).

\subsubsection{Toolbar, Ruler and Status Bar Settings}\label{sec:rulers}

\widgetdef{borders.title}

\FloatFig
  {fig:rulers}
  {\includeimg{configureUI-rulers}}
  {Configuration UI Dialog Box (Toolbar, Ruler and Status Bar Settings)}

You can use the \widget{borders.title} tab (\figureref{fig:rulers})
to customise the \glspl{ruler}, \glspl{toolbar} and \gls{statusbar}.
You can show or hide the \glspl{toolbar} using the
\widget{borders.showtools} checkbox. If this check box is selected,
you can also specify the location of the vertical toolbar.

\iconstartpar{toolbarleft,toolbarright}%
The vertical toolbar is located on the left of \FlowframTk's main
window by default, but can be switched to the right by selecting the
\widget{borders.vtoolbar} \widget{borders.vtoolbar.right} radio
button. You will need to quit and restart for the change to take
effect.

The width (in pixels) of the vertical ruler can be specified in the
\widget{borders.rulerwidth} field, and the height (in pixels) of
the horizontal ruler can be specified in the
\widget{borders.rulerheight} field. For example, if you find that
the $y$ co-ordinates don't fit on the vertical ruler, you can make the
ruler wider, say, to 30 pixels. There's a sample panel on the right
that shows the dimensions (but doesn't show the ruler annotations).

The way the co-ordinates appear on the rulers is governed by the
\widget{borders.rulerpattern} field.  The pattern syntax is as for
\href{https://docs.oracle.com/javase/8/docs/api/java/text/DecimalFormat.html}{\filefmt{java.text.DecimalFormat}},
described in the Java API.  You can also set the locale governing
the pattern, but this pattern is only applied to the \glspl{ruler},
not to the co-ordinate dialog boxes or \gls{statusbar}. The font
used by the rulers can also be changed.  This panel doesn't provide
a setting to show or hide the rulers as this is done for the
currently selected \gls{canvas} using the main menu item
\menu{settings.rulers}.

You can show or hide the \gls{statusbar} using the
\widget{borders.showstatus} check box. If this box is
selected, you can additionally choose to show or hide the
individual \gls{statusbar} elements.

\seealsorefs{sec:thetoolbars,sec:therulers,sec:thestatusbar}

\subsubsection{Normalization}\label{sec:normalize}

\widgetdef{normalize.title}

\FloatFig
  {fig:normalize}
  {\includeimg{configureUI-normalize}}
  {Configuration UI Dialog Box (Normalization)}

When an image is \glslink{painting}{drawn} on the \gls{canvas} via Java's painting
methods, the co-ordinates of all the objects are converted into
PostScript points (including any scaling applied by the zoom
settings). However, it is rare for any display device, such as a
standard monitor, to have pixels that are exactly one PostScript point square
in size. This means that a 1in (or 72\gls{bp}) line drawn on the
screen at 100\% magnification may not actually measure 1in if you
hold a ruler up against the screen. This is fairly typical for
most graphical applications but, if it bothers you, you can
adjust the horizontal and vertical normalization factors used by
\FlowframTk\ in the \widget{normalize.title} tab
(\figureref{fig:normalize}).

If you happen to already know the normalization factors
for your device, you can enter them in the
\widget{normalize.norm.x} and
\widget{normalize.norm.y} fields. If you don't know them,
\FlowframTk\ can compute them for you. To do this, you need to hold
a ruler against the \widget{normalize.nonorm_horizontal}
line, enter the measurement in the length field below the line and
\gls{click} on the \widget{normalize.x_compute_norm} button. This
will insert the normalization factor into the
\widget{normalize.norm.x} field. Next, hold the ruler
against the \widget{normalize.nonorm_vertical} line, enter
the measurement in the length field below that line and \gls{click} on the
\widget{normalize.y_compute_norm} button. This will insert
the normalization factor into the \widget{normalize.norm.y}
field. You then need to \gls{click} on the okay button to set these
normalization factors.

\subsubsection{\TeX\ Editor Settings}\label{sec:texeditorui}

\widgetdef{texeditorui.title}

\FloatFig
  {fig:texeditorui}
  {\includeimg{configureUI-texeditorui}}
  {Configuration UI Dialog Box (TeX Editor Settings)}

\FlowframTk\ has small text editor that's opened when you want to
edit the contents of a static or dynamic frame.
The font used by this editor can be changed in the
\widget{texeditorui.title} tab (\figureref{fig:texeditorui}).
You can also set the default width (in terms of approximate number
of characters per line) and the default height (in terms of lines)
for the editor when it's first created at startup. You can also
enable or disable syntax highlighting by selecting or deselecting
the button marked \widget{texeditorui.highlight}. If
selected, you can also set the \manmsg{colour} used for comments
(indicated by \TeX's \glssymbol{percent} comment character) or for commands. The
editor currently doesn't have any spell-checking support.
The \hyperref[sec:texconfigpreamble]{default preamble} panel
of the \dialog{texconfig} dialog also
uses these settings.

In \FlowframTk\ version 0.7, this editor window was also used for changing the
preamble text, but as from version~0.8, the preamble editor has been
switched to a pane attached to the \gls{canvas} that can remain open
while you're editing the image. This pane uses these settings,
except for the default width and height, which is now governed
by the size of the \gls{canvas} window and the location of the
divider. By default, the preamble pane is to the right of the
\gls{canvas}, but you can change this using the radio buttons in the
\widget{texeditorui.split} area.

\seealsorefs{sec:thepreamblepane,sec:flowframe}

\subsubsection{Look and Feel Settings}\label{sec:lookandfeel}

\widgetdef{lookandfeel.title}

\FloatFig
  {fig:lookandfeel}
  {\includeimg{configureUI-lookandfeel}}
  {Configuration UI Dialog Box (Look and Feel Settings)}

Java displays \gls{gui} elements, such as buttons and menus,
according to the selected \qt{\gls{look-and-feel}}. There are a number of
different \glspl{look-and-feel} that may, or may not, be available on your
platform. Some of the \glspl{look-and-feel} don't support certain features
such as the click-to-collapse setting on the divider bar in split
panes, including the divider between the \gls{canvas} and the preamble
pane. The \widget{lookandfeel.title} tab has a
\gls{dropdown} with the list of available \gls{look-and-feel}. For
example, \figureref{fig:lookandfeel} has the \qt{Metal} look and
feel selected. You can change this to a different \gls{look-and-feel}, but
you must restart \FlowframTk\ for the change to be implemented.

Some example \glspl{look-and-feel} are shown in \figureref{fig:lookandfeelex}.
Note that the main window's title bar and outer border isn't
governed by Java's \gls{look-and-feel}, but by the operating
system's window manager. For example, I've used different window
themes whilst taking snapshots of \FlowframTk\ for this manual,
which is why the \manmsg{colour} and format of the title bar for the main
window and for the dialog boxes vary.

\FloatSubFigs{fig:lookandfeelex}
{
  {fig:lookandfeelex-metal}
    {\includeimg[width=0.45\linewidth]{lookandfeel-metal}}{Metal},
  {fig:lookandfeelex-nimbus}
    {\includeimg[width=0.45\linewidth]{lookandfeel-nimbus}}{Nimbus},
  {fig:lookandfeelex-cdemotif}
    {\includeimg[width=0.45\linewidth]{lookandfeel-cdemotif}}{CDE\slash Motif},
  {fig:lookandfeelex-gtk}
    {\includeimg[width=0.45\linewidth]{lookandfeel-gtk}}{GTK+}
}
{Look and Feel Examples}

In addition to changing the look and feel, you can also change how
the buttons that have icons are displayed. By default, the buttons
with icons have a bevel bordered style, with a different icon image for
the up, down, rollover and disabled states, but if you don't like
them, you can select a different type from the
\widget{lookandfeel.button.style} area. Each type has some sample
buttons displayed on the right. They don't perform any actions, but
you can press them to try them out. The button types come in the
following \manmsgpl{flavour}:

\begin{deflist}
\itemtitle{Bordered}

\begin{itemdesc}
This is the default button type described above.
These buttons have distinctive up and down states regardless of the
\gls{look-and-feel}.
\end{itemdesc}

\itemtitle{Small Bordered}

\begin{itemdesc}
Similar to bordered but a smaller version.
\end{itemdesc}

\itemtitle{Plain}

\begin{itemdesc}
This has a single icon image for the button. The up and
down look is dealt with by the \gls{look-and-feel}. This means that if the
selected look and feel doesn't draw an up or down effect, you won't
be able to see the button state.
\end{itemdesc}

\itemtitle{Small Plain}

\begin{itemdesc}
Similar to plain but a smaller version.
\end{itemdesc}

\itemtitle{Highlights}

\begin{itemdesc}
This is like plain but there is a
second icon for the down state of radio and check buttons that is
like the up state but has a highlighted background. This means that
you can now see a difference between the up and down states for the
radio and check buttons when used with a \gls{look-and-feel} style that
doesn't draw an up or down effect.
\end{itemdesc}

\itemtitle{Small Highlights}

\begin{itemdesc}
Similar to highlights but a smaller version.
\end{itemdesc}

\itemtitle{Bordered Switches}

\begin{itemdesc}
Uses the plain type for press buttons and
the bordered type icons for radio and check buttons. Unlike the
default bordered style, in this case the bordered icons also include
the \gls{look-and-feel}['s] up and down effect (where supported).
\end{itemdesc}

\itemtitle{Small Bordered Switches}

\begin{itemdesc}
Like bordered switches but a smaller version.
\end{itemdesc}

\end{deflist}

Some of these styles also display a textual label, which may be
above, below, to the left or to the right of the icon.
Alternatively, you can just select the text only option. If you
prefer to have a different style for the dialog windows, you can
uncheck the \widget{lookandfeel.dialog.as_general} button,
which will display a set of button styles that you can apply to the
dialog windows. For example, \figureref{fig:icontext} shows the
dialog window from \figureref{fig:graphics} with the
small icons and trailing text dialog button option.

As with the look and feel setting, you must restart \FlowframTk\
before the changes can be implemented.

\FloatFig
  {fig:icontext}
  {\includeimg{configureUI-graphics-icontext}}
  {Dialog Buttons with Small Icons and Trailing Text}


\section{Configuration Directory}\label{sec:configdir}


When you quit \FlowframTk, the settings will be saved in
\FlowframTk's \inlineglsdef{config_dir}.  This directory is
determined (and created if necessary) as follows:

\begin{itemize}
  \item If the environment variable \gls{JDRSETTINGS} exists
        and is a directory, that directory is used.

  \item If \FlowframTk\ detects the existence of \gls{JpgfDraw}['s]
        configuration directory, that will be used. If the file
        \file{jpgfdraw.conf} exists in that directory but the file
        \file{flowframtk.conf} doesn't exist, \FlowframTk\ will load the
        settings from \file{jpgfdraw.conf} and then save the new settings to
        \file{flowframtk.conf}. If you like, after you have quit
        \FlowframTk, you can remove the now unneeded file
        \file{jpgfdraw.conf} and rename the directory \file{.flowframtk} or
        \file{flowframtk-settings}, as appropriate.

  \item If the directory \file{home/.flowframtk} exists
        and is a directory, that directory is used (where
        \meta{home} indicates the user's home directory as given
        by the Java \code{user.home} property).

  \item If the directory \file{home/flowframtk-settings} exists
        and is a directory, that directory is used.

  \item If the operating system (as identified by the Java \code{os.name}
        property) is a version of Windows and the
        directory \file{home/flowframtk-settings} can
        be created, that directory is used.

  \item For other operating systems, if the directory
        \file{home/.flowframtk} can be created, that
        directory is used.

  \item If the directory \file{settings/user} or
        \file{settings} can be created in \FlowframTk's
        installation directory, that directory will be used
        (where \meta{user} is the current user's user name). This is
        sub-optimal and not recommended as it may be removed when
        upgrading to a new version.

  \item If none of the above, an error will occur and you will need
        to set the environment variable \gls{JDRSETTINGS} to
        a sensible location with read and write permissions.

\end{itemize}

The configuration directory may also contain:
\begin{itemize}
\item the list of \hyperref[sec:recentfiles]{recent files}
(written by \FlowframTk\ when it quits normally);

\item the \hyperref[sec:accelerators]{accelerator settings}
file \file{accelerators.prop}
(written by \FlowframTk\ when it quits normally);

\item the \mathsmode\ and \textmode\
\hyperref[sec:texconfigtext]{character mapping} files,
\file{mathmappings.prop} and \file{textmappings.prop}
(written by \FlowframTk\ when it quits normally);

\item the \hyperref[sec:preamble]{preamble} file \file{preamble.tex}
(created by you, if you want it, in any text editor) used by some of
the \filefn{export} functions
(see \sectionref{sec:texconfigpreamble});

\item the \file{languages.conf} file that stores the
\hyperref[sec:languages]{language settings} used for the resources and manual;

\item the \file{startup.conf} file that stores the names of the
fonts used by the \hyperref[sec:graphics]{startup splash screen};

\item the \file{latexfontmap.prop} file (created by you, if you want
it, in any text editor) that contains any \gls{font-mappings} (see
\sectionref{sec:fontfamily}).  For compatibility with
\gls{JpgfDraw}, the file may simply be \filefmt{latexfontmap}
without an extension.

\end{itemize}

In addition, the configuration directory is used to save the log
file, \file{flowframtk.log}, in the event that
\accelerator{menu.debug.writelog} is used (or \menu{debug.writelog}
if the command line option \switch{debug} is used). The emergency
save all function \accelerator{menu.debug.dumpall} (or
\menu{debug.dumpall} if the command line option \switch{debug} is
used) will create a subdirectory (using the current date and time to
construct the name) and will save all open images to that directory
with filenames of the form \file{imagen.jdr}.

\chapter{The Basics}\label{sec:thebasics}

\FloatFig
  {fig:mainwindow}
  {\includeimg[width=\linewidth]{jpdwindow}}
  {The Main Window}

\FlowframTk\ uses a \gls{mdi}. This means that you can have multiple
images loaded in separate child windows, without having to start up
new instances of \FlowframTk. Most of the buttons and menu items
will only be applied to the child window that currently has the
focus. If there are no child windows, or if they have all been
minimized, then the relevant buttons and menu items will be disabled.
The only exceptions are the non-\gls{canvas} specific items.
The main \FlowframTk\ window is shown in \figureref{fig:mainwindow}.


\section{The Canvas}\label{sec:thecanvas}

The \inlineglsdef{canvas} is the white area (that may optionally
have a~grid) in each of \FlowframTk's child windows on which you create
your picture. It shares a~child window with the
\hyperref[sec:thepreamblepane]{preamble editor pane}. The canvas and
preamble areas are separated by a divider that can be adjusted to
allow one side to take up more room in the window. This can be done
by dragging the divider.  With some \glspl{look-and-feel}, the
divider has small triangular buttons which you can click on to
collapse or expand one side, but not all have this function nor can
some of them completely hide the preamble pane (such as the GTK+
Look and Feel).  You can also use the \menu{tex.set_preamble} menu
item to show the preamble pane.

In \selectmode\ (but not when you're editing or
distorting a shape) you can \gls{drag-and-drop} image files or
text onto the canvas to add to the current image. With the
\texttool\ or \mathstool\ you can only \gls{drag-and-drop} text onto
the canvas. If you attempt to drop a file instead, you will get a
new \gls{textarea} containing the file's path (or URI). 

\section{The Preamble Pane}\label{sec:thepreamblepane}

The preamble pane by default is on the right of the \gls{canvas},
which may be used to specify code that should go in the document
preamble of exported files. (This includes exports that use \LaTeX\
as an intermediate step.)

This area is split into three tabs for the early preamble code,
the mid-preamble code, and the late preamble code.
See \sectionref{sec:preamble} for further details on how this
code is added during the export process, and
see \sectionref{sec:configureuidialog} to set a default preamble
for all images and to reposition the preamble pane.

\section{The Toolbars}\label{sec:thetoolbars}

There are two \inlineglspluraldef{toolbar}.  The horizontal toolbar positioned at
the top of the main window, which allows you to manipulate
\glspl{object} on the \gls{canvas} (as well as the save, load and
new buttons) and the vertical toolbar positioned to the left of the
main window, which you can use to create new \glspl{path} and
\glspl{textarea}. If a toolbar is too wide\slash tall, scroll buttons will
appear.

You can show or hide the \gls{toolbar} or change the location of
the vertical toolbar so that it appears on the right of \FlowframTk's
main window using the \widget{borders.title} tab in the
\dialog{configui} dialog.

The \hyperref[sec:thepreamblepane]{preamble pane} has its own
toolbar with buttons to edit the preamble text.


\section{The Rulers}\label{sec:therulers}

There are two \inlineglspluraldef{ruler}.The horizontal ruler
positioned above the \gls{canvas} which marks out the $x$-ticks, and
the vertical ruler positioned to the left of the \gls*{canvas} which
marks out the $y$-ticks. The gap between tick marks can be changed
using the \menu{settings.grid.settings} menu item.

You can show or hide the \glspl*{ruler} using the menu item
\menu{settings.rulers}.  You can adjust their size and number format
using the \widget{borders.title} tab in the \dialog{configui}
dialog box. When the rulers are visible, the grid unit is displayed
in the corner between the rulers. You can \gls{doubleclick} on this
corner to open the grid settings dialog.

\section{The Status Bar}\label{sec:thestatusbar}

\FloatFig
  {fig:statusbar}
  {\includeimg[width=\linewidth]{annoteStatusBar}}
  {The Status Bar}

The \inlineglsdef{statusbar} (\figureref{fig:statusbar}) is positioned along
the bottom of the main window. This has the following elements:

\begin{itemize}

\item The current position of the pointer (or the pointer's last
position before it was moved away from the \gls{canvas}). You can
\gls{doubleclick} on this area to open the \dialog{goto} dialog.

\item The \hyperref[sec:controlsettings]{storage unit}. You
can \gls{doubleclick} on this to open the storage unit selector.

\item The file status area.  If the current picture has been modified, it
will display the word \qt{\gls{info.modified}}, otherwise it
will be blank.

\item The \gridlock\ indicator. This shows if the \gridlock\ is on
\includeimg{key} or off \includeimg{keycross}. You can
\gls{doubleclick} on this area to toggle the \gridlock\ setting as an
alternative to using the \menu{settings.grid.lock} menu item or
button.

\item The current magnification. You can adjust this value by
\glslink{click}{clicking} on the plus or minus icons to move up or
down the pre-defined list of zoom settings. Alternatively, in the area
between them where the magnification value is displayed, you can
\gls{menuclick} to popup a~menu or \gls{doubleclick} to open the
\dialog{zoom} dialog.

\item A help button \includeimg{statushelp} that you can \gls{click} on
to open the manual at the section related to the current operation.
This button is only present when the current operation has a section
in the manual.

\item A brief message about the currently selected tool or
operation. If the message is too long to fit in the provided area,
you can \gls{doubleclick} on it to open a dialog window with the
full message (\figureref{fig:information}).

\end{itemize}

\FloatFig
  {fig:information}
  {\includeimg{informationDialog}}
  {The Information Dialog}

You can show or hide the \gls*{statusbar}, or elements within the
bar, using the \dialog{configui} dialog. You can also use this
dialog to change the font used in the \gls*{statusbar}.


\chapter{The File Menu}\label{sec:filemenu}

\menudef{menu.file}

You can use the \menu{file} menu to create a
\hyperref[sec:newimage]{new} picture,
\hyperref[sec:loadimage]{load} a picture from a
\gls{JDR} or \gls{AJR} file,
\hyperref[sec:saveimage]{save} the current picture,
\hyperref[sec:exportimage]{export} the current
picture to a supported format (such as a \LaTeX\ file),
\hyperref[sec:imagedescription]{assign a description} to the
current picture, \hyperref[sec:print]{print} the
current picture or \hyperref[sec:quit]{quit} \FlowframTk.

\section{New}\label{sec:newimage}

\menudef{menu.file.new}

To start a new picture, select \menu{file.new}. This will open
a new child window. You can switch between child windows using the
\menu{window} menu.

\section{Open}\label{sec:loadimage}

\menudef{menu.file.open}

To load a \gls{JDR} or \gls{AJR} file, select \menu{file.open}.
If there is already a picture in the current child window, a new
child window will open to display the file. Note that although
\FlowframTk\ can export to other formats, it can only load
\gls{JDR} and \gls{AJR} files. (Although there is an import option
available in the experimental mode.)

If you load an image that contains a link to a \gls{bitmap} and
the bitmap is no longer in the same location, you will be prompted
for a new link or you can discard the link. Note that if you
select a new link, the \LaTeX\ link will also be updated.
If there is insufficient memory in the \gls{jre} to load a bitmap,
\FlowframTk\ will revert to draft mode for that bitmap.

On some systems you may be able to \gls{drag-and-drop} \gls{JDR} or
\gls{AJR} files from a filer window onto \FlowframTk's internal desktop
(the \manmsg{grey} background of \FlowframTk's main window) and this
will load each file into \FlowframTk. If you drag and drop a
\gls{bitmap} it will be equivalent to \hyperref[sec:newimage]{creating a
new image} and then \hyperref[sec:insertbitmap]{inserting the bitmap}.

For example, in \figureref{fig:dnda} three \gls{JDR} files and
a PNG file have been selected and then dragged onto \FlowframTk's
desktop, \figureref{fig:dndb}. This results in four child
windows, \figureref{fig:dndc}, where one of them is a new
untitled image with the link to the \gls{bitmap} inserted.

\FloatSubFigs{fig:dnd}
{
  {fig:dnda}{\includeimg{dnd1}}{},
  {fig:dndb}{\includeimg{dnd2}}{},
  {fig:dndc}{\includeimg{dnd3}}{}
}
{Drag and Drop onto \FlowframTk's Internal Desktop}

If you want to open an image in a new child window using drag and drop,
make sure you drop onto \FlowframTk's desktop. If you drop a file onto a
text area, such as the preamble pane or the text field used to
create a new \gls{textarea} then the filename (or URI) will be
inserted instead. If you drop onto a \gls{canvas}, the file contents
will be added to the current image (in the case of a \gls{JDR} or
\gls{AJR} file) or a link will inserted into the current image (in the
case of a \gls{bitmap} file). You can only drop a file onto the
\gls{canvas} in \selectmode\ (and no objects are being distorted or
edited).

\section{Recent Files}\label{sec:recentfiles}

\menudef{menu.file.recent}

To load a recently used \gls{JDR} or \gls{AJR} file, use the sub
menu \menu{file.recent}. A maximum of ten files, starting with the
most recently used are listed. Note that loading a file from this
list will change the open file chooser directory to that file's
directory.

\section{Image Description}\label{sec:imagedescription}

\menudef{menu.file.image_description}

You can use the \menu{file.image_description} dialog box to give the
image a description. The description is not visible in the image,
but may be saved as a comment if you \filefn{export} the image. If you have the
\widget{clssettings.pdfinfo} option selected in the
\widget{clssettings.title} tab of the \dialog{texconfig} dialog box,
the image description will be added to an exported \LaTeX\ document
in the \gls{pdfinfo/Title} attribute of the \gls{pdfinfo} argument.

\section{Save and Save As}\label{sec:saveimage}

\menudef{menu.file.save}
\menudef{menu.file.save_as}

You can save the current picture in \FlowframTk's native
\gls{JDR} (binary) or \gls{AJR} (ASCII) format using either
\menu{file.save} (if it already has a name)
or \menu{file.save_as} (if you want to specify the
filename).  I strongly recommend that you save your work frequently.
There is no auto-save function. The \gls{JDR} format can store
higher precision values than the \gls{AJR} format.

\section{Export}\label{sec:exportimage}

\menudef{menu.file.export}

\begin{warning}
Note that \FlowframTk\ can't load the files that it can export, so I
strongly recommend that you first save the picture as a
\gls{JDR} or \gls{AJR} file before exporting it, in case you
wish to edit it later.
\end{warning}

The \menu{file.export} menu allows you to export your image. The
\inlineglsdef{export.title} dialog box provides a way to specify the name
and location of the file.  The file format for the exported file
depends on the selected file filter.  A default filename will be
provided with the same basename as the image (if it has one).

The \gls{filter.png} file filter (export to PNG) will ignore all
\LaTeX\ settings and render the image as \glslink{painting}{drawn}
on the \gls{canvas} (without annotations or \glspl{selection-bbox}).
The other formats all expect a \TeX\ distribution.

\subsection{Export to PNG}\label{sec:exportpng}

Select the \inlineglsdef{filter.png} file filter in the
\dialog{export} dialog box to export to a PNG file. This export type
ignores all \LaTeX\ settings. Once you have selected the destination
\inlineglsdef{ext.png} file, the \inlineglsdef{exportpng.title}
dialog box will open (see \figureref{fig:exportpngdialog}).

\FloatFig
  {fig:exportpngdialog}
  {\includeimg{exportpngdialog}}
  {Export to PNG Settings}

Select the \widget{exportpng.usealpha} box if you want a transparent
background. If the box isn't checked, the background will be white.

Select the \widget{exportpng.crop} checkbox if you want the image cropped
to the image's \gls{image.bbox}. If the checkbox isn't checked, the image
will be the same size as the paper \gls{paper.size}. 

Note that all \manmsgpl{colour} will be converted to RGB.
The resulting PNG image won't include the grid or any annotations
but will otherwise render as the image shown on the \gls{canvas}.

\subsection{Export to a \LaTeX\ Class or Package}\label{sec:exportsty}

Select the \widget{filter.cls} file filter in the
\dialog{export} dialog box to export to a \LaTeX\ class
(\inlineglsdef{ext.cls}) or selected the \widget{filter.sty}
file filter to export to a \LaTeX\ package (\inlineglsdef{ext.sty}).

The class or package will require the \sty{flowfram} package to
create the page layout defined by the image through the functions in
the \menu{tex.flowframe} menu.  The \sty{pgf} package will also be
required to create borders and backgrounds, where applicable.

Only \glspl{object} that have been identified as static, flow or
dynamic frames will be exported (see \sectionref{sec:flowframe}).
This means that you can create additional \glspl{object} for use as
guidelines and they won't be included in the class or package.

The settings provided in the \dialog{textconfig} configuration
will be used. If exporting to a class, the provided class will be
used as a basis (using \gls{LoadClass}). The class setting is
ignored when exporting to a package.

The early-\hyperref[sec:preamble]{preamble text} is added before the
start of the package option declarations, which provides a means to
add extra package options, if required.  The mid-preamble text will
be placed after the class or packages are loaded and the
end-preamble text is placed at the end of the class or package.

\begin{important}
The \hyperref[sec:texconfigpreamble]{default preamble} is not used
when exporting to a class or package.
\end{important}

Any occurrences of \gls{usepackage} in the image
preamble will be converted to \gls{RequirePackage} on export.

\seealsorefs{sec:texconfig,
sec:preamble,
sec:flowframe,
sec:houseexample,
sec:neuronexample,
sec:postertutorial,
sec:newstutorial}

\subsection{Export to PGF}\label{sec:exportpgf}

Select the \widget{filter.pgf} file filter in the \dialog{export}
dialog box to export to a \LaTeX\ file that contains a
\env{pgfpicture} environment, which can then be included into a
\LaTeX\ document using \gls{input}.

The start of the file will include comments with
code between an \csfmt{iffalse} \ldots\ \csfmt{fi} block
that may be required in your preamble. If necessary, copy that block
into your document's preamble.

If you want to create a standalone file, complete with a document
class and document environment, you will need the
\nameref{sec:exportdoc} function instead.

\seealsorefs{sec:texconfig,
sec:preamble,
sec:houseexample,
sec:neuronexample}

\subsection{Export to Single-Paged Document}\label{sec:exportdoc}

To create a standalone \LaTeX\ document (\inlineglsdef{ext.tex}), select
the \widget{filter.pgfdoc} file filter, which will use the image's
paper \gls{paper.size}, or select the \widget{filter.pgfencapdoc} file filter to
set the page geometry to the image's \gls{image.bbox} (or \gls{image.typeblock}
if \widget{clssettings.use_typeblock_as_bbox} is set, see
\sectionref{sec:normalsize}).

\begin{information}
For the encapsulated export function,
if the bounding box is smaller than the baselineskip, the page
height is set to the baselineskip, according to the given
\hyperref[sec:normalsize]{normal size} for the image.
\end{information}

Alternatively, select the \widget{filter.pdf} file filter to create
a \inlineglsdef{ext.pdf} file.  This function works by first creating a
standalone encapsulated \LaTeX\ document in a temporary file and
then compiling it with \app{pdflatex}. The temporary file (and any
associated files created in the process) will then be deleted.

If a default preamble has been set (see
\sectionref{sec:texconfigpreamble}) the code will be included in the
\ext{tex} file's preamble.  This is in
addition to the contents of the image's early-, mid- and
late-\hyperref[sec:preamble]{preamble}. For example, I have set the
default preamble to:
\begin{verbatim}
\usepackage[utf8]{inputenc}
\usepackage[T1]{fontenc}
\usepackage{lmodern}
\end{verbatim}
If you prefer to use \sty{fontspec}, change the application path for
\app{pdflatex} to \app{lualatex}.

If your image includes \LaTeX\ commands that require a particular
package, you will need to add these packages to the
\hyperref[sec:preamble]{image preamble}. This is done automatically
by the mapping function, if it has been
\hyperref[mi:texmappings]{enabled}.  See
\sectionsref{sec:texconfigtext,sec:preamble} for further details.

\FlowframTk\ will automatically add
\verb|\usepackage{pgf}| to the preamble and other packages such as
\sty{geometry} and \sty{ifpdf}. It will also add
\begin{verbatim}
\usepgflibrary{decorations.text}
\end{verbatim}
in case the image contains any \glspl{textpath}.

\begin{warning}
Changes caused by differences in the way that \FlowframTk\ renders
\glspl{textarea} with the way that \LaTeX\ typesets them
can cause the image to be clipped if the bounding box
has been underestimated. If this is a problem, you can switch on
the \widget{clssettings.use_typeblock_as_bbox}
option and use the \gls{image.typeblock} function to mark your
preferred bounding box.
\end{warning}

For example, in \figureref{fig:equation} I have a \gls{textarea} with the
alternative \LaTeX\ text set to:
\begin{verbatim}
$\displaystyle f(\boldsymbol{\Upupsilon}; \eta ) = 
\frac{1}{2}\eta \sum_k\boldsymbol{\upupsilon}_k\cdot 
\boldsymbol{\upupsilon}_k$
\end{verbatim}
Below this is a \gls{path} for illustrative purposes. When viewed in
\FlowframTk, as shown in \figureref{fig:equationa}, the
\gls*{textarea} and \gls*{path} are the same widths, but when the
image is exported to PDF, shown in \figureref{fig:equationb}, the
equation generated by the alternative text is wider that its
representation on the \gls{canvas} and has consequently been clipped
(both the top of the fraction and the right end of the equation)
because it exceeds the image bounding box that was computed by
\FlowframTk.

In \figureref{fig:equationc}, a rectangle has been
added to the image. This now extends the image bounding box, so when
the image is now exported to PDF, as shown in
\figureref{fig:equationd}, the equation is no longer clipped,
but the new rectangle is now visible, which may not be desirable.

There are two ways to deal with this. The first is to give the
rectangle a transparent \pathattr{line-paint} (in addition to the
default transparent \pathattr{fill-paint}) but now it can only been detected on the
\gls{canvas} when it's selected. The second method is to set the
\gls{image.typeblock} to the area of that rectangle (see \sectionref{sec:typeblock}).
The easiest way to do this is to \select\ the rectangle, use the
\menu{tex.flowframe.set_typeblock} menu item to open the
\dialog{typeblock} dialog box, and click on the
\widget{typeblock.compute_from_path} button. Then delete the
rectangle. The \gls{image.typeblock} is now shown as a \manmsg{grey} rectangle in
\figureref{fig:equatione}. Now make sure that the
\widget{clssettings.use_typeblock_as_bbox}
setting is on. Redoing the export to PDF now
produces the image shown in \figureref{fig:equationf}.

\FloatSubFigs{fig:equation}
{
 {fig:equationa}{\includeimg[scale=0.7]{equationa}}{},
 {fig:equationb}{\includeimg[scale=0.7]{equationb}}{},
 {fig:equationc}{\includeimg[scale=0.7]{equationc}}{},
 {fig:equationd}{\includeimg[scale=0.7]{equationd}}{},
 {fig:equatione}{\includeimg[scale=0.7]{equatione}}{},
 {fig:equationf}{\includeimg[scale=0.7]{equationf}}{}
}
[Exported Text Areas Can Overflow Bounding Box]
{Exported Text Areas Can Overflow Bounding Box:
\subfigref{fig:equationa} image as shown on canvas;
\subfigref{fig:equationb} image exported to PDF (equation has been clipped);
\subfigref{fig:equationc} rectangle added to image;
\subfigref{fig:equationd} image from \subfigref{fig:equationc} exported to PDF;
\subfigref{fig:equatione} typeblock added and rectangle removed;
\subfigref{fig:equationf} image from \subfigref{fig:equatione} exported to PDF.}

If the exported PDF file has unexpected results or fails to be
created, try exporting to an encapsulated \LaTeX\ document and then
manually building the document, as the issue may be due to the
conversion process or it may be due to an unrecognised command or
syntax error in a \gls{textarea} or \gls{textpath}.

\seealsorefs{sec:texconfig,sec:preamble,sec:fontanchor}

\subsection{Export to SVG or EPS}\label{sec:exportsvgeps}

The export to SVG or EPS functions both work in the same way
as \hyperref[sec:exportdoc]{exporting to an encapsulating PDF}
in that a temporary encapsulated \LaTeX\ file is first created.

If the \widget{filter.svg} file filter is chosen, then the temporary
file will be compiled with \app{latex} followed by
\app{dvisvgm} to created an  \inlineglsdef{ext.svg} file.

If the \widget{filter.eps} file filter is chosen, then the temporary
file will be compiled with \app{latex} followed by
\app{dvips} to created an \inlineglsdef{ext.eps} file.

In either case, if the image contains any
\glspl{bitmap}, \FlowframTk\ will attempt to convert
them to EPS, where necessary.

\begin{important}
Note that PostScript does not support transparency.
\end{important}

As with the export to encapsulated PDF, if the resulting SVG or EPS
file has unexpected results or fails to be created, try exporting to
an encapsulated \LaTeX\ document and then manually building the
document.

\section{Page Setup}\label{sec:pagesetup}

\menudef{menu.file.page}

The \menu{file.page} menu item will open the printer's page
setup dialog box. This will require a print service to be available.
The \glspl{printer-margin} will be shown on the \gls{canvas} if the
\menu{settings.paper.margins} menu item is selected.

\section{Print}\label{sec:print}

\menudef{menu.file.print}

You can \glslink{printing}{print} the current image using the \menu{file.print} menu
item which will open the printer dialog box. If no printer is found,
the error message \qt{\errmsg{printing.no_service}} will
be displayed. If this happens, check that the printer is switched
on and connected to the computer. The print function uses the
\gls{canvas}['s] \gls{rendering} and \gls{anti-aliasing} settings
(provided in the \dialog{graphics} configuration) and draws
the image as it's displayed on the \gls{canvas} (at 100\%
magnification and without the grid or annotations).

\begin{information}
If you use the printer's \qt{Print to PDF} function (where
supported), the resulting PDF may be different from that created
with the export to PDF function (see \sectionref{sec:exportdoc})
as the \LaTeX-related settings won't be used by the printer.
\end{information}

\section{Message Window}\label{sec:messages}

\menudef{menu.file.messages}

You can display the \inlineglsdef{message.title} window using the
\menu{file.messages} item. This usually just lists the
files that have been opened, saved or exported, but warnings are
sometimes written there as well. This window is opened during a read
or write operation. When an external application is spawned, the
abort button becomes enabled, which you can use to kill the spawned
process if required.

\section{Close}\label{sec:closeimage}

\menudef{menu.file.close}

You can close the current child window, either by clicking on the
child window's close icon or by selecting \menu{file.close}.
If there is any unsaved data, \FlowframTk\ will ask for confirmation
before discarding the window (see \figureref{fig:confirmclose1}).
In this dialog box you can:

\begin{deflist}
\itemtitle{\includeimg{discard}}

\begin{itemdesc}
click on the discard button next to the image name to discard the
image;
\end{itemdesc}

\itemtitle{\includeimg{save}}

\begin{itemdesc}
click on the save button next to the image name to save the image;
\end{itemdesc}

\itemtitle{\includeimg{cancel}}

\begin{itemdesc}
click on the cancel button at the bottom of the dialog box to cancel
the close operation.
\end{itemdesc}

\end{deflist}

\FloatFig
  {fig:confirmclose1}
  {\includeimg{confirmclose1}}
  {Confirm Discard Dialog (1 Modified Image).}

An image will only be marked as unmodified if it has been saved as a
\gls{JDR} or \gls{AJR} file.  If you have
\hyperref[sec:exportimage]{exported} your image to another file
type, I recommend that you also \hyperref[sec:saveimage]{save} it as
a \gls{JDR} file as well, in case you want to edit it later.

Note that you must finish or discard any \gls{path} that is under
\hyperref[sec:newobjects]{construction} before you can close
an image.

\section{Quit}\label{sec:quit}

\menudef{menu.file.quit}

To quit \FlowframTk\ either use the menu item
\menu{file.quit} or click on the close icon on the
main window. All child frames will be closed. If any child frame
contains unsaved data, you will be asked for confirmation before
the window is discard. If there is only one modified image, the
dialog box is as for the \hyperref[sec:closeimage]{close operation} shown in
\figureref{fig:confirmclose1}, otherwise it's as shown in
\figureref{fig:confirmclose2}, where each modified file is listed.

\FloatFig
  {fig:confirmclose2}
  {\includeimg{confirmclose2}}
  {Confirm Discard Dialog (2 Modified Images).}

As with the close operation, for each listed file, you can:
\begin{deflist}
\itemtitle{\includeimg{save}}

\begin{itemdesc}
click on the save button next to the image name to save the image;
\end{itemdesc}

\itemtitle{\includeimg{discard}}

\begin{itemdesc}
click on the discard button next to the image name to discard the
image.
\end{itemdesc}

\end{deflist}

Alternatively you can:
\begin{deflist}
\itemtitle{\includeimg{save_all}}

\begin{itemdesc}
click on the save all button to save all the listed files;
\end{itemdesc}

\itemtitle{\includeimg{discard_all}}

\begin{itemdesc}
click on the discard all button to discard all the listed files and
quit;
\end{itemdesc}

\itemtitle{\includeimg{cancel}}

\begin{itemdesc}
click on the cancel button at the bottom of the dialog box to cancel
the quit operation.
\end{itemdesc}

\end{deflist}

\chapter{Creating New Objects}\label{sec:newobjects}

\menudef{menu.tools}

New \glspl{path} and \glspl{textarea} can be created using
\FlowframTk's \gls{construction}, which can be obtained using any tool
except the \tool{select}. The tools can be selected using either the
vertical \gls{toolbar} or the \menu{tools} menu. Once \glspl{path}
and \glspl{textarea} have been created they can then be edited or
transformed or combined to form a \gls{textpath} object. You can
also apply patterns to \glspl{path} or \glspl{textpath}.

\menudef{menu.tools.finish}

Finish the \gls{path} or \gls{textarea} by pressing
\accelerator{menu.tools.finish} or by selecting \menu{tools.finish}
or by clicking on the finish button or (\gls{path}) by
\glslink{doubleclick}{double-clicking} (instead of single-clicking)
on the final vertex or (\gls{textarea}) clicking on the \gls{canvas}
where you want the next \gls{textarea} to start.  

Note that transferring the focus to
another \FlowframTk\ child window or selecting a new tool whilst you
are constructing a \gls{path}, will complete the current path.

\menudef{menu.tools.abandon}

Cancel the current \gls{path} by pressing the
\accelerator{menu.tools.abandon} key or by selecting
\menu{tools.abandon} or by clicking on the abandon path button.
(Not applicable for \glspl{textarea}.)

\menudef{menu.tools.gap}

The \inlineglsdef{gap} or \qt{\gls{move-to}} function is only
available when constructing an open or closed
\hyperref[sec:newlinepath]{line path} or \hyperref[sec:newcurvepath]{curve
path}.  Once you have clicked on the vertex where you want the
\gls{gap} to start, select \menu{tools.gap} or click on the
\gls{gap} button or press \accelerator{menu.tools.gap}, then click
on the \gls{canvas} where you want the \gls{gap} to end. The next
segment will resume according to the tool (line or curve).

\begin{warning}
The undo/redo mechanism is disabled while you are constructing a
path, however while you are creating a line path or a curve path,
you can delete the previous segment using the
backspace \accelerator{action.delete_last} key.
\end{warning}

Note that \FlowframTk\ won't allow you to create a \gls{path} whose
total width and height are both less than 1.002\gls{bp}. This is to
prevent accidentally creating a tiny path that can't be seen but
contributes to the total image dimensions. This restriction only
applies when creating \glspl*{path} and does not apply to editing
paths.

The \gls{pathattributes} will only be set
once the \gls{path} has been completed. Whilst the
\gls{path} is under construction you will
only see a draft version (see \figureref{fig:draftfinal}).
If you want a \gls{path} with a mixture of line and curve
segments, first construct a \gls{path} with only one
type of segment (e.g.\ lines), and then use the
\gls{menu.edit.path.edit} function to convert
the required segments.

\FloatSubFigs{fig:draftfinal}
{
 {fig:draftfinala}{\includeimg{underconstruction}}{},
 {fig:draftfinalb}{\includeimg{curvecompleted}}{}
}
[Path Attributes Are Only Set Once the Path is Completed]
{Path attributes are only set once the path is completed:
\subfigref{fig:draftfinala} path under construction;
\subfigref{fig:draftfinalb} path completed.}

If you are unable to use the mouse, you can move the pointer using
\menu{navigate.goto}. The keystroke
\accelerator{action.construct_click} will emulate a single mouse
click in \gls{construction}.

\seealsorefs{sec:editpath,sec:symmetric,sec:patterns}

\section{Line Paths}\label{sec:newlinepath}

\menudef{menu.tools.open_line}

To construct an open line \gls{path}, select the open line tool,
either by clicking on the open line button or by selecting
\menu{tools.open_line}. You can later close the \gls{path}
with the \gls{menu.edit.path.edit} function.

\menudef{menu.tools.closed_line}

To construct a closed line \gls{path}, select the closed line tool,
either by clicking on the closed line button or by selecting
\menu{tools.closed_line}. You can later open the \gls{path}
with the \gls{menu.edit.path.edit} function.

Use the \glslink{primaryclick}{primary mouse button} to click on each vertex
defining the path. To complete the \gls{path}, do one of the following:
\begin{itemize}
\item \glslink{doubleclick}{Double-click} instead of single-clicking
on the final vertex: this performs the combined function of defining
the vertex and finishing the path. If you use this method, be
careful not to accidentally create two coincident vertices at the
end point, or it will cause a problem for any mid or end
\hyperref[sec:markers]{marker} that you apply.

\item Single-\gls{click} on the final vertex and then
complete the path by pressing \accelerator{tools.finish} or by
clicking on the finish button or by selecting
\menu{tools.finish}.
\end{itemize}
If you have used the closed line tool, the path will automatically
be closed by inserting a line between the end vertex
and the initial vertex.

\seealsorefs{
 sec:rectangles,
 sec:linepaint,
 sec:fillpaint,
 sec:pathstyle,
 sec:editpath,
 sec:houseexample,
 sec:accesstutorial
}

\section{Curve Paths}\label{sec:newcurvepath}

\menudef{menu.tools.open_curve}

To construct an open curve \gls{path}, select the open curve tool,
either by clicking on the open curve button or by selecting
\menu{tools.open_curve}. You can later close the \gls{path}
with the \gls{menu.edit.path.edit} function.

\menudef{menu.tools.closed_curve}

To construct a closed curve \gls{path}, select the closed curve tool,
either by clicking on the closed curve button or by selecting
\menu{tools.closed_curve}. You can later open the \gls{path}
with the \gls{menu.edit.path.edit} function.

Use the \glslink{primaryclick}{primary mouse button} to click on each vertex in
the \gls*{path}.  There is no way to specify the location of the
\glspl{controlpt} defining the curvature of the \gls*{path} whilst the
\gls*{path} is under construction, however, once the
\gls*{path} has been completed, it is possible to move these
\glspl*{controlpt} in \editpathmode.

To complete the \gls*{path}, do one of the following:
\begin{itemize}
\item \glslink{doubleclick}{Double-click} instead of single-clicking
on the final vertex: this performs the combined function of defining
the vertex and finishing the path. If you use this method, be
careful not to accidentally create two coincident vertices at the
end point, or it will cause a problem for any mid or end
\hyperref[sec:markers]{marker} that you apply.

\item Single-\gls{click} on the final vertex and then
complete the path by pressing \accelerator{tools.finish} or by
clicking on the finish button or by selecting
\menu{tools.finish}.
\end{itemize}
If you have used the closed curve tool, the path will automatically
be closed by inserting a curve between the end vertex
and the initial vertex.

\seealsorefs{
 sec:ellipses,
 sec:linepaint,
 sec:fillpaint,
 sec:pathstyle,
 sec:editpath,
 sec:neuronexample
}

\section{Rectangles}\label{sec:rectangles}

\menudef{menu.tools.rectangle}

To construct a rectangle, select the rectangle tool either by
clicking on the rectangle tool button or by selecting
\menu{tools.rectangle}.

Use the \glslink{primaryclick}{primary mouse button} to click where
you want the first corner to go, then move (not drag) the mouse to
the opposite corner, and \gls{click} or press
\accelerator{menu.tools.finish} to complete the \gls{path}.

Note that this function is just a shortcut to using the
\gls{menu.tools.closed_line} function. Once the rectangle is
created, it is simply another closed \gls{path}, and can be edited in
exactly the same way.

\seealsorefs{
 sec:newlinepath,
 sec:linepaint,
 sec:fillpaint,
 sec:pathstyle,
 sec:editpath,
 sec:houseexample,
 sec:neuronexample
}


\section{Ellipses}\label{sec:ellipses}

\menudef{menu.tools.ellipse}

To construct an ellipse, select the ellipse tool either by clicking
on the ellipse tool button or by selecting
\menu{tools.ellipse}.

Use the \glslink{primaryclick}{primary mouse button} to click on the
\manmsg{centre} point of the ellipse, and then move (not drag) the mouse
until the ellipse has reached the desired dimension, and \gls{click}
or press \accelerator{menu.tools.finish} to complete the \gls{path}. If you want
to create a circle, I recommend that you first enable the \gridlock\ or use the
\menu{navigate.goto} menu item.

Note that this function is just a shortcut to using the
\gls{menu.tools.closed_curve} function. Once the ellipse is
created, it is simply another closed \gls{path}, and can be edited in
exactly the same way.

\seealsorefs{
 sec:newcurvepath,
 sec:linepaint,
 sec:fillpaint,
 sec:pathstyle,
 sec:editpath,
 sec:cheeseexample
}

\section{Text}\label{sec:newtext}

\Glspl{textarea} can be created to annotate images. Each
\gls*{textarea} has text that's displayed on the \gls{canvas} but
may have alternative text to use when
\hyperref[sec:exportimage]{exporting} to a \LaTeX\ file. (This includes
the export functions that create a temporary \LaTeX\ file, such as
the export to PDF function.) There are two tools to create a
\gls*{textarea}: the \texttool\ and the \mathstool. These
affect the default alternative text for the new \gls*{textarea} in
different ways. You can later \hyperref[sec:edittext]{edit the
alternative text}, if the default isn't suitable.

Line breaks are not supported in \gls{textarea}. If you are using
the \hyperref[sec:flowframe]{flowframe} functions, you can create a
static frame and provide multi-lined text for the frame content.

\menudef{menu.tools.textarea}

The text tool can be selected either by clicking on the
\gls{menu.tools.textarea} tool button or by selecting \menu{tools.textarea}.

If \widget{textconfig.escape} is enabled, the alternative text will be set
to the original supplied text with any identified characters
substituted according to the \widget{textconfig.textmode} settings.

\menudef{menu.tools.math}

The \manmsg{maths} tool can be selected either by clicking on the
\gls{menu.tools.math} tool button or by selecting \menu{tools.math}.

If \widget{textconfig.escape.mathchar} is enabled, the alternative text will be set
to the original supplied text with any identified characters
substituted according to the \widget{textconfig.textmode} settings.

After any substitutions on the alternative have been performed, 
the \gls{mshift} character will be added to the start and end
of the alternative text.


With either the \texttool\ or the \mathstool,
once you have selected the required tool, click with the
\glslink{primaryclick}{primary mouse button} at the position where
you want the text to start. This will produce a shaded area with a
cursor in which you can type no more than a single line of text
(\figureref{fig:textareafield}).
Clicking inside this shaded area will move the cursor around the
text area under construction.

\FloatFig
  {fig:textareafield}
  {\includeimg{textareafield}}
  {Text Area Construction Field}

When you want to complete a \gls{textarea}, press
\accelerator{menu.tools.finish} (which will start a new
\gls{textarea} on the line below) or click anywhere on the canvas
outside of the current text area (which will start a new
\gls{textarea} at the new location) or use the
\gls{menu.tools.finish} button or \menu{tools.finish}.  Selecting
another tool whilst a \gls{textarea} is under construction will
finish the current \gls{textarea}, unless you're switching between
the \texttool\ and \mathstool.  (If you do switch between the
text and \manmsg{maths} tools, the alternative \LaTeX\ text is only
created when you finish the \gls*{textarea}, so the mappings applied
will be governed by the tool currently selected when you completed
the \gls*{textarea}.) An empty \gls{textarea} will be discarded.

If a mapping is performed that has an associated package, then the
package will be automatically add to the image's
\hyperref[sec:preamble]{early-preamble}. The mapping function will
first search the early-preamble code to check if the package is
already present, but it doesn't check the mid- or late-preamble code
or the code in the \hyperref[sec:texconfigpreamble]{default
preamble}.

Once a \gls{textarea} has been completed, the only way to edit it is via the
\hyperref[sec:edittext]{edit text function}. If you click the
mouse on the location of a completed text area (while the
\texttool\ or \mathstool\ is selected) you will simply create a new
overlapping \gls*{textarea}.

If your operating system supports \gls{drag-and-drop}, you can also
\gls{drag-and-drop} text onto the \gls{canvas}. If the drop point is
inside the \gls{textarea} construction field, the dropped text will
be inserted into the field (newline characters will be converted to
spaces). If the drop point is outside the \gls{textarea}
construction field, a new text area will be created for each line of
dropped text. The mapping applied to the dropped text will be
according to the currently selected text tool.  If text is dropped
onto the \gls{canvas} in \selectmode, the text mappings (not
\manmsg{maths} mappings) will be applied. If a file is dropped onto
the \gls{canvas} when either the \texttool\ or \mathstool\ is
active, the file path (or URI) will be added as a new
\gls{textarea}.

\figureref{fig:dndtext} illustrates \gls{drag-and-drop} to create new
\glspl{textarea}.
In \figureref{fig:dndtexta} I have four lines of
text selected in a text editor. The second line simply contains two
space characters (which aren't visible). These four lines of text
are then dragged onto the \gls{canvas},
\figureref{fig:dndtextb}, and dropped. The drop location (the
location of the pointer when the mouse button was released) is
directly on the canvas, not on the \gls{textarea} construction field (which
currently isn't visible). This creates three new \glspl*{textarea},
\figureref{fig:dndtextc}.
The line solely consisting of white space hasn't created a 
\gls{textarea} but has contributed to the vertical offset of the following
\gls{textarea}. In \figureref{fig:dndtextd}, the same three lines of
text are dragged onto the \gls{textarea} construction field and dropped
at the cursor, \figureref{fig:dndtexte}. This has inserted the
dropped text into the \gls{textarea} construction field. The new line
characters have been converted to spaces and the \gls{textarea}
construction field is awaiting further input.

\FloatSubFigs*{fig:dndtext}
{
 {fig:dndtexta}{\includeimg{dndtext1}}{},
 {fig:dndtextb}{\includeimg{dndtext2}}{},
 {fig:dndtextc}{\includeimg{dndtext3}}{},
 {fig:dndtextd}{\includeimg{dndtext4}}{},
 {fig:dndtexte}{\includeimg{dndtext5}}{}
}
[Drag and Drop Text]
{Drag and Drop Text: \subfigref{fig:dndtexta} drag initiated on some
selected text in another application; \subfigref{fig:dndtextb}
selected text dragged onto canvas; \subfigref{fig:dndtextc} the text
that was dropped onto the canvas in \subfigref{fig:dndtextb} has
been converted into three \glspl{textarea}; \subfigref{fig:dndtextd}
the selected text from \subfigref{fig:dndtexta} is now dragged onto
the \gls{textarea} construction field; \subfigref{fig:dndtexte} the
dropped text has been inserted into the construction field.}

\menudef{index.menu.textarea}

Whilst a \gls{textarea} is under construction, you can activate the
\gls{index.menu.textarea}, illustrated in
\figureref{fig:textpopup}, with a \gls{menuclick}. Alternatively,
you can use the \accelerator{popup} key or, if supported by your
system, the \accelerator{context_menu} \gls{context-menu} key.

\FloatFig
  {fig:textpopup}
  {\includeimg{textpopup}}
  {Text Area Popup Menu}

\menudef{menu.textarea.copy}

The \menu{textarea.copy} menu item copies to the clipboard any text within 
the text construction field that has been selected.

\menudef{menu.textarea.cut}

The \menu{textarea.cut} menu item cuts to the clipboard any text within 
the text construction field that has been selected.

\menudef{menu.textarea.paste}

The \menu{textarea.paste} menu item pastes text from the
clipboard to the cursor position in the text construction field.

\menudef{menu.textarea.select_all}

The \menu{textarea.select_all} menu item selects all text in 
the text construction field.

\menudef{menu.textarea.insert_symbol}

The \menu{textarea.insert_symbol} menu item 
opens the \dialog{symbolselector} dialog  if you want to
enter a symbol that doesn't appear on your keyboard. 

\plabel[Insert Symbol Dialog Box]{mi:insertsymbol}% HelpSet id
\widgetdef{symbolselector.title}

\FloatFig
  {fig:insertsymbol}
  {\includeimg[width=\linewidth]{annoteInsertSymbol}}
  {Insert Symbol Dialog Box}

The \dialog{symbolselector} dialog box (see \figureref{fig:insertsymbol})
has a field at the top which contains the text currently in the text
area. If you know the hexadecimal Unicode value for the character
you want to insert, you can type the number into the
\widget{symbolselector.unicode} box and press the
\includeimg{insertAtCaret} \widget{symbolselector.insertAtCaret} button to
insert it into the text field at the caret. 

On the right hand panel below the \widget{symbolselector.unicode} field
there is an enlarged image of the selected character. If there's an
associated mapping it will be displayed below the image
(see \figureref{fig:insertsymbol}), but remember that this mapping
may change if you switch between the \texttool\ and \mathstool\ before
completing the \gls{textarea}.

Alternatively, you can use the button panels on the left to
select the character you want to insert into the text field. Use the
left hand list to display the require Unicode block and either click
on the button with the required character on it to insert and
display the symbol or hold the shift key down while you click to
just display the symbol in the right hand panel without inserting it
into the text field. The available Unicode blocks and symbols are govern by the
\widget{lang.title} tab in the \dialog{configui} dialog.

Once the \gls{textarea} has been finished (by clicking on the
\gls{menu.tools.finish} button or by pressing
\accelerator{menu.tools.finish} or by switching to a non-text
tool) any packages that are required by the mappings will be added
to the \hyperref[sec:preamble]{early-preamble pane}, as shown in
\figureref{fig:addpackage}.  Note that the \gls{canvas} and the
preamble panes have separate undo\slash redo managers so if you undo
a new text area it will remove the new text area but won't remove
the modification to the preamble. You will need to switch to the
early-preamble panel and use its undo button or \gls{popupmenu} item.

\FloatFig
  {fig:addpackage}
  {\includeimg{addpackage}}
  {Package Required by Mapping Added to the Early Preamble}

\begin{important}
The text in your \env{pgfpicture} environment may not
look exactly the same as in \FlowframTk\ due to font differences as
well as the translation of \LaTeX\ commands.
\end{important}

\seealsorefs{%
sec:edittext,
sec:fontanchor,
sec:texconfig,
sec:preamble,
sec:textpaint,
sec:textstyle,
sec:texttopath,
sec:splittext,
sec:exportdoc,
sec:neuronexample,
sec:accesstutorial
}

\chapter{Bitmaps}\label{sec:insertbitmap}

\menudef{menu.bitmap}

\FlowframTk\ is primarily a \gls{vectorgraphics} application,
however it is possible to insert a \inlineglsdef{bitmap}
(\gls{rastergraphics}) into your picture for background effects or
if you want to annotate a \gls{bitmap} (as was done in
\figureref{fig:insertsymbol} in the previous section).  Note that
\FlowframTk\ does not save the actual \glslink{rastergraphics}{raster} data in either
the \gls{JDR} or the \gls{AJR} file, but instead it creates a reference
to the original file. You can't edit the actual bitmap data in
\FlowframTk. However you can scale, rotate or shear the link
using the \gls{affine-transformation} functions described in 
\sectionref{sec:affinetrans} or by directly editing the
bitmap's \gls{bitmap.transformation-matrix}. 

\begin{important}
If you change the location of the file containing the bitmap, when
\FlowframTk\ reloads the \gls*{JDR} or \gls*{AJR} file it will
prompt you to either provide the new location or discard the reference.
\end{important}

\menudef{menu.bitmap.insert}

To insert a \gls{bitmap} into your picture, first make sure you are using
the \tool{select} (and no shapes are
being edited or distorted), and then select the menu item
\menu{bitmap.insert} and a file selector dialog box will appear in which
you can choose the required \gls{bitmap}. The \gls{bitmap} will initially appear
in the top left hand corner of the \gls{canvas} but can be
\hyperref[sec:moveobjects]{moved} to a new location.  If your
operating system supports \gls{drag-and-drop}, you can also drag a
\gls{bitmap} file onto the \gls{canvas} (in \selectmode) and it will
be inserted at the drop location.

If there is insufficient memory in the \gls{jre} to load a bitmap,
\FlowframTk\ will revert to draft mode to display that bitmap.  For
example, in \figureref{fig:draftbitmap} several photos have been
inserted into an image. Since photos tend to be quite large, there
is insufficient memory to load the final photo, so it is displayed
in draft mode instead. Note that draft mode will also be used when
printing or when \hyperref[sec:exportpng]{exporting to PNG}. Since \LaTeX\ files
only contain a link to the bitmap, draft mode should not affect
exporting to \LaTeX\ files or to formats that use \app{latex} or
\app{pdflatex} as an intermediate step.

Note that the amount of memory available to any Java application is
set at startup. The default maximum value can be changed via the
\gls{jre} command line options.  If you run \FlowframTk\ from the
shell script \app{flowframtk}, then you can set the environment
variable \gls{JDR_JVMOPTS} to change the default configuration. See
\sectionref{sec:cmdargs} for further details. If you are running
\FlowframTk\ from Windows, you will need to check the \gls*{jre}
documentation.

\FloatFig
  {fig:draftbitmap}
  {\includeimg{draftBitmap}}
  [Bitmaps Are Displayed in Draft Mode When There is Insufficient Memory in the JRE]
  {Bitmaps are displayed in draft mode
when there is insufficient memory in the \gls{jre}. The area taken up by
the image is displayed as a semi-transparent light \manmsg{grey} rectangle
with the bitmap's file name in square brackets.}

\menudef{menu.bitmap.refresh}

If you use another application to edit the bitmap whilst
you have a picture with a link to it displayed in \FlowframTk, you will
need to select \menu{bitmap.refresh} to update the image.

\menudef{menu.bitmap.properties}

The \menu{bitmap.properties} menu item will open the 
\dialog{bitmap_properties} dialog (see \sectionref{sec:bitmapprops}).

\menudef{menu.bitmap.vectorize}

The \menu{bitmap.vectorize} function is an experimental
feature. It's only available if \app{flowframtk} is invoked with
the \switch{experimental} switch.

Currently undocumented.

\section{Properties}\label{sec:bitmapprops}

\widgetdef{bitmap_properties.title}

To change a \gls{bitmap}['s] properties (file reference, alternative
\LaTeX\ path, or \gls{bitmap.transformation-matrix}), \select\
the required \gls{bitmap} and select the \menu{bitmap.properties}
menu item. This will open up the \dialog{bitmap_properties} dialog box shown in
\figureref{fig:bitmapprops}. If you want to change the path name to
the bitmap, you can either type it directly into the
\widget{label.filename} field or click on the
\widget{browsebitmap.browse} button. You can also change the
transformation matrix applied to the bitmap link.

\FloatFig
 {fig:bitmapprops}
 {\includeimg{bitmapprops}}
 {Bitmap Properties Dialog}

If you want to \filefn{export} your picture to a \LaTeX\ file or
to a format that uses \app{latex} or \app{pdflatex} as an intermediate
step, the \LaTeX\ command can either be \gls{pgfimage} or
\gls{includegraphics} and is specified in the
\widget{bitmap_properties.command} field. The command argument in
either case must use a \gls{forwardslash} as the directory divider. If
the \widget{bitmap_properties.auto_latexfilename} \gls*{checkbox}
is selected, this substitution will be performed automatically by
the export function. Alternatively, you can unselect
the \widget{bitmap_properties.auto_latexfilename} \gls*{checkbox}
which will enable the
\widget{bitmap_properties.latexfilename} field in which you
can enter the argument to be used by the image command. If you use the
export to PostScript or \gls{svg} function, \FlowframTk\ will
attempt to create an EPS version of the bitmap (if it doesn't
already exist) since \app{latex} (as opposed to
\app{pdflatex}) can't include bitmap formats, such as PNG and
JPEG.

\seealsorefs{sec:bitmapconfig}

\section{Vectorizing}\label{sec:vectorize}

The \menu{bitmap.vectorize} function is an experimental
feature. It's only available if \FlowframTk\ is invoked with
the \switch{experimental} switch.

Currently undocumented.


\chapter{Selecting, Navigation and Basic Editing}\label{sec:selectobjects}

\menudef{menu.tools.select}

In order to edit an \gls{object}, you must be in \selectmode. To do
this either click on the \gls{menu.tools.select} button or use the 
\menu{tools.select} menu item. An \gls*{object} can
be selected using any of the following methods:
\begin{itemize}
\item Click on it with the \glslink{primaryclick}{primary mouse button}.

\item \Gls{doubleclick} the primary mouse button to
select the \gls*{object} \glslink{backobject}{behind} the current \gls*{object}.

\item Use \keyref{ctrl}-click (that is, \gls{click} whilst holding down the control
\keys{\keyref{ctrl}} key) if you want to add an \gls*{object} to the current
selection.

\item \Gls{click} on an empty part of the \gls{canvas} and
drag. A dashed rectangle will appear. When you release the mouse
button, any \glspl*{object} within that
region will be selected. (If you have the shift \keys{\keyref{shift}} key depressed,
only those objects which are completely inside the dashed
rectangle will be selected, but make sure you release the mouse
button before releasing the shift key.)

\item Use the \menu{navigate} menu (see \sectionref{sec:navigate}).

\end{itemize}

\menudef{menu.edit.select_all}

The \menu{edit.select_all} menu item will select all
\glspl{object}.

\menudef{menu.edit.deselect_all}

The \menu{edit.deselect_all} menu item will deselect all
\glspl{object}. You can also click on
an empty part of the \gls{canvas} to deselect all \glspl{object}.
Selecting another tool will also deselect all \glspl*{object}.

\begin{information}
To deselect an individual \gls*{object}, click on that object whilst
depressing the shift \keys{\keyref{shift}} key.
\end{information}

When an \gls{object} is selected, a dashed red\slash\manmsg{grey}
rectangle will be displayed around it. This rectangle is the 
\emph{\inlineglsdef{selection-bbox}} and may optionally have
\gls{hotspot} regions. Note that individual elements of a
\gls{group} can not be selected independently of the group.  When
you select a group, you will only see a dashed
red\slash\manmsg{grey} rectangle around the \gls{bbox} of the group,
not around the individual elements of the group.

In \selectmode\ you can \gls{drag-and-drop} \gls{JDR} or \gls{AJR}
files onto the \gls{canvas} (provided your operating system supports
drag and drop) which will add all the objects from those files to
the current image. The new objects will be added to the current
selection. You can't drag and drop if a shape is being edited or
distorted.

If text is dropped onto the \gls{canvas} in \selectmode, the text
mappings (not \manmsg{maths} mappings) will be applied. If a file is
dropped onto the \gls{canvas} in \textmode\ or \mathsmode, instead of
in \selectmode, the file path (or URI) will be added as a new \gls{textarea}.

\section{Selection Popup Menus}
\label{sec:selectpopups}

In \selectmode, you can also \gls{menuclick} to activate the
selection \gls{popupmenu}. The contents of this menu vary according to
what types of \glspl*{object} have been \selected, if any.

% COMMON POPUP MENU ITEMS

\menudef{index.menu.selected}

Common popup menu items are listed below.

\menudef{menu.selected.object_description}

To assign a description to the selected \gls{object}, use
\menu{selected.object_description}. This is the same as using the
\menu{edit.object_description} menu item from the main menu bar.
This option will only be available if there is exactly one
\gls{object} selected.

\menudef{menu.selected.cut}

Cut the selected \gls{object} (or \glspl{object}) to the clipboard.
This is the same as using the \menu{edit.cut} menu item from the
main menu bar. At least one \gls{object} needs to be selected.

\menudef{menu.selected.copy}

Copy the selected \gls{object} (or \glspl{object}) to the clipboard.
This is the same as using the \menu{edit.copy} menu item from the
main menu bar. At least one \gls{object} needs to be selected.

\menudef{menu.selected.paste}

Paste any \gls{object} or \glspl{object} from the clipboard onto the
\gls{canvas}.
This is the same as using the \menu{edit.paste} menu item from the
main menu bar.

\menudef{menu.selected.select_all}

Select all \glspl{object}.
This is the same as using the \menu{edit.select_all} menu item from the
main menu bar.

\menudef{menu.selected.deselect_all}

Deselect all \glspl{object}.
This is the same as using the \menu{edit.deselect_all} menu item from the
main menu bar. There must be at least one \gls{object} selected.

\menudef{menu.selected.group}

Group all selected \glspl{object}. There must be more than one
object selected.
This is the same as using the \menu{transform.group} menu item from the
main menu bar.

% NONE SELECTED POPUP

\menudef{index.menu.none}

\FloatFig
  {fig:selectpopups.none}
  {\includeimg{selectPopupNone}}
  {Popup Menu When No Objects are Selected}

If nothing is selected, there will only be a few options available
(\figureref{fig:selectpopups.none}).

\menudef{menu.none.image_description}

To assign a description to the whole image, use
\menu{none.image_description}. This is the same as using the
\menu{file.image_description} menu item from the main menu bar.

\menudef{menu.none.find_by_description}

To select an \gls{object} by its description, use
\menu{none.find_by_description}. This is the same as using the
\menu{navigate.by_description} menu item.

\menudef{menu.none.insert_bitmap}

To insert a \gls{bitmap}, use \menu{none.insert_bitmap}. This
is the same as using the \menu{bitmap.insert} menu item from the main menu bar.

The \menu{selected.select_all} and \menu{selected.paste} common menu
items are also available.

% PATH POPUP

\menudef{index.menu.selectedpath}

\FloatFig
  {fig:selectpopups.path}
  {\includeimg{selectPopupPath}}
  {Popup Menu when only Paths are Selected}

In addition to the \gls{index.menu.selected},
if a path is selected then the following options will be available
(\figureref{fig:selectpopups.path}).

\menudef{menu.selectedpath.edit}

To edit the path, use \menu{selectedpath.edit}. This is the same as using
the \menu{edit.path.edit} menu item from the main menu bar.
This option won't be available if there is more than one
path selected.

\menudef{menu.selectedpath.distort}

To distort the path, use \menu{selectedpath.distort}. This is the same as using
the \menu{transform.distort} menu item from the main menu bar.
This option won't be available if there is more than one
path selected.

\menudef{menu.selectedpath.line_colour}

To change the path's \pathattr{line-paint}, use
\menu{selectedpath.line_colour}. This is the same as using the
\menu{edit.path.line_colour} menu item from the main menu bar.

\menudef{menu.selectedpath.fill_colour}

To change the path's \pathattr{fill-paint}, use
\menu{selectedpath.fill_colour}. This is the same as using the
\menu{edit.fill_colour} menu item from the main menu bar.

\menudef{menu.selectedpath.all_styles}

To change all the path styles for the selected paths,
use \menu{selectedpath.all_styles}. This is the same as using the
\menu{edit.path.style.all_styles} menu item from the main menu bar.

\menudef{menu.selectedpath.linewidth}

To change the line width for the selected paths,
use \menu{selectedpath.linewidth}. This is the same as using the
\menu{edit.path.style.linewidth} menu item from the main menu bar.

\menudef{menu.selectedpath.dashpattern}

To change the line pattern (dash or solid) for the selected paths,
use \menu{selectedpath.dashpattern}. This is the same as using the
\menu{edit.path.style.dashpattern} menu item from the main menu bar.

\menudef{menu.selectedpath.capstyle}

To change the cap style for the selected paths,
use the \menu{selectedpath.capstyle} sub menu. This is the same as using the
\menu{edit.path.style.capstyle} sub menu from the main menu bar.

\menudef{menu.selectedpath.joinstyle}

To change the join style for the selected paths,
use \menu{selectedpath.joinstyle}. This is the same as using the
\menu{edit.path.style.joinstyle} menu item from the main menu bar.

To change the winding rule for the selected paths,
use the \menu{selectedpath.windingrule} sub menu. This is the same as using the
\menu{edit.path.style.windingrule} sub menu from the main menu bar.

\menudef{menu.selectedpath.marker}

The \menu{selectedpath.marker} sub menu has the following options:

\menudef{menu.selectedpath.marker.all}

To change all the path markers to the same type for the selected paths,
use \menu{selectedpath.marker.all}. This is the same as using the
\menu{edit.path.style.all_markers} menu item from the main menu bar.

\menudef{menu.selectedpath.marker.start}

To change the start marker for the selected paths,
use \menu{selectedpath.marker.start}. This is the same as using the
\menu{edit.path.style.startarrow} menu item from the main menu bar.

\menudef{menu.selectedpath.marker.mid}

To change the mid marker for the selected paths,
use \menu{selectedpath.marker.mid}. This is the same as using the
\menu{edit.path.style.midarrow} menu item from the main menu bar.

\menudef{menu.selectedpath.marker.end}

To change the end marker for the selected paths,
use \menu{selectedpath.marker.end}. This is the same as using the
\menu{edit.path.style.endarrow} menu item from the main menu bar.


% TEXT POPUP

\menudef{index.menu.selectedtext}

\FloatFig
  {fig:selectpopups.text}
  {\includeimg{selectPopupText}}
  {Popup Menu when only Text-Areas are Selected}

In addition to the \gls{index.menu.selected},
if a \gls{textarea} is selected then the following options will be available
(\figureref{fig:selectpopups.text}).

\menudef{menu.selectedtext.edit}

To edit the \gls{textarea}['s] displayed text or \LaTeX\
alternative, use \menu{selectedtext.edit}. This is the same as using
the \menu{edit.textarea.edit} menu item from the main menu bar.
This option won't be available if there is more than one
\gls{textarea} selected.

\menudef{menu.selectedtext.colour}

To change the \gls{textarea}['s] pen (or \textattr{outline}) \textattr{paint}, use
\menu{selectedtext.colour}. This is the same as using the
\menu{edit.textarea.colour} menu item from the main menu bar.

\menudef{menu.selectedtext.outline}

To toggle \textattr{outline} mode for the \gls{textarea}, use \menu{selectedtext.outline}.
This is the same as using the \menu{edit.textarea.outline} menu item
from the main menu bar.

\menudef{menu.selectedtext.fill_colour}

To change the \gls{textarea}['s] \textattr{fill-paint}, use
\menu{selectedtext.fill_colour}. This is the same as using the
\menu{edit.fill_colour} menu item from the main menu bar.
This action is only available if the \textattr{outline} setting is on.

\menudef{menu.selectedtext.all_styles}

To change all the text styles for the selected \glspl{textarea},
use \menu{selectedtext.all_styles}. This is the same as using the
\menu{edit.textarea.font.all_styles} menu item from the main menu bar.

\menudef{menu.selectedtext.family}

To change just the font family for the selected \glspl{textarea},
use \menu{selectedtext.family}. This is the same as using the
\menu{edit.textarea.font.family} menu item from the main menu bar.

\menudef{menu.selectedtext.size}

To change just the font size for the selected \glspl{textarea},
use \menu{selectedtext.size}. This is the same as using the
\menu{edit.textarea.font.size} menu item from the main menu bar.

\menudef{menu.selectedtext.shape}

To change just the font shape for the selected \glspl{textarea},
use \menu{selectedtext.shape}. This is the same as using the
\menu{edit.textarea.font.shape} menu item from the main menu bar.

\menudef{menu.selectedtext.series}

To change just the font series (weight) for the selected \glspl{textarea},
use \menu{selectedtext.series}. This is the same as using the
\menu{edit.textarea.font.series} menu item from the main menu bar.

\menudef{menu.selectedtext.anchor}

To change just the \textattr{anchor} (horizontal, vertical or both) for the
selected \glspl{textarea}, use the \menu{selectedtext.anchor}
sub-menu. This is the same as using the
\menu{edit.textarea.font.anchor.both} sub-menu from the main menu
bar.

\menudef{menu.selectedtext.reset}

To reset the \gls{textarea}['s] transformation matrix, use 
\menu{selectedtext.reset}. This is the same as using the 
\menu{transform.reset} menu item from the main menu bar.
This option won't be available if there is more than one
\gls{textarea} selected.

% TEXT-PATH POPUP

\menudef{index.menu.selectedtextpath}

\FloatFig
  {fig:selectpopups.textpath}
  {\includeimg{selectPopupTextPath}}
  {Popup Menu when a Text-Path is Selected}


If a text-path has been selected then the
\gls{index.menu.selectedtextpath} is very similar to the
\gls{index.menu.selectedtext} but additionally has the
\menu{selectedpath.edit} menu item to switch on edit mode for the
underlying path.

% BITMAP POPUP

\menudef{index.menu.selectedbitmap}

\FloatFig
  {fig:selectpopups.bitmap}
  {\includeimg{selectPopupBitmap}}
  {Popup Menu when only Bitmaps are Selected}

In addition to the \gls{index.menu.selected},
if a \gls{bitmap} is selected then the following options will be available
(\figureref{fig:selectpopups.bitmap}).

\menudef{menu.selectedbitmap.properties}

To edit the \gls{bitmap}['s] properties, use
\menu{selectedbitmap.properties}. This is the same as using
the \menu{bitmap.properties} menu item from the main menu bar.
This option won't be available if there is more than one
\gls{bitmap} selected.

\menudef{menu.selectedbitmap.reset}

To reset the \gls{bitmap}['s]
transformation matrix, use \menu{selectedbitmap.reset}. This is the
same as using the \menu{transform.reset} menu item from the main menu
bar.

\menudef{menu.selectedbitmap.insert}

To insert another \gls{bitmap}, use \menu{selectedbitmap.insert}. This is the
same as using the \menu{bitmap.insert} menu item from the main menu
bar.

% GROUPS OR MIXTURE

\menudef{index.menu.selectedgroup}

\FloatFig
  {fig:selectpopups.other}
  {\includeimg{selectPopupOther}}
  {Popup Menu when Groups or Mixture are Selected}

In addition to the \gls{index.menu.selected},
if any \glspl{group} or mixture of \glspl{object} are selected 
then following option will be available
(\figureref{fig:selectpopups.other}).

\menudef{menu.selected.ungroup}

To ungroup any selected \glspl{group}, use
\menu{selected.ungroup}. This is the same as using the
\menu{transform.ungroup} menu item from the main menu bar.
This option won't be available if there are no
\glspl{group} selected.

\menudef{menu.selected.path}

If any \glspl{path} have been selected or if any selected
\glspl{group} contain any \glspl{path}, then the
\menu{selected.path} sub menu will be available. This has the same
menu items as \gls{index.menu.selectedpath}.

\menudef{menu.selected.textarea}

If any \glspl{textarea} have been selected or if any selected
\glspl{group} contain any \glspl{textarea}, then the
\menu{selected.textarea} sub menu will be available. This has the same
menu items as \gls{index.menu.selectedtext}.

\menudef{menu.selected.fill_colour}

If any \glspl{path} have been selected or if any \glspl*{textarea}
have been selected with \textattr{outline} mode on, then the
\menu{selected.fill_colour} menu item will be available. This is the
same as using the \menu{edit.fill_colour} menu item from the main
menu bar.

\menudef{menu.selected.bitmap}

The \menu{selected.bitmap} sub menu has the same
menu items as \gls{index.menu.selectedbitmap}.

\menudef{menu.selected.justify}

To justify the contents of any selected \glspl{group}, use the
\menu{selected.justify} sub menu. This is the same as using the
\menu{transform.justify} sub menu from the main menu bar.
This option won't be available if there are no
\glspl{group} selected.


\section{Navigation Menu}\label{sec:navigate}

\menudef{menu.navigate}

The \menu{navigate} menu provides a way to navigate around the
\gls{canvas} and can help to \select\ \glspl{object}
in a cluttered image.

\menudef{menu.navigate.goto}

The \menu{navigate.goto} menu item opens the
\inlineglsdef{goto.title} dialog (\figureref{fig:goto}).  Enter the
$x$ and $y$ co-ordinates in the \widget{coordinates.x} and
\widget{coordinates.y} fields if you are using a rectangular grid,
or enter the angle and radius in the \widget{coordinates.angle} and
\widget{coordinates.radius} fields, if you are using a radial grid.

\FloatFig
  {fig:goto}
  {\includeimg{goto}}
  {Go To Co-ordinate Dialog Box}

If your system allows applications to move the cursor
(via the \gls{robot}), the cursor will be moved to the entered
co-ordinates (scrolling the \gls{canvas} if necessary).

\menudef{menu.navigate.select}

The \menu{navigate.select} menu item 
\selects\ the next \gls{object} in the \gls{stack}. (Starting from the
\gls{front} and heading towards the \gls{back}.) This will cycle back to
the start when it reaches the end of the \gls{stack}.

\menudef{menu.navigate.skip}

The \menu{navigate.skip} menu item will deselect the
selected \gls{object} closest to the \gls{back}, and \select\
the next \gls{object} in the \gls{stack} (heading towards the \gls{back}). This
will cycle back to the start when it reaches the end of the stack.
If you have more than one \gls{object} selected, the remaining
\glspl*{object} will stay selected.

\menudef{menu.navigate.add_next}

The \menu{navigate.add_next} menu item will add
the next \gls{object} in the \gls{stack} to the selection (starting from
the \gls{front} and heading towards the \gls{back}). This will cycle back to
the start when it reaches the end of the stack.

\menudef{menu.navigate.find}

If an \gls{object} is already \selected, the \menu{navigate.find}
menu item will scroll the \gls{canvas} to ensure the \gls{object} is
in the viewport. If your system allows applications to move the
cursor (via the \gls{robot}), the cursor will be moved to the top left corner of the
\gls{object}['s] bounding box.

\menudef{menu.navigate.by_description}

The \menu{navigate.by_description} menu item will open the
\dialog{description} dialog box. This has all the \glspl{object}
listed by their description.  If a description hasn't been set, a
generic description is used.  Select the required description and
click \btn{okay} to select the corresponding \gls{object} (and
deselect any currently selected \gls{object}).

\menudef{menu.navigate.add_description}

The \menu{navigate.add_description} menu item is similar to 
\menu{navigate.by_description} but will select the \gls{object}
without deselecting any other \glspl{object} that are already
selected.

\section{Moving an Object Up or Down the Stacking Order}\label{sec:moveupordown}

\Glspl{object} are \glslink{drawing-on-canvas}{painted} on the page according to the
\gls{stackingorder}. This means that objects can
partially or wholly obscure other objects in the same location.
For example, in \figureref{fig:stackingordera} the green rectangle is
at the bottom (\gls{back}) of the stack ($z=0$), the blue circle is in
the middle of the stack ($z=1$) and the yellow triangle is at the
top (\gls{front}) of the stack ($z=2$). The stacking order has
been reversed in \figureref{fig:stackingorderb}, with the green
rectangle now at the top (front) of the stack ($z=2$) and
the yellow triangle now at the bottom (back) of the stack
($z=0$).

\FloatSubFigs{fig:stackingorder}
{
  {fig:stackingordera}{\includeimg{stackingordera}}{},
  {fig:stackingorderb}{\includeimg{stackingorderb}}{}
}
{Stacking Order}

If \glspl{object} don't overlap or if they don't have a
\gls{index.fill-colour}, the \gls{stackingorder} may not be
immediately apparent, but the order is important when you're
defining frames for the \sty{flowfram} package. You can cycle the
selection in reverse order from front to back of the stack using the
\menu{navigate.select} menu item.

\menudef{menu.edit.front}

Each new \gls{object} is automatically added to the \gls{front} when it is
created, but \glspl{object} can also be moved to the \gls{front} 
with the \menu{edit.front} menu item.

\menudef{menu.edit.back}

A selected \gls{object} can be moved to the
\gls{back} of the \gls{stack} (so that other \glspl{object} in the
same location obscure it) with the \menu{edit.back} menu item.

\menudef{menu.edit.moveup}

An object can be moved up the \gls{stackingorder} using
the \menu{edit.moveup} menu item.

\menudef{menu.edit.movedown}

An object can be moved down the \gls{stackingorder} using
the \menu{edit.movedown} menu item.


\section{Grouping and Ungrouping Objects}\label{sec:grouping}

A \gls{group} is a collection of \glspl{object} that are treated as
though they are a single entity. When you select a \gls{group}, you
will only see the \gls{bbox} of the entire group, not the
\glspl*{bbox} of each individual object within the group. If a
\gls{group} is \hyperref[sec:rotateobjects]{rotated},
\hyperref[sec:scaleobjects]{scaled} or \hyperref[sec:shearobjects]{sheared}, each
\gls*{object} within the group will maintain its relative position.
Objects within a \gls{group} may also be
\hyperref[sec:alignobjects]{aligned}.  Note that a \gls{group} can
not be edited. Grouping and ungrouping \glspl*{object} may change
the \gls{stackingorder}.

\menudef{menu.transform.group}

To group \glspl*{object}, first \select\
all the objects you want in the \gls{group}, and then select the
\menu{transform.group} menu item.

\menudef{menu.transform.ungroup}

To \inlineglsdef{ungroup} a group, first \select\ the
group, and then select the \menu{transform.ungroup} menu item. Note
that this function is not recursive: if a group contains other
groups when you ungroup the outer group, the inner groups will
remain.  Any description assigned to a group will be lost when it's
ungrouped.

\begin{warning}
Note that if you ungroup an \gls*{object} containing
\hyperref[sec:flowframe]{flowframe} related
data, the \gls{flowframe} information will be lost. If you group objects
containing flowframe data, and then assign that group flowframe data,
any flowframe data assigned to the contents of that group will be
removed.
\end{warning}

\seealsorefs{
 sec:rotateobjects,
 sec:scaleobjects,
 sec:shearobjects,
 sec:moveupordown,
 sec:alignobjects
}


\section{Aligning Objects}\label{sec:alignobjects}

\menudef{menu.transform.justify}

It is only possible to align \glspl{object} that form part of a
\gls{group}. Objects within a group can be aligned vertically or
horizontally using the \menu{transform.justify} sub menu.

Note that alignment is not recursive: if a group contains another
group, the contents of the sub group will not be aligned, each
element in the sub group will be moved by the same amount.

\menudef{menu.transform.justify.left}

The \menu{transform.justify.left} menu item will move all objects
within the \gls{group} so that the left edge of each object's \gls{bbox}
lies along the left edge of the group's \gls*{bbox}.  (See
\figureref{fig:alignex1b}.)

\menudef{menu.transform.justify.centre}

The \menu{transform.justify.centre} menu item will move all objects
within the \gls{group} so that they are \manmsg{centred} horizontally
within the group's \gls*{bbox}.  (See \figureref{fig:alignex1c}.)

\menudef{menu.transform.justify.right}

The \menu{transform.justify.right} menu item will move all objects
within the \gls{group} so that the right edge of each object's \gls*{bbox}
lies along the right edge of the group's \gls*{bbox}.  (See
\figureref{fig:alignex1d}.)

\menudef{menu.transform.justify.top}

The \menu{transform.justify.top} menu item will move all objects
within the \gls{group} so that the top of each object's \gls*{bbox} lies along the
top of the group's \gls*{bbox}.

\menudef{menu.transform.justify.middle}

The \menu{transform.justify.middle} menu item will move all objects
within the \gls{group} so that they are \manmsg{centred} vertically
within the group's \gls{bbox}.

\menudef{menu.transform.justify.bottom}

The \menu{transform.justify.bottom} menu item will move all objects
within the \gls{group} so that the bottom of each object's \gls*{bbox} lies
along the bottom of the group's \gls*{bbox}.

\FloatSubFigs{fig:alignex1}
{
  {fig:alignex1a}{\includeimg{alignex1a}}{},
  {fig:alignex1b}{\includeimg{alignex1b}}{},
  {fig:alignex1c}{\includeimg{alignex1c}}{},
  {fig:alignex1d}{\includeimg{alignex1d}}{}
}
[Aligning a Group Consisting of Three Objects]
{Aligning a group consisting of three objects: 
\subfigref{fig:alignex1a} original; 
\subfigref{fig:alignex1b} left justified; 
\subfigref{fig:alignex1c} \manmsg{centre} justified; 
\subfigref{fig:alignex1d} right justified.}

If the \widget{textconfig.anchor} \gls{checkbox} in the
\hyperref[sec:texconfig]{TeX Configuration Settings Dialog} is
selected, any \glspl{textarea} that are contained in a group that is
justified will automatically have their \textattrpl{anchor} changed.
For example, in \figureref{fig:alignex1} one of the objects is a
\gls*{textarea}.  If the auto anchor update facility is enabled, the
\gls*{textarea} in \figureref{fig:alignex1b} will have its
horizontal \textattr{anchor} changed to \widget{font.anchor.left},
in \figureref{fig:alignex1c} it will have its horizontal
\textattr{anchor} changed to \widget{font.anchor.hcentre} and in
\figureref{fig:alignex1d} it will have its horizontal
\textattr{anchor} changed to \widget{font.anchor.right}. Similarly,
applying a vertical alignment will change the vertical
\textattr{anchor} to one of: \widget{font.anchor.top},
\widget{font.anchor.vcentre} or \widget{font.anchor.bottom}.

\begin{information}
Note that there is no way of aligning \glspl{textarea} along their
baseline. (However it is possible that the baseline may coincide
with the bottom of the \gls*{textarea} if the \gls*{textarea} doesn't contain
any characters with descenders.)
\end{information}

\strong{Tip:} Sometimes you might want to \manmsg{centre} an
\gls{object} relative to another thinner object. In this case it's
better to create a \hyperref[sec:rectangles]{rectangle}
\manmsg{centred} on the thin object that encompasses all the objects
you want to justify.  Include this rectangle in the \gls{group},
justify, ungroup and then delete the rectangle. For example, the
image shown in \figureref{fig:alignex2a} has a \gls{textarea} below
the middle line. It would look better if the text was
\manmsg{centred} below the line, so I grouped the middle line and
\gls{textarea} and justified them using
\menu{transform.justify.centre}. The result is shown in
\figureref{fig:alignex2b}. Although the \gls{textarea} and line are
now \manmsg{centred} relative to each other, the line was moved to
the \manmsg{centre} of the \gls{textarea}, not the other way round.
This was not what was intended. Instead, in
\figureref{fig:alignex2c}, I created a new rectangle that is
\manmsg{centred} on the line. Since the line is on a tick mark and
the grid lock is on, it is relatively easy to create this rectangle
(much easier than trying to move the \gls{textarea} to manually
align it). I then grouped the rectangle, the middle line and the
text area and justified them using \menu{transform.justify.centre}.
The result is shown in \figureref{fig:alignex2d}. The justified
objects were then ungrouped and the rectangle was deleted to
produced \figureref{fig:alignex2e}.

\FloatSubFigs{fig:alignex2}
{
  {fig:alignex2a}{\includeimg{alignex2a}}{},
  {fig:alignex2b}{\includeimg{alignex2b}}{},
  {fig:alignex2c}{\includeimg{alignex2c}}{},
  {fig:alignex2d}{\includeimg{alignex2d}}{},
  {fig:alignex2e}{\includeimg{alignex2e}}{}
}
[Aligning a Wider Object Relative to a Thinner Object]
{Aligning a wider object relative to a thinner object:
\subfigref{fig:alignex2a} original image;
\subfigref{fig:alignex2b} middle line and text area have
been grouped and justified;
\subfigref{fig:alignex2c} rectangle added to original
image \manmsg{centred} on the middle line;
\subfigref{fig:alignex2d} rectangle, middle line
and text area have been grouped and justified;
\subfigref{fig:alignex2e} justified
objects have been ungrouped and the rectangle has been deleted.}

\seealsorefs{
  sec:grouping,
  sec:texconfig,
  sec:neuronexample
}

\section{Cut}\label{sec:cutobjects}

\menudef{menu.edit.cut}

To cut a \selection\ of \glspl{object} to the
clipboard, use \menu{edit.cut} menu item. Note that it will be stored
on the clipboard as a \code{JDRGroup} Java object, not as text or
\gls{rastergraphics}, so you won't be able to paste it into a different
application.  If you want to cut a piece of text
from a \gls{textarea}, you will need to use the
\hyperref[sec:edittext]{edit text area} function.
Note that the \hyperref[sec:preamble]{preamble panes} have
their own cut button.

\section{Copy}\label{sec:copyobjects}

\menudef{menu.edit.copy}

To copy a \selection\ of \glspl{object} to the
clipboard, use \menu{edit.copy}. Note that it will be stored
on the clipboard as a \code{JDRGroup} Java object, not as text or
\gls{rastergraphics}, so you won't be able to paste it into a different
application.  If you want to copy a piece of text from a
\gls{textarea}, you will need to use the
\hyperref[sec:edittext]{edit text area} function.
Note that the \hyperref[sec:preamble]{preamble panes} have
their own copy button.

\section{Paste}\label{sec:pasteobjects}

\menudef{menu.edit.paste}

To paste a \selection\ of \glspl{object} from the clipboard, use
\menu{edit.paste}. If you want to copy text from another
application, and paste it into a \gls{textarea} in \FlowframTk, you
will have to \hyperref[sec:newtext]{create a new text area}, and
use the \gls{index.menu.textarea} to paste the text into the
\gls*{textarea}.  If you want to paste plain text into an existing
\gls*{textarea}, you will need to use the
\hyperref[sec:edittext]{edit text area} function.  Note that the
\hyperref[sec:preamble]{preamble panes} have their own paste button.

\section{Object Description}\label{sec:objectdescription}

\menudef{menu.edit.object_description}

You can assign a description to a \selected\ \gls*{object} using the
\menu{edit.object_description} menu item. This will display the
dialog box shown in \figureref{fig:objectDescription}. Type the
description into the text field, and click on \btn{okay} or press
\keys{\keyref{return}}.  Note that the \gls{menu.edit.object_description}
menu item is only available if exactly one object is \selected.

\FloatFig
  {fig:objectDescription}
  {\includeimg{objectDescription}}
  {Setting an Object's Description}

The description will not appear in the image, but it can be used to
locate and select \glspl*{object} on the canvas using the
\menu{navigate.by_description} or
\menu{navigate.add_description} menu items. The description
may also be used as a comment when
\hyperref[sec:exportimage]{exporting images}, depending on the
file type.

\begin{important}
If you assign a description to a \gls{group},
you will lose the description if you later ungroup it.
\end{important}

\chapter{Paint (\Glsentrytext{manual.colour} and Shading)}
\label{sec:paintcolourandshading}

\Inlineglsdef{index.paint} is the \gls{index.colour} or \gls{shading} used for
\dglspl{text-colour}, \dglspl{line-colour} and \dglspl{fill-colour}. When selecting a
\dgls{paint} for a particular attribute, you may have a choice of:
transparent (no paint), a \dgls{paint.solid-colour} or a \dgls{gradient-paint}.
(Some attributes, such as marker \gls{index.marker.colour},
only support a \gls{index.paint.solid-colour}.)

\section{Paint Selector}\label{sec:paint}

\widgetdef{colour.title}

The \widget{colour.title} panel is used in various \gls{index.paint}
selection dialog boxes. Where only a \gls{index.paint.solid-colour} 
is supported, the radio buttons will be omitted and only the 
\widget{colour.single} panel will be available.

\widgetdef{colour.transparent}

Selecting the \widget{colour.transparent} radio button indicates no
\dgls{colour}. For example, if a \gls{index.path-attribute.line-paint} is set to
\widget{colour.transparent} then this indicates that the \gls{shape}
outline should not be \glslink{drawing-on-canvas}{drawn} or if a
\gls{index.path-attribute.fill-paint} is set to \widget{colour.transparent} then
this indicates that the \gls{shape} should not be filled.

\widgetdef{colour.single}

\FloatFig{fig:paint-solid}
{\includeimg{paint-solid}}
{Solid \Manmsg{colour} Selector}

Selecting the \widget{colour.single} radio button indicates a
\inlineglsdef{index.colour.solid} \dgls{colour} and enables the
\widget{colour.single} panel (see \figureref{fig:paint-solid}).  A
\dgls{paint.solid-colour} may be expressed as
\inlineglsdef{index.colour.rgb} (red, green, blue),
\inlineglsdef{index.colour.cmyk} (cyan, magenta, yellow, black),
\inlineglsdef{index.colour.hsb} (hue, saturation, brightness) or
\inlineglsdef{index.colour.greyscale}. In each case,
\inlineglsdef{index.colour.transparency} may be added by adjusting
the alpha channel.  If the alpha value is 100\% then the
\dgls{colour} is opaque.  Smaller values increase
\gls{translucency}, allowing \glspl{object} behind to show
\manmsg{through}.

There are swatches to the right of the \widget{colour.single} panel 
which can be used to quickly set a \manmsg{colour}. Click on the 
appropriate swatch and the \manmsg{colour} specification will be
filled in.

\widgetdef{colour.gradient}

\FloatFig{fig:paint-linear-gradient}
{\includeimg{paint-linear-gradient}}
{Gradient Paint Selector (Linear Selected)}

Selecting the \widget{colour.gradient} radio button indicates a 
\inlineglsdef{index.gradient-paint} and enables the
\widget{colour.gradient} panel (see
\figureref{fig:paint-linear-gradient}). 

A \gls{index.gradient-paint} is a shading that smoothly transitions
from a \inlineglsdef{index.gradient-paint.start-colour} to an
\inlineglsdef{index.gradient-paint.end-colour}.  There is a choice
of \gls{index.gradient-paint.linear} or
\gls{index.gradient-paint.radial}. Both are a shading that
transitions from one \dgls{colour} (the
\inlineglsdef{index.gradient-paint.start-colour}) to another (the
\inlineglsdef{index.gradient-paint.end-colour}).  The
\inlineglsdef{colour.start} and \inlineglsdef{colour.end} selectors
each have the same interface as the \widget{colour.single} panel.

\FloatSubFigs{fig:shadingex}
{
 {fig:linearshading}{\includeimg{linear-shading}}{Linear},
 {fig:radialshading}{\includeimg{radial-shading}}{Radial}
}
{Gradient Shading}

\widgetdef{colour.linear}

A \inlineglsdef{index.gradient-paint.linear} shading transitions in
a band from the \gls{index.gradient-paint.start-colour} to the
\gls{index.gradient-paint.end-colour} in the given direction. There
are eight directions which are enabled when the
\widget{colour.linear} radio button is selected: North, North-East,
East, South-East, South, South-West, West, and North-West.  The
direction buttons can either be clicked on or selected with the
keystrokes \keys{\keyref{alt}+\actualkey{1}} \ldots\
\keys{\keyref{alt}+\actualkey{8}}.

For example, \figureref{fig:linearshading} has a solid black
\pathattr{line-paint} but the \pathattr{fill-paint} is a linear
gradient paint from yellow to magenta. The direction is set to North
so the bottom of the shape is yellow and the shading transitions to
magenta at the top of the shape.

\widgetdef{colour.radial}

A \inlineglsdef{index.gradient-paint.radial} shading radiates
outwards from a starting location. There are nine locations which
become enabled with the \widget{colour.radial} radio button is
selected. These are arranged in a three by three grid referencing
regions of a \gls*{shape}['s] \gls{bbox}.
The location buttons can either be clicked on or
selected with the keystrokes \keys{\keyref{alt}+\actualkey{1}} \ldots\
\keys{\keyref{alt}+\actualkey{9}} (starting from the bottom left).

For example, \figureref{fig:radialshading} has a solid black
\pathattr{line-paint} but the \pathattr{fill-paint} is a radial
gradient paint from yellow to magenta. The location is set to the
middle bottom location. So the middle bottom area of the \gls{shape}
is yellow but the transition to magenta radiates outwards from there
in all directions (clipped by the shape).

\seealsorefs{
 sec:linepaint,
 sec:fillpaint,
 sec:textpaint,
 sec:convertcolspace,
 sec:fade,
 sec:removetrans
}

\section{\Glsentrytext{manual.colour} Space Conversions}\label{sec:convertcolspace}

\menudef{menu.edit.adjustcol}

\FlowframTk\ supports the
\inlineglsdef{index.colour.rgb} (red, green, blue),
\inlineglsdef{index.colour.cmyk} (cyan, magenta, yellow, black),
\inlineglsdef{index.colour.hsb} (hue, saturation, brightness) and
\gls{index.colour.greyscale} \manmsg{colour} spaces.

\Glspl{index.line-colour}, \glspl{index.fill-colour} and
\glspl{index.text-colour} can be converted to a different
\manmsg{colour} space.

\menudef{menu.edit.adjustcol.togrey}

\Glspl{index.line-colour}, \glspl{index.fill-colour} and
\glspl{index.text-colour} can be reduced to
\inlineglsdef{index.colour.greyscale} using
\menu{edit.adjustcol.togrey}.  Only selected \glspl{path} and
\glspl{textarea} will be affected. If you want to reduce a
\gls{bitmap} to \gls{index.colour.greyscale} you will need to use a
bitmap editor.

\menudef{menu.edit.adjustcol.rgb}

The \menu{edit.adjustcol.rgb} menu item will convert to
\gls{index.colour.rgb}.

\menudef{menu.edit.adjustcol.cmyk}

The \menu{edit.adjustcol.cmyk} menu item will convert to
\gls{index.colour.cmyk}.

\menudef{menu.edit.adjustcol.hsb}

The \menu{edit.adjustcol.hsb} menu item will convert to
\gls{index.colour.hsb}.

\begin{warning}
Note that \manmsg{colour} space conversions are not exact. You may
not end up with the desired effect. Some \filefn{export} functions
may not support the chosen \manmsg{colour} space and may convert
during the export process.
\end{warning}

\seealsorefs{
 sec:fade,
 sec:removetrans,
 sec:linepaint,
 sec:fillpaint,
 sec:textpaint
}

\section{Fade}\label{sec:fade}

\menudef{menu.edit.adjustcol.fade}

\Glspl{index.line-colour}, \glspl{index.fill-colour} and
\glspl{index.text-colour} can be faded (transparency increased)
using the \menu{edit.adjustcol.fade} menu item. Only selected
\glspl{path} and \glspl{textarea} will be affected. If you want to
fade a \gls{bitmap} you will need to use a bitmap editor that
provides that function.

\seealsorefs{
 sec:convertcolspace,
 sec:removetrans,
 sec:linepaint,
 sec:fillpaint,
 sec:textpaint
}

\section{Removing Translucency}\label{sec:removetrans}

\menudef{menu.edit.adjustcol.removetrans}

\Glspl{index.line-colour}, \glspl{index.fill-colour} and
\glspl{index.text-colour} can have the translucency removed using
the \menu{edit.adjustcol.removetrans} menu item. Only selected
\glspl{shape} and \glspl{textarea} will be affected. This function
sets the paint to none (completely transparent) if the alpha value
is less than 0.5 otherwise it sets the alpha value to 1 (opaque).
For example, \figureref{fig:removetransa} shows three shapes: a
filled rectangle with alpha set to 100\%, a blue filled circle with
alpha set to 60\% and a yellow filled circle with alpha set to 40\%.
In \figureref{fig:removetransb}, the circles have had their
translucency removed. The blue filled circle now has the alpha set
to 100\% but the other circle, which formerly had a yellow interior,
has had its \gls{index.fill-colour} removed.

\FloatSubFigs{fig:removetrans}
{
 {fig:removetransa}{\includeimg{removetransa}}{},
 {fig:removetransb}{\includeimg{removetransb}}{},
}
[Removing Translucency]
{Removing Translucency: \subfigref{fig:removetransa} original shapes;
  \subfigref{fig:removetransb} translucency removed.}

\begin{warning}
Note that PostScript doesn't support transparency so the
alpha channel will be ignored if you \filefn{export} to PostScript.
\end{warning}

\seealsorefs{
 sec:convertcolspace,
 sec:fade,
 sec:linepaint,
 sec:fillpaint,
 sec:textpaint
}

\chapter{Text}\label{sec:text}

\Glspl{textarea} consist of a single line of text. Alternative
\LaTeX\ code may be provided to use instead when the image
is \hyperref[sec:exportimage]{exported} to a \TeX-related format.
New \glspl{textarea} can be created with the
\texttool\ or \mathstool, as described in \sectionref{sec:newtext}.
An existing \gls{textarea} can be edited (\sectionref{sec:edittext}),
converted to a \gls{textpath} (\sectionref{sec:textpath}), or
converted to an \textattr{outline} (\sectionref{sec:textoutline}).

To change text style attributes (font and \textattr{anchor}) see
\sectionref{sec:textstyle}.

\section{Editing Text Areas}\label{sec:edittext}

\menudef{menu.edit.textarea.edit}

To insert or delete characters from a \gls{textarea} or \gls{textpath} first
\select\ the \gls*{textarea} or \gls*{textpath}, and then
select the \menu{edit.textarea.edit} menu item.  (Note
that you should not have any other \glspl{object} selected.) This
will display the \inlineglsdef{edittext.title} dialog box
(\figureref{fig:edittexta}) in which you can modify the text as
appropriate. If you are editing the text of a \gls{textpath}, there
will be an extra panel visible, shown in
\figureref{fig:edittextb}.

\begin{information}
You can not edit a \gls*{textarea} or \gls*{textpath} if
it belongs to a \gls{group}. Deleting all the characters within a
\gls*{textarea} or \gls*{textpath} isn't permitted and will result
in the error message \qt{\errmsg{empty_string}}.
\end{information}

\FloatSubFigs{fig:edittext}
{
  {fig:edittexta}{\includeimg{editText}}{},
  {fig:edittextb}{\includeimg{editTextPath}}{}
}
[Edit Text Dialog Box]
{Edit Text Dialog Box: \subfigref{fig:edittexta} regular text area;
\subfigref{fig:edittextb} text-path}

The \gls{index.menu.textarea} is available in the top text field.
It has the same items as for the text input field used when a \gls{textarea}
is under construction.

If you want to specify alternative text to appear in a \LaTeX\
document, click on the button marked \widget{edittext.different},
and enter the alternative in the bottom field (see
\figureref{fig:edittextb}). The \gls{index.menu.textarea} is not
available in this field, but you can still use
\accelerator{menu.textarea.select_all},
\accelerator{menu.textarea.copy}, \accelerator{menu.textarea.cut}
and \accelerator{menu.textarea.paste} to select all the text, copy
to the clipboard, cut to the clipboard or paste from the clipboard,
respectively. Note that the \hyperref[sec:exportpgf]{exported} text
may occupy a larger or smaller area in the \LaTeX\ document than it
does in \FlowframTk, so you may also need to change the
\textattr{anchor} or set the image \glspl{image.typeblock}
(see \sectionref{sec:exportdoc}).

You can use the \widget{edittext.remap} button to generate
the alternative text in the bottom field by applying
\hyperref[mi:texmappings]{mappings} to the text in the top field. If
the original alternative text starts and ends with a \gls{mshift}
symbol, the \mathsmode\ mappings will be used, otherwise the
\textmode\ mappings will be used. If the original alternative text is
empty, you can just type a \glssymbol{mshift} symbol in it and click on the
\widget{edittext.remap} button to generate a \mathsmode\
mapping. The original alternative text will be completely replaced by the text
from the top field with the appropriate mappings applied. You can
then edit this if necessary.

\widgetdef{edittext.textpath}

The \LaTeX-related \filefn{export} functions use the \sty{pgf} text
decoration to implement \glspl*{textpath}. This allows you to insert
declarations, such as \csfmt{large}, part way through the text by
placing them between \glspl{pgf.delimiter}. By default these delimiters are the
vertical bar (pipe) character \glssymbol{pipe} but this can be changed in
the \widget{edittext.textpath.left_delim} and
\widget{edittext.textpath.right_delim} fields (see
\figureref{fig:edittextb}). Note that the vertical \textattr{anchor}
setting is ignored by the \LaTeX\ export functions. It's best to avoid
\manmsg{maths} in \glspl{textpath}. See the \sty{pgf} manual for
further details about text decorations.

\begin{important}
The alternative text may not be used by \filefn{export} functions if
the \gls*{textpath} has been made an \textattr{outline} or if the
\gls*{textarea}\slash\gls*{textpath} has a gradient paint.
The export \manmsg{behaviour} under those circumstances is dependent on
the \hyperref[sec:texconfigtext]{\LaTeX\ text settings}.
The alternative text is not used when exporting to PNG or when
\gls{printing}.
\end{important}

\seealsorefs{
 sec:texconfig,
 sec:newtext,
 sec:textpath,
 sec:textoutline,
 sec:textpaint,
 sec:textstyle,
 sec:splittext,
 sec:texttopath,
 sec:neuronexample
}

\section{Text Transformation Matrix}\label{sec:textmatrix}

You can scale, rotate or shear text
using the \gls{affine-transformation} functions described in 
\sectionref{sec:affinetrans} or by directly editing the
\inlineglsdef{index.text-attribute.transformation-matrix}.

TODO


\section{Combining a Text Area and Path to Form a Text-Path}\label{sec:textpath}

\menudef{menu.transform.textpath}

A \gls{textarea} and a \gls{path} can be combined to form a
\gls{textpath}. Ensure that only a single \gls{textarea} and a
single \gls{path} are selected and use the \menu{transform.textpath}
menu item. The text from the original \gls{textarea} will then follow
the shape of the original path (which will now be the underlying
path).

The underlying path will not be visible (except in \editpathmode).
If the original \gls*{textarea} was in \textattr{outline} mode, the new
\gls*{textpath} will also be in \textattr{outline} mode, possibly with a 
\textattr{fill-paint}, depending on the original \gls*{textarea} and \gls*{path}
fill paint. (If the original \gls*{textarea} didn't have a
\textattr{fill-paint}, the \gls*{path} \pathattr{fill-paint} will be applied.)

The horizontal \textattr{anchor} determines
whether the text should start at the first \gls{controlpt} of the
underlying path or if it should be \manmsg{centred} along the path or if
it should be right aligned at the end \gls*{controlpt}. The vertical
\textattr{anchor} determines whether the base, bottom, top or middle of the
text should be aligned on the path. 

\begin{important}
The vertical \textattr{anchor} isn't implemented by the
\filefn{export} \TeX-related filters. Note that if the text is
longer than the path, the text will be truncated.
\end{important}

For example, the \gls{textarea} and \gls{path} in
\figureref{fig:textpatha} are combined to form a \gls*{textpath}.
The original \gls*{textarea}['s] horizontal \textattr{anchor} was
set to left, so the text along the path starts at the first
\gls{controlpt} in \figureref{fig:textpathb}. In
\figureref{fig:textpathc} the horizontal \textattr{anchor} has been
changed to \manmsg{centre}, and in \figureref{fig:textpathc} the
horizontal \textattr{anchor} has been changed to right.

\FloatSubFigs{fig:textpath}
{
 {fig:textpatha}{\includeimg{textpath}}{},
 {fig:textpathb}{\includeimg{textpathleft}}{},
 {fig:textpathc}{\includeimg{textpathcentre}}{},
 {fig:textpathd}{\includeimg{textpathright}}{}
}
[Combining a Text Area and Path to Form a Text-Path]
{Combining a \gls{textarea} and \gls{path} to form a \gls{textpath}:
\subfigref{fig:textpatha} the original \gls{textarea} and \gls{path};
\subfigref{fig:textpathb} the resulting \gls{textpath} with left horizontal 
\textattr{anchor}; 
\subfigref{fig:textpathc} \manmsg{centred} \textattr{anchor}; 
\subfigref{fig:textpathd} right \textattr{anchor}. 
(The \glspl{textpath} are in edit mode to show the underlying path.)}

Once a \gls{path} has been combined with a \gls{textarea}, the line
style \glspl{index.path-attribute} are lost as the path is only used
as a guide to position the text. The \gls{cag} functions, such as
\transformfn{union}, are applied to the underlying path and the text
is adjusted to follow the new path.  Transformations using the
\transformfn{rotate}, \transformfn{scale} and \transformfn{shear}
functions are applied to the underlying path not the text. You can
either transform the text using the transformation functions before
combining it with a path or transform it after combining by changing
the text \textattr{transformation-matrix}.

Note there is a difference between applying \gls{symmetry} to a
\gls{textpath} and converting a \gls*{textarea} and
\gls{symmetricshape} to a \gls*{textpath}. For example, consider the
\gls{textarea} and \gls{path} in \figureref{fig:textpathsyma}. If
you first combine them to form a \gls*{textpath}
(\figureref{fig:textpathsymb}) and then add symmetry
(\figureref{fig:textpathsymc}), the result is a \gls*{textpath}
where the text is reflected across the \gls{line-of-symmetry}.
Conversely, applying \gls{symmetry} to the path first
(\figureref{fig:textpathsymd}) and then combining with the
\gls*{textarea} yields a \gls*{textarea} where only the underlying
path has \gls{symmetry} (\figureref{fig:textpathsyme}).

A similar effect applies with other types of \glspl{compositeshape}.

\FloatSubFigs*{fig:textpathsym}
{%
 {fig:textpathsyma}{\includeimg{textpath-syma}}{},
 {fig:textpathsymb}{\includeimg{textpath-symb}}{},
 {fig:textpathsymc}{\includeimg{textpath-symc}}{},
 {fig:textpathsymd}{\includeimg{textpath-symd}}{},
 {fig:textpathsyme}{\includeimg{textpath-syme}}{}
}
[Symmetric Text-Paths]
{Symmetric \glspl{textpath}:
\subfigref{fig:textpathsyma} original \gls{textarea} and \gls{path};
\subfigref{fig:textpathsymb} \gls{textarea} and \gls{path} in 
\subfigref{fig:textpathsyma} have been combined to
form a \gls{textpath};
\subfigref{fig:textpathsymc} the \gls{textpath} in \subfigref{fig:textpathsymb} 
has had \gls{symmetry} applied to it in \editpathmode; 
\subfigref{fig:textpathsymd} the \gls{path} in
\subfigref{fig:textpathsyma} has had \gls{symmetry} applied to it;
\subfigref{fig:textpathsyme} the \gls{textarea} and symmetric path in 
\subfigref{fig:textpathsymd} have been combined to form a
\gls{textpath}.}

\begin{warning}
Restrictions apply to \glspl*{textpath} with the
\filefn{export} \TeX-related filters as some \gls*{textpath}
effects aren't emulated in \LaTeX, such as the vertical \textattr{anchor},
\gls{index.gradient-paint} and \textattrpl{outline}. You can determine 
the \manmsg{behaviour} on export via the \dialog{texconfig} dialog.
\end{warning}

\seealsorefs{
 sec:edittext,
 sec:separate,
 sec:patterns
}

\section{Text Outlines}\label{sec:textoutline}

\menudef{menu.edit.textarea.outline}

A \gls{textarea} or \gls{textpath} can be rendered as an
\inlineglsdef{index.text-attribute.outline} by
selecting it and then using the \menu{edit.textarea.outline}
menu item. For example, \figureref{fig:textoutlinea} shows a
standard \gls*{textarea} and \figureref{fig:textoutlineb} shows
that \gls*{textarea} rendered as an outline. If the outline mode is
set, you can apply a \gls{index.fill-colour} using
\menu{edit.fill_colour}.

\begin{warning}
Note that although \glspl*{textpath} can be rendered with an outline in \FlowframTk,
the \LaTeX\ \filefn{export} functions can't emulate
this and will either export the \gls*{textpath} without the outline
setting or will export it as a path (in which case the alternative
text will be ignored) depending on your
\hyperref[sec:texconfig]{\TeX\ Configuration Setting}.
\end{warning}

\FloatSubFigs{fig:textoutline}
{
 {fig:textoutlinea}{\includeimg{outlinea}}{},
 {fig:textoutlineb}{\includeimg{outlineb}}{},
}
[Text Outline]
{Text Outline: \subfigref{fig:textoutlinea} original text area;
  \subfigref{fig:textoutlineb} outline.}

% TODO document \jdroutline

\section{Splitting Text Areas}\label{sec:splittext}

\menudef{menu.transform.split}

TODO

\chapter{Modifying Shapes}\label{sec:modshape}

\Glspl{shape} are defined by their segment types (line, \gls{curve}
or \glspl{gap}) and \glspl{controlpt}, which may be adjusted in edit
path mode (see \sectionref{sec:editpath}). It's also possible to
modify \glspl{shape} using \glspl{affine-transformation} (see
\sectionref{sec:affinetrans}), \gls{cag} (see \sectionref{sec:cag}),
distortion (see \sectionref{sec:distort}), reversing the order of
the \glspl{controlpt} and segments (see \sectionref{sec:reversing}), merging paths (see
\sectionref{sec:mergepaths}), or converting into one or more other
\glspl{shape} (see \sectionref{sec:converttopolygon},
\sectionref{sec:separate}, \sectionref{sec:patterns},
\sectionref{sec:converttopath}, \sectionref{sec:texttopath},
\sectionref{sec:outlinetopath}). 

\section{Edit Path Mode}\label{sec:editpath}

\menudef{menu.edit.path.edit}

\Glspl{path} consist of a sequence of lines, curves and \glspl{gap} 
that are defined by \glspl{controlpt}. These \glspl{controlpt} may be adjusted in
\emph{\inlineglsdef{edit-path-mode}} which can be switched on and off
with the \menu{edit.path.edit} toggle menu item.

\begin{information}
Edit mode can't be enabled for a \gls{shape} that is part of a \gls{group}.
If the selected shape is a \gls{compositeshape}, edit mode will allow the
underlying path to be edited.
\end{information}

The \gls{menu.edit.path.edit} menu item
is only available if exactly one \gls{shape} and no other
\gls{object} is \selected.

\subsection{Control Points}\label{sec:editcontrolpt}

The \glspl{controlpt} that define a \gls{shape} only become visible
and interactive when the shape is in \editpathmode.
To move, delete or add \glspl{controlpt}, open or close
\glspl{path}, or to convert segments from one form (line, \gls{gap}, cubic
\gls{Bezier-curve}) to another, first \select\ the \gls*{shape}, and
then either click on the \gls{menu.edit.path.edit} icon or select
\menu{edit.path.edit}.

The \gls*{path} will then be displayed in draft format.  The
currently selected \gls*{controlpt} and the currently selected
segment will appear in red. The other \glspl*{controlpt} will be
orange.

\begin{information}
The \manmsgpl{colour} can be changed via the \widget{graphics.title}
tab in the \dialog{configui} dialog. The size of the
\glspl{controlpt} can be changed in the \widget{controls.title} tab
of the \dialog{config} dialog. You may also want to adjust the
\widget{editpathui.title} in the \widget{graphics.title} tab of the
\dialog{configui} dialog.
\end{information}

\Glspl{compositeshape} may also be edited with 
the \gls{menu.edit.path.edit} function. The
\glspl{controlpt} will only be those on the underlying path and 
the special controls that define the \gls{compositeshape}.

A \gls{textpath} object can have its underlying path edited in the
same way as a normal \gls{path}, but in edit mode you will also see
the text (without \gls{anti-aliasing}).  Note that you can not edit
a \gls{path} if it belongs to a \gls{group}; you must first
\hyperref[sec:grouping]{ungroup it}.

\begin{information}
\Glspl{controlpt} rather than segments are selectable. The segment
that the selected \gls{controlpt} belongs to will then be
highlighted. You can either select a \gls{controlpt} with the mouse
or use the \gls{menu.editpath.next_control} or \gls{menu.editpath.prev_control}
items in the \gls{index.menu.editpath}.
\end{information}

A selected \gls{controlpt} can be moved by dragging with the mouse.
In a cluttered image with overlapping \glspl{controlpt} it can be
difficult to select and move the desired \gls{controlpt}.
If two or more points coincide
with the location of the mouse, the point with the lowest
\glslink{ptindex}{index} will be selected.
You may find it easier to use the \gls{index.menu.editpath}.

Dragging doesn't automatically select a \gls{controlpt} otherwise it
would be easy to accidentally select a coincident \gls{controlpt}
when trying to move a currently selected \gls{controlpt}.
This means that you can initiate dragging outside of the selected
\gls*{controlpt}, but make sure that you
first click to change the selection before initiating a drag or you
may move the wrong control.

\begin{important}
With the \widget{editpathui.canvasclick} option on, if you
click outside of any \gls{controlpt} the path edit mode will be
switched off. Otherwise you will need to toggle the
\menu{edit.path.edit} menu item to switch off path edit mode.
\end{important}

If the \gridlock\ is on, mouse clicks will be translated to the
nearest tick mark, so even if the pointer is positioned over a
control point, the nearest tick mark may be outside the control
point bounds. Enabling the \widget{editpathui.ignorelock} option in 
the \widget{graphics.title} panel will ignore the \gridlock\ for the
purposes of selecting \glspl{controlpt}.

\menudef{index.menu.editpath}

In path edit mode, the \gls{popupmenu} will change from the usual
\selectmode\ menus to provide functions to edit the \gls{path}.

\FloatFig{fig:editPathPopup}
 {\includeimg{editPathPopup}}
 {Edit Path Popup Menu}

\menudef{menu.editpath.next_control}

The \menu{editpath.next_control} menu item will select the next \gls{controlpt}
(wrapping round to the start from the final \gls{controlpt}).
This is an alternative to using the mouse to select a point.

\menudef{menu.editpath.prev_control}

The \menu{editpath.prev_control} menu item will select the previous \gls{controlpt}
(wrapping round to the end from the first \gls{controlpt}).
This is an alternative to using the mouse to select a point.

\menudef{menu.editpath.delete_control}

The \menu{editpath.delete_control} menu item will delete the current
\gls{controlpt}. 
Note that certain \glspl{controlpt}, such as curvature controls or points
the define a \gls{compositeshape}, can't be deleted.

If the \gls{controlpt} is the first or last point in an open path
this function will delete the corresponding segment, otherwise it
will replace two adjacent segments with a single segment.  If the
path is open and only has one segment, or if the path is closed and
has two segments, deleting a control point will delete the
\gls{shape} or the \gls{textpath} object. (The text will also be lost when
the \gls{textpath} is deleted.)

\menudef{menu.editpath.add_control}

The \menu{editpath.add_control} menu item will add a new
\gls{controlpt} in the middle of the currently selected segment.
This actually replaces the selected segment with two new segments of 
half the length of the original that maintain the same shape.
This function is only available for normal path segments and not for
segments used to alter \glspl{compositeshape}.

\menudef{menu.editpath.convert_to_line}

The \menu{editpath.convert_to_line} menu item will convert the
current segment to a line. This function will be unavailable if the
current segment is already a line or if it isn't a normal path segment.

\menudef{menu.editpath.convert_to_curve}

The \menu{editpath.convert_to_curve} menu item will convert the
current segment to a \gls{Bezier-curve}. The new curvature 
\glspl{controlpt} can then be moved as required.
This function will be unavailable if the
current segment is already a \gls{Bezier-curve} or if it isn't a normal path segment.

\menudef{menu.editpath.convert_to_move}

The \menu{editpath.convert_to_move} menu item will convert the
current segment to a \gls{gap} (\qt{move to}). This function will be
unavailable if the current segment is already a \gls{gap} or if it
isn't a normal path segment.

\menudef{menu.editpath.symmetry}

The \menu{editpath.symmetry} sub menu can be used to add or remove
\gls{symmetry}.
See \sectionref{sec:symmetric} for further details.

\menudef{menu.editpath.continuity}

If the selected segment is a \gls{Bezier-curve}, the
\menu{editpath.continuity} sub menu provides functions that adjust
the curvature \gls*{controlpt} to ensure that the gradient at the
nearest join is continuous. This sub menu isn't available if it's not
possible to do this (for example, if the nearest join is an end
point). See \sectionref{sec:continous} for further details.

\menudef{menu.editpath.open_path}

The \menu{editpath.close_path} sub menu provides functions to open a
closed path.

\menudef{menu.editpath.open_path.remove_last}

The \menu{editpath.open_path.remove_last} menu item
opens the path, removing the last segment (\figureref{fig:openpathb}).

\menudef{menu.editpath.open_path.keep_last}

The \menu{editpath.open_path.keep_last} menu item
opens the path, but keeps the last segment (\figureref{fig:openpathb}).

\FloatSubFigs{fig:openpath}
{
 {fig:openpatha}{\includeimg{open-patha}}{},
 {fig:openpathb}{\includeimg{open-pathb}}{},
 {fig:openpathc}{\includeimg{open-pathc}}{}
}
  [Opening a Path]
  {Opening a path: \subfigref{fig:openpatha} the original closed path;
   \subfigref{fig:openpathb} the path in \subfigref{fig:openpatha} was opened,
   removing the final segment; \subfigref{fig:openpathc} the path in 
   \subfigref{fig:openpatha} was opened, keeping the last segment.}

\menudef{menu.editpath.close_path}

The \menu{editpath.close_path} sub menu provides functions to join
the end of the \gls{shape} to its start to close an open path.

\menudef{menu.editpath.close_path.line}

The \menu{editpath.close_path.line} menu item will
close the path with a line between the last and first
\glspl*{controlpt} of the original path (\figureref{fig:closepatha}).

\menudef{menu.editpath.close_path.cont}

The \menu{editpath.close_path.cont} menu item will
close the path with a \gls{Bezier-curve} between the last and first
\glspl*{controlpt} such that the curve is continuous at the join
between the first and last segments of the original path
(\figureref{fig:closepathb}).

\menudef{menu.editpath.close_path.merge}

The \menu{editpath.close_path.merge} menu item will
close the path, merging the last \gls*{controlpt} of the original
path with the first \gls*{controlpt} (\figureref{fig:closepathc}).

\FloatSubFigs {fig:closepath}
{%
  {fig:closepatha}{\includeimg{close-patha}}{},
  {fig:closepathb}{\includeimg{close-pathb}}{},
  {fig:closepathc}{\includeimg{close-pathc}}{},
  {fig:closepathd}{\includeimg{close-pathd}}{}
}
[Closing a Path]
{Closing a path: \subfigref{fig:closepatha} the original path;
\subfigref{fig:closepathb}
the path in \subfigref{fig:closepatha} was closed with a line;
\subfigref{fig:closepathc} the path in \subfigref{fig:closepatha} was closed with a
curve continuous at the join between adjacent segments;
\subfigref{fig:closepathd} the path in \subfigref{fig:closepatha} was closed, merging
the end points}

\menudef{menu.editpath.coordinates}

The \menu{editpath.coordinates} menu item will display the
\inlineglsdef{coordinates.title} dialog box in which you can set the
\gls{controlpt}['s] $x$ and $y$ values (instead of dragging the
point to the required location).  Note that rounding errors may
occur if the unit used in this dialog doesn't have a convenient
conversion factor with the \gls{storageunit}.

\plabel[Segment Details]{mi:segmentinfo}% HelpSet id
\menudef{menu.editpath.info}

The \menu{editpath.info} menu item will option the 
\inlineglsdef{segmentinfo.title} dialog box shown in
\figureref{fig:segmentinfo}, which describes the segment type and
provides numeric fields in which to adjust the co-ordinates of the
segment. Note that the end \gls{controlpt} won't be enabled unless 
it's the end point of an open path.

Any changes made in this dialog won't take effect until you click on
the \btn{okay} button. The \btn{cancel} button will discard all the
changes (after prompting for confirmation). Use the 
\widget{segmentinfo.default} button to revert the working segment
back to its current state in the actual path.

\FloatFig{fig:segmentinfo}
{\includeimg{segmentinfo}}
{Segment Details Dialog}

\menudef{menu.editpath.snap}

The \menu{editpath.snap} menu item will
move the currently selected \gls*{controlpt} to the nearest
tick mark.

\plabel[Breaking a Path]{mi:breakpath}
\menudef{menu.editpath.break_path}

The \menu{editpath.break_path} menu item will split the current path
into two distinct \glspl{path} at the \emph{end} of the currently
selected segment (not at the currently selected \gls*{controlpt}).
One path will remain in \editpathmode. 
If the object is a \gls*{textpath}, the new \glspl*{textpath} will
both have the same text (that is, the text is not broken between
them).

\seealsorefs{
 sec:reversing,
 sec:mergepaths,
 sec:pathunion,
 sec:xorpath,
 sec:pathintersect,
 sec:pathsubtract,
 sec:converttopath,
 sec:linepaint,
 sec:fillpaint,
 sec:styles,
 sec:toastexample,
 sec:busexample,
 sec:accesstutorial
}

\subsection{Gradient Continuity}\label{sec:continous}

Joins between \gls{path} segments, where one of both segments is a
\gls{Bezier-curve}, can be made continuous (a smooth join)
in \editpathmode\ with the \menu{editpath.continuity} sub menu.

\menudef{menu.editpath.continuity.equi}

The \menu{editpath.continuity.equi} menu item will move the selected
curvature \gls*{controlpt} so that it has the same gradient
direction and magnitude as the gradient vector on the other side of
the join.
This function is only available if the selected \gls{controlpt} is
a curvature control.

\menudef{menu.editpath.continuity.relative}

The \menu{editpath.continuity.relative} menu item will move the
selected curvature control so that it has the same direction as the
gradient on the other side of the join, but its magnitude will
remain unchanged. (See \figureref{fig:continuous}.)
This function is only available if the selected \gls{controlpt} is
a \gls{curvature-controlpt}.

\menudef{menu.editpath.continuity.anchor}

The \menu{editpath.continuity.anchor} menu item is only available
when a \gls*{controlpt} on the join between two \glspl{Bezier-curve}
has been selected. If this item is selected, when you adjust one of
the adjacent \glspl{curvature-controlpt}, the corresponding curvature
control on the other segment will be adjusted to maintain
\gls{continuity}. An \gls{continuity.anchor} image will appear in the control joining the
two segments when this setting is on (as shown in
\figureref{fig:continuityanchor}).

For example, in \figureref{fig:continuous} the path was originally
an open line \gls{path} with three line segments.  The middle segment was
selected and converted to a \gls{Bezier-curve} using the
\menu{editpath.convert_to_curve} function
(\figureref{fig:continuousa}). The
\menu{editpath.continuity.relative} function was then
used to change the starting \gls{gradient} of the \gls{Bezier-curve} to make
a smooth join between the first two segments
(\figureref{fig:continuousb}).  The \gls{Bezier-curve}['s] third
\gls*{controlpt}, which governs the end curvature, was selected, and
the \menu{editpath.continuity.relative} function was
again used to change the end \gls{gradient} of the \gls{Bezier-curve} to
make a smooth join between the last two segments
(\figureref{fig:continuousc}).

\FloatSubFigs{fig:continuous}
{
 {fig:continuousa}{\includeimg{continuous1a}}{},
 {fig:continuousb}{\includeimg{continuous1b}}{},
 {fig:continuousc}{\includeimg{continuous1c}}{}
}
[Making the Join Between Segments Continuous]
{Making the join between segments continuous: 
\subfigref{fig:continuousa} the middle segment of
an open line path has been converted into a \gls{Bezier-curve};
\subfigref{fig:continuousb} the gradient at the start of the curve is now the same as
the gradient at the end of the previous segment; 
\subfigref{fig:continuousc} the
gradient at the end of the curve is now the same as the gradient at
the start of the next segment.}

\FloatFig
  {fig:continuityanchor}
   {\includeimg{continuityanchor}}
   {Continuity Anchor}

\seealsorefs{sec:editpath}

\subsection{Symmetric Shapes}\label{sec:symmetric}

\Glspl{symmetricshape} are created by applying
\inlineglsdef{symmetry} to an existing \gls{shape} in 
\editpathmode\ via the \menu{editpath.symmetry} sub menu.

\FloatSubFigs{fig:symmetric}
{
  {fig:symmetrica}{\includeimg{symmetrica}}{},
  {fig:symmetricb}{\includeimg{symmetricb}}{},
  {fig:symmetricc}{\includeimg{symmetricc}}{},
  {fig:symmetricd}{\includeimg{symmetricd}}{},
  {fig:symmetrice}{\includeimg{symmetrice}}{}
}
[Adding Symmetry to a Path]
{Adding Symmetry to a Path: \subfigref{fig:symmetrica} original path;
\subfigref{fig:symmetricb} symmetry added to path in \subfigref{fig:symmetrica} the
two blue controls govern the line of symmetry;
\subfigref{fig:symmetricc} the line of symmetry has been moved, altering the
overall appearance of the shape; \subfigref{fig:symmetricd} the end anchor
constraint has been removed and the end control has been moved
away from the line of symmetry; \subfigref{fig:symmetrice} the joining segment
has been converted to a curve with only one curvature control.}

\menudef{menu.editpath.symmetry.has_symmetry}

The \menu{editpath.symmetry.has_symmetry} menu item can be toggled
to add or remove \gls{symmetry}. Adding symmetry will replace the existing
\gls{shape} with a \gls{symmetricshape}. This has extra
\glspl{controlpt} that govern the \inlineglsdef{line-of-symmetry}.
Removing \gls{symmetry} will replace the existing \gls{symmetricshape}
with its underlying \gls{shape}.

For example, \figureref{fig:symmetrica} shows a \gls*{path}
in \editpathmode, and \figureref{fig:symmetricb} shows the
path with \gls{symmetry} applied to it. There are now two extra
\glspl{controlpt} (\manmsg{coloured} blue). These points govern the 
\gls{line-of-symmetry}. In \figureref{fig:symmetricc}, these two controls
have been moved.

\begin{information}
The \gls{line-of-symmetry} extends infinitely though the two controls,
but only the part of the line between the two points is actually
displayed in edit mode.
\end{information}

If you later decide to remove the symmetry, deselect
\menu{editpath.symmetry.has_symmetry}.

\begin{warning}
Adding \gls{symmetry} to a closed shape may cause unexpected
results as the shape will be first opened (without removing the
last segment), the \gls{symmetry} will be added, and then the symmetric
shape will be closed, merging the end points.
\end{warning}

\menudef{menu.editpath.symmetry.join_anchor}

The \menu{editpath.symmetry.join_anchor} menu item can be toggled to
determine whether or not the reflection is
\glslink{line-of-symmetry.anchor}{anchored} to the underlying
path at its end point. If this option is on, the end \gls{controlpt}
will be constrained to the \gls{line-of-symmetry}.  That is, the end
\gls{controlpt} can only move along the line defined by the two blue
\glspl*{controlpt}.  In \figureref{fig:symmetricd}, the
\menu{editpath.symmetry.join_anchor} item was deselected, and the
end control was then moved away from the \gls{line-of-symmetry}.

Note that this function places a \gls{gap} (move) segment between the
end control and its symmetric counterpart, which will produce an
unsymmetric effect if the path is then closed. This gap can be changed to
a line or curve, using \menu{editpath.convert_to_line} or
\menu{editpath.convert_to_curve}, as described in
\sectionref{sec:editpath}. In \figureref{fig:symmetrice},
the join has been changed to a curve. Unlike the \glspl{Bezier-curve}
in the non-symmetric paths, this curve only has one 
\gls{curvature-controlpt}.

The \gls{menu.editpath.symmetry.join_anchor} function is only available
if \gls{menu.editpath.symmetry.has_symmetry} is on.

\menudef{menu.editpath.symmetry.close_anchor}

The \menu{editpath.symmetry.close_anchor} menu item can be toggled
to determine whether or not the reflection is
\glslink{line-of-symmetry.anchor}{anchored} to the
underlying path at is start point.  If this option is on, the start
\gls{controlpt} will be constrained to the \gls{line-of-symmetry}.  This
function is only available if
\gls{menu.editpath.symmetry.has_symmetry} is on and the
underlying \gls{path} is closed.

For example, \figureref{fig:closedsymmetrica} shows a closed
symmetric path (a closed version of \figureref{fig:symmetrice}). The
\gls{line-of-symmetry.anchor} constraint on the first control was
then removed and the control was moved to the left
(\figureref{fig:closedsymmetricb}). As with the join segment (above)
the closing segment between the start control and its reflection can
be changed to a curve with one \gls{curvature-controlpt}
(\figureref{fig:closedsymmetricc}).

\FloatSubFigs{fig:closedsymmetric}
{
 {fig:closedsymmetrica}{\includeimg{closedsymmetrica}}{},
 {fig:closedsymmetricb}{\includeimg{closedsymmetricb}}{},
 {fig:closedsymmetricc}{\includeimg{closedsymmetricc}}{}
}
[Closed Symmetric Path]
{Closed symmetric path: \subfigref{fig:closedsymmetrica} the symmetric path in
\figureref{fig:symmetrice} has been closed\dash the
first control is now anchored to the line of symmetry;
\subfigref{fig:closedsymmetricb} deselecting the close anchor constraint allows
the start control to be moved away from the line of symmetry;
\subfigref{fig:closedsymmetricc} the segment closing the symmetric path has been
changed to a curve.}

\begin{warning}
Take care with closed symmetric paths. Unexpected results
may occur, particularly if the path contains any gaps. This may
cause the stroked or filled shape to appear unsymmetric.
For example, \figureref{fig:symgapa} shows the original
(non-symmetric path). This was then given a line of symmetry and
the path appears symmetric, as shown in \figureref{fig:symgapb}. In
\figureref{fig:symgapc}, two of the line segments have been
converted to gaps. The shape still appears symmetric although the
filled area has changed, but in \figureref{fig:symgapd} the path
has been closed with a line so although the control points that
make up right side of the complete path are a reflection of the
original points on the left side, the shape no longer appears
symmetric.
\end{warning}

\FloatSubFigs{fig:symgap}
{
  {fig:symgapa}{\includeimg{symgapa}}{},
  {fig:symgapb}{\includeimg{symgapb}}{},
  {fig:symgapc}{\includeimg{symgapc}}{},
  {fig:symgapd}{\includeimg{symgapd}}{}
}
[Symmetric Path with Gaps]
{Symmetric Path with Gaps:
\subfigref{fig:symgapa} the original path;
\subfigref{fig:symgapb} symmetry has been added;
\subfigref{fig:symgapc} two line segments have been replaced with gaps;
\subfigref{fig:symgapd} the path has been closed.}

\seealsorefs{
 sec:editpath,
 sec:patterns,
 sec:rosetutorial
}

\section{Merging Paths}\label{sec:mergepaths}

\menudef{menu.transform.merge}

Multiple \glspl{shape} can be merged into a single \gls*{shape}
using the \menu{transform.merge} menu item. Note that this is not
the same as \hyperref[sec:grouping]{grouping}.  A \gls{gap} (move
to) will be placed between the last \gls{controlpt} of one path and
the first \gls*{controlpt} of the next path. Any \gls{pattern} in
the selection will be first \hyperref[sec:converttopath]{converted
to a full path} before merging.

\begin{information}
To split an existing \gls{path} into two separate \glspl{path}, use the
\menu{editpath.break_path} popup menu item in \editpathmode.
\end{information}

Once the shape has been merged, it can then be edited as usual in
\editpathmode.  If the original shapes had different styles, the new
shape will retain the style of the first shape (the lowest one in
the \gls{stack}). For example, in \figureref{fig:mergea} there are
two \glspl{path} with different styles. \figureref{fig:mergeb} shows
the single \gls{path} created from merging the two original paths.
Since the first path used the even-odd winding rule, the new shape
has a hole in it.

\FloatSubFigs{fig:merge}
{
  {fig:mergea}{\includeimg{mergepatha}}{},
  {fig:mergeb}{\includeimg{mergepathb}}{}
}
[Merging Two Paths]
{Merging two paths:
\subfigref{fig:mergea} the first path has a solid line pattern, a
green fill \manmsg{colour} and even-odd winding rule, and the second
path has a dashed line pattern and a yellow fill \manmsg{colour};
\subfigref{fig:mergeb} resulting merged path has a solid line
pattern, green fill \manmsg{colour} and even-odd winding rule.}

The same applies if one or more of the selected \glspl*{object} is a
\gls{textpath}. For example, in \figureref{fig:mergetextpathsa}
there are two \glspl*{textpath}.  These are merged to form a single
\gls*{textpath} shown in \figureref{fig:mergetextpathsb}. Note that
the text from the second \gls{textpath} is lost. The resulting path
is shown in \editpathmode\ in \figureref{fig:mergetextpathsc} to
illustrate the underlying path. A mixture of \glspl*{path} and
\glspl*{textpath} can be merged. The resulting \gls*{object} will be
a \gls*{textpath} if the first object to be merged is a
\gls*{textpath}, otherwise it will be a \gls*{path}.

\FloatSubFigs{fig:mergetextpaths}
{
  {fig:mergetextpathsa}{\includeimg{mergetextpathsa}}{},
  {fig:mergetextpathsb}{\includeimg{mergetextpathsb}}{},
  {fig:mergetextpathsc}{\includeimg{mergetextpathsc}}{}
}
[Merging Two Text-Paths]
{Merging two text-paths:
\subfigref{fig:mergetextpathsa} the first path is on the left and
the second path is on the right;
\subfigref{fig:mergetextpathsb} resulting merged path;
\subfigref{fig:mergetextpathsc} resulting merged path in
edit mode to illustrate the underlying path}

\Glspl{path} are merged according to their \gls{stackingorder}. For
example, in \figureref{fig:merge2a} there are two \glspl{path}, both
with a bar start marker, and an arrow end marker. The path on the
right is further back in terms of the \gls{stackingorder}. (That is,
it gets \glslink{painting}{painted} on the \gls{canvas} before the
other path.) \figureref{fig:merge2b} shows the result of merging the
two paths\dash the left hand path has been appended to the right
hand path. \figureref{fig:merge2c} shows the same two paths as in
\figureref{fig:merge2a} except that now the left path is the
\gls{back}. There is no visible difference between
\figureref{fig:merge2a} and \figureref{fig:merge2c}, but the result
of merging the paths in \figureref{fig:merge2c} (see
\figureref{fig:merge2d}) is different to
\figureref{fig:merge2b}\dash the right hand path has been appended
to the left hand path.

\FloatSubFigs{fig:merge2}
{
  {fig:merge2a}{\includeimg{mergepath2a}}{},
  {fig:merge2b}{\includeimg{mergepath2b}}{},
  {fig:merge2c}{\includeimg{mergepath2a}}{},
  {fig:merge2d}{\includeimg{mergepath2d}}{}
}
[Paths Are Merged According to the Stacking Order]
{Paths are merged according to the stacking order:
\subfigref{fig:merge2a} two straight line paths where the path 
on the right is at the back of the stack;
\subfigref{fig:merge2b} new single path resulting from merging 
the two paths in \subfigref{fig:merge2a};
\subfigref{fig:merge2c} same as \subfigref{fig:merge2a} but the path on the left
is at the back of the stack;
\subfigref{fig:merge2d} new single path resulting from merging 
the two paths in \subfigref{fig:merge2c}.}

\seealsorefs{
 sec:pathunion,
 sec:xorpath,
 sec:pathintersect,
 sec:pathsubtract,
 sec:reversing,
 sec:moveupordown,
 sec:cheeseexample
}

\section{Reversing a Path's Direction}\label{sec:reversing}

\menudef{menu.transform.reverse}

The direction of a \gls{path} or \gls{textpath} can be reversed
using the \menu{transform.reverse} menu item. For example, the path
in \figureref{fig:reverse} has a bar start marker, pointed arrow
mid-markers and a \LaTeX\ style arrow end marker.
\figureref{fig:reversea} shows the original path, and
\figureref{fig:reverseb} shows the reversed path. Note that all the
\glspl{controlpt} are in the same place, but their ordering has
changed, so that what was originally the first \gls{controlpt} is
now the last \gls{controlpt}.

\FloatSubFigs{fig:reverse}
{
  {fig:reversea}{\includeimg{reverse1a}}{},
  {fig:reverseb}{\includeimg{reverse1b}}{}
}
[Reversing the Direction of a Path]
{Reversing the direction of a path:
\subfigref{fig:reversea} original path;
\subfigref{fig:reverseb} reversed path\dash
the vertices are in the same location, but the order has been
reversed.}

In \figureref{fig:reversetextpath}, the \gls*{textpath} in
\figureref{fig:reversetextpatha} is reversed to form 
\figureref{fig:reversetextpathb}. Note that the text now starts
from the right instead of the left, since the first \gls*{controlpt}
is now on the right, and it is upside-down.

\FloatSubFigs{fig:reversetextpath}
{
  {fig:reversetextpatha}{\includeimg{reversetextpatha}}{},
  {fig:reversetextpathb}{\includeimg{reversetextpathb}}{}
}
[Reversing the Direction of a Text-Path]
{Reversing the direction of a text-path:
\subfigref{fig:reversetextpatha} original text-path;
\subfigref{fig:reversetextpathb} reversed text-path\dash 
the vertices are in the same location, but the order has been
reversed so the text starts from the other end.}

\section{Convert to Polygon}\label{sec:converttopolygon}

If you want to convert a single \gls{curve} segment to a single
line segment, you can simply enable \editpathmode, and use the 
\menu{editpath.convert_to_line} item from the popup menu.
However, if you want to replace every curve with one or more straight
lines then you may prefer to use the \transformfn{convert_to_polygon}
function.

\menudef{menu.transform.convert_to_polygon}

The \gls{curve} segments of a \gls{path} (or underlying path)
can be converted into a series of straight line segments with the
\menu{transform.convert_to_polygon} menu item, which will result in
a polygonal shape. The number of lines used to replace an individual
curve segment is determined by the \emph{flatness} parameter.

\begin{information}
There is a maximum limit of 1024 replacement line segments per
\gls{curve}.
\end{information}

First \select\ the \gls{shape} with \glspl{curve} that need to be transformed.
For example, in \figureref{fig:converttopolygona} a closed \gls{path}
with four \gls{curve} segments has been selected.

\FloatSubFigs{fig:converttopolygon}
{
  {fig:converttopolygona}{\includeimg{converttopolygona}}{},
  {fig:converttopolygonb}{\includeimg{converttopolygonb}}{}
}
[Converting a Shape With Curves to a Polygon]
{Converting a shape with curves to a polygon:
\subfigref{fig:converttopolygona} original shape with four
curve segments;
\subfigref{fig:converttopolygonb} converted shape (flatness 4.0) 
with fourteen line segments.}


\widgetdef{polygon.title}

The \menu{transform.convert_to_polygon} menu item will open the
\dialog{polygon} dialog, shown in \figureref{fig:polygondialog}.
This will show a draft outline of the selected shape (in this case,
the shape from \figureref{fig:converttopolygona}).

\FloatFig{fig:polygondialog}
{\includeimg{polygondialog}}
{Convert to Polygon Dialog}

\widgetdef{polygon.flatness}

The \widget{polygon.flatness} numeric field should be set to the
flatness parameter. This is the maximum allowable distance (in \gls{bp})
between the \glspl{controlpt} and the flattened \gls{curve}. A small
value will create a series of line segments that are as close to the
original curve as possible. This may result in many short line
segments depending on the curvature of the original \gls{curve}.

\widgetdef{polygon.dotask}

Once you have set the desired flatness parameter, click on the
\widget{polygon.dotask} button. This will draw the calculated
polygon over the original shape for comparison. For example, 
\figureref{fig:polygondialog1} shows the calculated polygon with
the flatness parameter set to 1.

\FloatFig{fig:polygondialog1}
{\includeimg{polygondialog1}}
{Convert to Polygon Dialog: Result with 1.0 Flatness}

On the right-hand side of the \dialog{polygon} dialog is an
information panel that details the number of lines used, the length
of the polygon's perimeter, the polygon's area and the \gls{xor} area.
It also lists the path's definition where \qt{M} indicates a move to
(followed by the co-ordinate pair to move to), \qt{L} indicates a line to
(followed by the co-ordinate pair to draw a line to), and \qt{Z}
indicates close path.

If the result isn't quite satisfactory, you can adjust the flatness
parameter and click on the \widget{polygon.dotask} button again.
For example, I changed the flatness to 4 and recalculated. This
produced the polygon shown in \figureref{fig:polygondialog2} which
has fewer line segments with a greater deviation from the original
shape.

\FloatFig{fig:polygondialog2}
{\includeimg{polygondialog2}}
{Convert to Polygon Dialog: Result with 4.0 Flatness}

\widgetdef{polygon.reload}

You can clear the dialog and reset it back to just the original
selected shape with the \widget{polygon.reload} button.
There is also a zoom widget on the bottom left which can be used to
adjust the magnification.

If you want to replace the selected path with the
calculated polygon, click on the \widget{okay} button.  For example,
\figureref{fig:converttopolygonb} is the result with the flatness set
to 4. Note that this new polygonal shape has more segments that the 
original curving shape.

\begin{information}
The \widget{okay} button is only enabled once a polygon has been
calculated. If you change you mind you can click on the \widget{cancel}
button to discard the draft. You will be prompted for confirmation
if a polygon has been calculated.
\end{information}


\section{Affine Transformations}\label{sec:affinetrans}

\FlowframTk\ provides functions to perform 
\inlineglsdef{affine-transformation} on \glspl{object}:
\affinetrans{translation} (\sectionref{sec:moveobjects}), 
\affinetrans{rotation} (\sectionref{sec:rotateobjects}), 
\affinetrans{scaling} (\sectionref{sec:scaleobjects}) and 
\affinetrans{shearing} (\sectionref{sec:shearobjects}).


\subsection{Moving an Object}\label{sec:moveobjects}

To move an \gls{object} or \glspl*{object}, first \select\ the
\glspl*{object} you want to move and then depress the
\glslink{primaryclick}{primary mouse button} somewhere inside the
selection, and drag the mouse.  Release the mouse button when the
\glspl*{object} have reached their required location.

\begin{important}
If the \gridlock\ is on, the \selected\
\gls{object} (or \glspl{object}) will move
in increments of the minor tick distance.
\end{important}

If you only want to move the object by a very small amount, and your
mouse is very sensitive or you have difficulties with fine motor
co-ordination, depress the mouse button and instead of dragging use
the arrow keys to move the pointer. (This will require the
\gls{robot} to be available.)

\menudef{menu.edit.moveby}

An alternative if you need to move a small amount or if you need to
move by an exact amount is to use the \menu{edit.moveby} menu item.
This will open the \dialog{moveby} dialog box (\figureref{fig:moveby}).
In the field marked \widget{coordinates.x} enter the horizontal
displacement, and in the field marked \widget{coordinates.y}
enter the vertical (\strong{downward}) displacement. For example, to move the
selected \glspl*{object} 10 units to the right and 20 units down,
type 10 in the \widget{coordinates.x} field and 20 in the
\widget{coordinates.y} field. To move left or up, use a negative
value.

\begin{information}
The \menu{edit.moveby} function is unaffected by the \gridlock.
\end{information}

\FloatFig
  {fig:moveby}
  {\includeimg{moveby}}
  {Move Selected Objects Dialog Box}


\subsection{Rotating Objects}\label{sec:rotateobjects}

\menudef{menu.transform.rotate}

Selected \glspl{object} can be rotated with the
\menu{transform.rotate} menu item. This will display a dialog box in
which you can specify the angle of rotation.

Notes:
\begin{itemize}
\item Individual \glspl{object}
will be rotated relative to the \manmsg{centre} of the \gls{object}.

\item \Glspl*{object} within a
\gls{group} will be rotated relative to the \manmsg{centre} of the
group.

\item Rotating a \gls{textpath} will rotate the path and the text
will adjust to follow the transformed path.

\item Rotating a \gls*{textarea} and \gls*{path} and then combining
them to form a \gls*{textpath} is not the same as first combining
and then rotating.

\end{itemize}

To illustrate this, in \figureref{fig:rotateex1a} there are three
\glspl*{object} selected.  The selection is then rotated
\degrees{90}.  The result is shown in \figureref{fig:rotateex1b}.

\FloatSubFigs{fig:rotateex1}
{
 {fig:rotateex1a}{\includeimg{rotateex1a}}{},
 {fig:rotateex1b}{\includeimg{rotateex1b}}{}
}
[Three Selected Objects Rotated by 90 Degrees]
{Three selected objects rotated by 90 degrees: 
\subfigref{fig:rotateex1a} before;
\subfigref{fig:rotateex1b} after.}

In \figureref{fig:rotateex2}, the three objects in
\figureref{fig:rotateex1} were first \hyperref[sec:grouping]{grouped}
(\figureref{fig:rotateex2a}) and then rotated \degrees{90}
(\figureref{fig:rotateex2b}).

\FloatSubFigs{fig:rotateex2}
{
 {fig:rotateex2a}{\includeimg{rotateex2a}}{},
 {fig:rotateex2b}{\includeimg{rotateex2b}}{}
}
[A Group Consisting of Three Objects Rotated by 90 Degrees]
{A group consisting of three objects rotated by 90 degrees:
\subfigref{fig:rotateex2a} before; \subfigref{fig:rotateex2b} after.}

In \figureref{fig:rotatetextpath}, the \gls*{path} and
\gls*{textarea} in \figureref{fig:rotatetextpatha} are
combined into a \gls*{textpath}, shown in
\figureref{fig:rotatetextpathb}. This \gls*{textpath} is then
rotated by \degrees{90} resulting in
\figureref{fig:rotatetextpathc}. Note that this is different
from first rotating the original \gls*{path} and \gls*{textarea},
shown in \figureref{fig:rotatetextpathd}, and then combining
them to form a \gls*{textpath}, shown in
\figureref{fig:rotatetextpathe}.

\FloatSubFigs*{fig:rotatetextpath}
{
 {fig:rotatetextpatha}{\includeimg{rotatetextpatha}}{},
 {fig:rotatetextpathb}{\includeimg{rotatetextpathb}}{},
 {fig:rotatetextpathc}{\includeimg{rotatetextpathc}}{},
 {fig:rotatetextpathd}{\includeimg{rotatetextpathd}}{},
 {fig:rotatetextpathe}{\includeimg{rotatetextpathe}}{}
}
[Rotating a Text-Path]
{Rotating a text-path: 
\subfigref{fig:rotatetextpatha} original text area and path; 
\subfigref{fig:rotatetextpathb} text area and path in
\subfigref{fig:rotatetextpatha} combined to form a text-path;
\subfigref{fig:rotatetextpathc} text-path in
\subfigref{fig:rotatetextpathb} rotated by 45 degrees;
\subfigref{fig:rotatetextpathd} text area and path in
\subfigref{fig:rotatetextpatha} rotated by 45 degrees;
\subfigref{fig:rotatetextpathe} rotated text area and
path in \subfigref{fig:rotatetextpathd} combined to form a text-path.}

If you prefer to rotate an \gls{object} using the mouse (which is
less precise), you first need to \hyperref[mi:hotspots]{enable the hotspots}.
Then drag the bottom left hotspot to rotate. Note that even if you
have more than one object selected, only the object whose hotspot you
are dragging will be transformed.

\seealsorefs{
 sec:grouping,
 sec:moveobjects,
 sec:scaleobjects,
 sec:shearobjects,
 sec:textpath
}


\subsection{Scaling Objects}\label{sec:scaleobjects}

\menudef{menu.transform.scale}

Selected \glspl{object} can be scaled with the
\menu{transform.scale} menu item. This will open up a dialog box
in which you can specify the scale factor. There is a choice of
scaling just the $x$ dimension, just the $y$ dimension
or both dimensions.

Notes:
\begin{itemize}
\item \Inlineglsdef{affine-transformation.reflection} can be
achieved with a negative scale factor.

\item Individual \glspl*{object} will be
scaled relative to the top left corner of the
object's \gls{bbox}.

\item \Glspl*{object} within a
\gls{group} will be scaled relative to the top left corner
of the group's \gls*{bbox}.

\item Scaling a \gls{textpath} will scale the path and the text
will adjust to follow the transformed path. Note that the text
itself will not be scaled.

\item Scaling a \gls*{textarea} and \gls*{path} and then combining
them to form a \gls*{textpath} is not the same as first combining
and then scaling.
\end{itemize}

To illustrate this, in \figureref{fig:scaleex1a} there are three
\glspl*{object} selected.
The selection is then scaled by a factor of 2.
The result is shown in \figureref{fig:scaleex1b}.

\FloatSubFigs{fig:scaleex1}
{
 {fig:scaleex1a}{\includeimg{scaleex1a}}{},
 {fig:scaleex1b}{\includeimg{scaleex1b}}{},
}
[Three Selected Objects Scaled by a Factor of 2]
{Three selected objects scaled by a factor of 2:
\subfigref{fig:scaleex1a} before;
\subfigref{fig:scaleex1b} after.}

In \figureref{fig:scaleex2}, the three objects in
\figureref{fig:scaleex1} were first
\hyperref[sec:grouping]{grouped} (\figureref{fig:scaleex2a})
and then scaled by a factor of 2 (\figureref{fig:scaleex2b}).

\FloatSubFigs{fig:scaleex2}
{
 {fig:scaleex2a}{\includeimg{scaleex2a}}{},
 {fig:scaleex2b}{\includeimg{scaleex2b}}{}
}
[A Group Consisting of Three Objects Scaled by a Factor of 2]
{A group consisting of three objects scaled by a factor of 2:
\subfigref{fig:scaleex2a} before;
\subfigref{fig:scaleex2b} after.}

In \figureref{fig:scaletextpath}, the \gls*{path} and
\gls*{textarea} in \figureref{fig:scaletextpatha} are combined
into a \gls*{textpath}, shown in
\figureref{fig:scaletextpathb}. This \gls*{textpath} is then
scaled by a factor of 2 resulting in
\figureref{fig:scaletextpathc}. Note that this is different
from first scaling the original \gls*{path} and \gls*{textarea},
shown in \figureref{fig:scaletextpathd}, and then combining
them to form a \gls*{textpath}, shown in
\figureref{fig:scaletextpathe}.

\FloatSubFigs*{fig:scaletextpath}
{
 {fig:scaletextpatha}{\includeimg{scaletextpatha}}{},
 {fig:scaletextpathb}{\includeimg{scaletextpathb}}{},
 {fig:scaletextpathc}{\includeimg{scaletextpathc}}{},
 {fig:scaletextpathd}{\includeimg{scaletextpathd}}{},
 {fig:scaletextpathe}{\includeimg{scaletextpathe}}{}
}
[Scaling a Text-Path]
{Scaling a text-path:
\subfigref{fig:scaletextpatha} original text area and path;
\subfigref{fig:scaletextpathb} text area and path in
\subfigref{fig:scaletextpatha} combined to form a text-path;
\subfigref{fig:scaletextpathc} text-path 
in \subfigref{fig:scaletextpathb} scaled by a factor of 2;
\subfigref{fig:scaletextpathd} text area and path 
in \subfigref{fig:scaletextpatha} scaled by a factor of 2;
\subfigref{fig:scaletextpathe} scaled text area and path 
in \subfigref{fig:scaletextpathd} combined to form a text-path.}

If you prefer to scale an \gls{object} using the mouse (which is
less precise), you first need to \hyperref[mi:hotspots]{enable the
hotspots}. Then drag the bottom \manmsg{centre} hotspot to scale
vertically, the bottom right hotspot to scale in both directions or
the middle right hotspot to scale horizontally. Note that even if
you have more than one object selected, only the object whose
hotspot you are dragging will be transformed.


\seealsorefs{
 sec:grouping,
 sec:moveobjects,
 sec:rotateobjects,
 sec:shearobjects,
 sec:textpath
}


\subsection{Shearing Objects}\label{sec:shearobjects}

\menudef{menu.transform.shear}

Selected \glspl{object} can be sheared with
the \menu{transform.shear} menu item. This will open up a dialog box
in which you can specify the shear factors. The shearing
transformation is given by:
\TeXParserLibToImage
[div=displaymath,alt={
 (
   [ 1 , sX ]
   [ sY , 1 ]
 )
 (
   [ x ]
   [ y ]
 )
=
 (
   [ x + sX \texttimes\ y ]
   [ y + sY \texttimes\ x ]
 )
}]
{
\[
  \left(
  \begin{array}{cc}
  1 & s_x\\
  s_y & 1
  \end{array}
  \right)
  \left(
  \begin{array}{c}
  x\\
  y
  \end{array}
  \right)
  =
  \left(
  \begin{array}{l}
  x + s_x y\\
  y + s_y x
  \end{array}
  \right)
  \]
}

Notes:
\begin{itemize}
\item Individual \glspl*{object}
will be sheared relative to the bottom left corner of the
object's \gls{bbox}.

\item \Glspl*{object} within a 
\gls{group} will be sheared relative to the bottom
left corner of the group's \gls*{bbox}.

\item Shearing a \gls{textpath} will shear the path and the text
will adjust to follow the transformed path. Note that the text
itself will not be sheared.

\item Shearing a \gls*{textarea} and \gls*{path} and then combining
them to form a \gls*{textpath} is not the same as first combining
and then shearing. 

\item Shearing a \gls{pattern} will shear the underlying path not
the complete shape.
\end{itemize}

To illustrate this, in \figureref{fig:shearex1a} there are three
\glspl*{object} selected.  The selection is then sheared with shear
factors $s_x=1$ and $s_y=0$.  The result is shown in
\figureref{fig:shearex1b}.

\FloatSubFigs{fig:shearex1}
{
 {fig:shearex1a}{\includeimg{shearex1a}}{},
 {fig:shearex1b}{\includeimg{shearex1b}}{}
}
[Two Selected Objects Sheared Horizontally]
{Two selected objects sheared horizontally:
\subfigref{fig:shearex1a} before;
\subfigref{fig:shearex1b} after.}

In \figureref{fig:shearex2}, the three objects in
\figureref{fig:shearex1} were first \hyperref[sec:grouping]{grouped}
(\figureref{fig:shearex2a}) and then sheared with shear factors
$s_x=1$ and $s_y=0$ (\figureref{fig:shearex2}).

\FloatSubFigs{fig:shearex2}
{
 {fig:shearex2a}{\includeimg{shearex2a}}{},
 {fig:shearex2b}{\includeimg{shearex2b}}{}
}
[A Group Consisting of Two Objects Sheared Horizontally]
{A group consisting of two objects sheared horizontally:
\subfigref{fig:shearex2a} before;
\subfigref{fig:shearex2b} after.}

In \figureref{fig:sheartextpath}, the \gls*{path} and
\gls*{textarea} in \figureref{fig:sheartextpatha} are combined into
a \gls*{textpath}, shown in \figureref{fig:sheartextpathb}. This
\gls*{textpath} is then sheared with shear factors $s_x=1$ and
$s_y=0$ (\figureref{fig:sheartextpathc}). Note that this is
different from first shearing the original \gls*{path} and
\gls*{textarea}, shown in \figureref{fig:sheartextpathd}, and then
combining them to form a \gls*{textpath}, shown in
\figureref{fig:sheartextpathe}.

\FloatSubFigs*{fig:sheartextpath}
{
 {fig:sheartextpatha}{\includeimg{sheartextpatha}}{},
 {fig:sheartextpathb}{\includeimg{sheartextpathb}}{},
 {fig:sheartextpathc}{\includeimg{sheartextpathc}}{},
 {fig:sheartextpathd}{\includeimg{sheartextpathd}}{},
 {fig:sheartextpathe}{\includeimg{sheartextpathe}}{}
}
[Shearing a Text-Path]
{Shearing a text-path:
\subfigref{fig:sheartextpatha} original text area and path; 
\subfigref{fig:sheartextpathb} text area and path 
in \subfigref{fig:sheartextpatha} combined to form a text-path; 
\subfigref{fig:sheartextpathc} text-path 
in \subfigref{fig:sheartextpathb} sheared horizontally; 
\subfigref{fig:sheartextpathd} text area and path 
in \subfigref{fig:sheartextpatha} sheared horizontally; 
\subfigref{fig:sheartextpathe} sheared text area and path 
in \subfigref{fig:sheartextpathd} combined to form a text-path.}

In \figureref{fig:shearpatterna}, a \gls*{path} created using the
ellipse tool has been converted into a rotational pattern with two
replicas. This pattern is then sheared with shear factors $s_x=1$
and $s_y=0$, shown in \figureref{fig:shearpatternb}. This is
different from the effect obtained by applying the same shear
factors to a complete path rather than a pattern.
\figureref{fig:shearpatternc} is a full path version of
\figureref{fig:shearpatternd}. This path is then sheared using the
same factors and the result is shown in
\figureref{fig:shearpatternd}.

\FloatSubFigs{fig:shearpattern}
{
 {fig:shearpatterna}{\includeimg{shearpatternexa}}{},
 {fig:shearpatternb}{\includeimg{shearpatternexb}}{},
 {fig:shearpatternc}{\includeimg{shearpatternexa}}{},
 {fig:shearpatternd}{\includeimg{shearpatternexd}}{}
}
[Shearing a Pattern]
{Shearing a pattern: 
\subfigref{fig:shearpatterna} a pattern consisting of an
ellipse that has 2 rotational replicas; 
\subfigref{fig:shearpatternb} the pattern 
in \subfigref{fig:shearpatterna} has been sheared horizontally;
\subfigref{fig:shearpatternc} the pattern 
in \subfigref{fig:shearpatterna} has been converted to a full path; 
\subfigref{fig:shearpatternd} the path 
in \subfigref{fig:shearpatternc} has been sheared horizontally.}

If you prefer to shear an \gls{object} using the mouse,
you first need to \hyperref[mi:hotspots]{enable the hotspots}.
Then drag the top right hotspot to shear vertically
or the top left hotspot to shear horizontally. Note that even if you
have more than one object selected, only the object whose hotspot you
are dragging will be transformed.

\seealsorefs{
 sec:grouping,
 sec:moveobjects,
 sec:rotateobjects,
 sec:scaleobjects,
 sec:textpath
}

\section{Constructive Area Geometry (CAG)}\label{sec:cag}

\subsection{Path Union}\label{sec:pathunion}

\menudef{menu.transform.union}

Area \inlineglsdef{cag.addition} can be performed on
\glspl{shape} with the \menu{transform.union} menu item. That is,
multiple \glspl{shape} can be combined into a single \gls*{shape} by
performing a union on all the selected \glspl*{shape}.

At least two \glspl*{shape} must be selected to perform this
function. As with the \hyperref[sec:mergepaths]{merge path function}, the
new shape has the same styles as the \glslink{back}{backmost} path
in the selection.

For example, in \figureref{fig:pathuniona}, there are three
overlapping paths.  In \figureref{fig:pathunionb} the paths have
been replaced by a single path created using the path union
function. For comparison, the same three paths in
\figureref{fig:pathuniona} were replaced using the
\hyperref[sec:mergepaths]{merge function}. The result is shown in
\figureref{fig:pathunionc}.

\FloatSubFigs*{fig:pathunion}
{
  {fig:pathuniona}{\includeimg{geom}}{},
  {fig:pathunionb}{\includeimg{pathunion}}{},
  {fig:pathunionc}{\includeimg{geommerge}}{}
}
[Path Union]
{Path union:
\subfigref{fig:pathuniona} original paths (the rear path has an
orchid fill \manmsg{colour} 2bp line width and round join style);
\subfigref{fig:pathunionb} the three paths in
\subfigref{fig:pathuniona} have been replaced by a single path using
the path union function;
\subfigref{fig:pathunionc} for comparison, the three paths in
\subfigref{fig:pathuniona} have been replaced by a single path using
the merge paths function.}

In \figureref{fig:textpathunion}, a \gls*{textpath} and a
\gls*{path} are combined: \figureref{fig:textpathuniona} shows
the original objects and \figureref{fig:textpathunionb} shows
the resulting object. In this case, the resulting object is a
\gls*{textpath} since the \glslink{back}{backmost} path in
\figureref{fig:textpathuniona} was the \gls*{textpath} object.

\FloatSubFigs{fig:textpathunion}
{
  {fig:textpathuniona}{\includeimg{textpathuniona}}{},
  {fig:textpathunionb}{\includeimg{textpathunionb}}{}
}
[Text-Path Union]
{Text-path union: \subfigref{fig:textpathuniona} original text-path
and path; \subfigref{fig:textpathunionb} objects in
\subfigref{fig:textpathuniona} have been replaced with text-path.}

Any \gls{pattern} in the selection will first be converted to a full
\gls{path} before the union is applied. For example, in
\figureref{fig:patternunion}, two patterns are combined:
\figureref{fig:patternuniona} shows the original patterns (both have
a rotational pattern with 2 replicas) and
\figureref{fig:patternunionb} shows the resulting path. This path is
illustrated in edit mode in \figureref{fig:patternunionc} to show
that it is now a full path.

\FloatSubFigs*{fig:patternunion}
{
  {fig:patternuniona}{\includeimg{patternunionexa}}{},
  {fig:patternunionb}{\includeimg{patternunionexb}}{},
  {fig:patternunionc}{\includeimg{patternunionexc}}{}
}
[Pattern Union]
{Pattern union: \subfigref{fig:patternuniona} original patterns;
\subfigref{fig:patternunionb} patterns in
\subfigref{fig:patternuniona} have been combined to form a full
path; \subfigref{fig:patternunionc} result shown in edit mode.}

\seealsorefs{
 sec:mergepaths,
 sec:xorpath,
 sec:pathintersect,
 sec:pathsubtract,
 sec:reversing,
 sec:busexample
}

\subsection{Exclusive Or Function}\label{sec:xorpath}

\menudef{menu.transform.xor}

Area \inlineglsdef{cag.xor} can be performed on \glspl{shape} with
the \menu{transform.xor} menu item. That is, multiple \glspl{shape}
can be combined into a single \gls*{shape} by performing an
\gls{cag.xor} operation on all the selected \glspl*{shape}.  At
least two \glspl*{shape} must be selected to perform this function.
As with the \hyperref[sec:mergepaths]{merge path function}, the new
\gls*{shape} has the same styles as the \glslink{back}{backmost}
path in the selection and \glspl{pattern} will be converted to full
\glspl{path}.

For example, in \figureref{fig:xorpatha}, there are three
overlapping paths.  The rear path has a non-zero winding rule. In
\figureref{fig:xorpathb} the paths have been replaced by a
single path created using the \transformfn{xor} function. For comparison,
the same three paths in \figureref{fig:xorpatha} were replaced
using the \hyperref[sec:mergepaths]{merge function}. The result is
shown in \figureref{fig:xorpathc}. Both paths in
\figureref{fig:xorpathb} and \figureref{fig:xorpathc} use
a non-zero winding rule, since that was used by the rear path in
\figureref{fig:xorpatha}.

\FloatSubFigs*{fig:xorpath}
{
 {fig:xorpatha}{\includeimg{geom}}{},
 {fig:xorpathb}{\includeimg{xorpaths}}{},
 {fig:xorpathc}{\includeimg{geommerge}}{}
}
[Exclusive Or Function]
{Exclusive Or function: 
\subfigref{fig:xorpatha} original paths (the rear path has 
a non-zero winding rule, orchid
fill \manmsg{colour} and round join style);
\subfigref{fig:xorpathb} the three paths in
\subfigref{fig:xorpatha} have been replaced by 
a single path using the exclusive or function; 
\subfigref{fig:xorpathc} for comparison, the three paths in
\subfigref{fig:xorpatha} have been replaced by 
a single path using the merge paths function.}

\seealsorefs{
 sec:mergepaths,
 sec:pathunion,
 sec:pathintersect,
 sec:pathsubtract,
 sec:reversing
}

\subsection{Path Intersection}\label{sec:pathintersect}

\menudef{menu.transform.intersect}

Area \inlineglsdef{cag.intersection} can be performed on \glspl{shape} with
the \menu{transform.intersect} menu item. That is, multiple \glspl{shape}
can be combined into a single \gls*{shape} by performing an
\gls{cag.intersection} operation on all the selected \glspl*{shape}.  At
least two \glspl*{shape} must be selected to perform this function, 
and at least two of the shapes (or underlying path
in the case of a \gls{textpath}) must overlap.
As with the \hyperref[sec:mergepaths]{merge path function}, the new
\gls*{shape} has the same styles as the \glslink{back}{backmost}
path in the selection and \glspl{pattern} will be converted to full
\glspl{path}.

For example, in \figureref{fig:pathintersecta},
there are three overlapping paths.  In
\figureref{fig:pathintersectb} the paths have been replaced by
a single path created using the path intersect function.

\FloatSubFigs{fig:pathintersect}
{
  {fig:pathintersecta}{\includeimg{geom}}{},
  {fig:pathintersectb}{\includeimg{intersectpaths}}{}
}
[Path Intersection Function]
{Path intersection function: 
\subfigref{fig:pathintersecta} original paths (the rear path has 
an orchid fill \manmsg{colour} and round join style); 
\subfigref{fig:pathintersectb} the three paths 
in \subfigref{fig:pathintersecta} have been
replaced by a single path using the path intersect function.}

\seealsorefs{
 sec:mergepaths,
 sec:pathunion,
 sec:xorpath,
 sec:pathsubtract,
 sec:reversing
}

\subsection{Path Subtraction}\label{sec:pathsubtract}

\menudef{menu.transform.subtract}

Area \inlineglsdef{cag.subtraction} can be performed on \glspl{shape} with
the \menu{transform.subtract} menu item. That is, multiple \glspl{shape}
can be combined into a single \gls*{shape} by performing a
\gls{cag.subtraction} operation on all the selected \glspl*{shape}.  At
least two \glspl*{shape} must be selected to perform this function.
The new shape is the \glslink{back}{backmost} selected shape with
the other selected shapes subtracted from it. Any \glspl{pattern} in
the selection will be converted to full paths.

For example, in
\figureref{fig:pathsubtracta}, there are three overlapping
paths.  In \figureref{fig:pathsubtractb} the paths have been
replaced by a single path created using the path subtraction
function.

The new path will be a \gls*{textpath} if the backmost selected
object was a \gls*{textpath} and the text will adjust to fit the
new underlying path. For example, in
\figureref{fig:textpathsubtracta}, there is a \gls*{textpath}
and a \gls*{path}. The \gls*{textpath} is the backmost path.
In \figureref{fig:textpathsubtractb}, the two objects have
been replaced by a single \gls*{textpath} using the path subtraction
function.

\FloatSubFigs{fig:pathsubtract}
{
  {fig:pathsubtracta}{\includeimg{geom}}{},
  {fig:pathsubtractb}{\includeimg{subtractpaths}}{}
}
[Path Subtraction Function]
{Path subtraction function:
\subfigref{fig:pathsubtracta} original paths (the rear path 
has an orchid fill \manmsg{colour} and round join style);
\subfigref{fig:pathsubtractb} the three paths 
in \subfigref{fig:pathsubtracta} have
been replaced by a single path using the path subtraction function.}

\FloatSubFigs{fig:textpathsubtract}
{
  {fig:textpathsubtracta}{\includeimg{subtracttextpatha}}{},
  {fig:textpathsubtractb}{\includeimg{subtracttextpathb}}{}
}
[Subtracting From a Text-Path]
{Subtracting from a text-path: 
\subfigref{fig:textpathsubtracta} original text-path and path; 
\subfigref{fig:textpathsubtractb} the path has been 
subtracted from the underlying path of the text-path.}

\seealsorefs{
 sec:mergepaths,
 sec:pathunion,
 sec:xorpath,
 sec:pathintersect,
 sec:reversing,
 sec:busexample
}

\section{Distorting Shapes}\label{sec:distort}

\menudef{menu.transform.distort}

A selected \gls{path} (not a \gls{compositeshape}) can be distorted using the
\menu{transform.distort} menu item. While this toggle
button\slash item is on, the \gls{path} is in distortion mode.
This has four round controls that are initially located at each
corner of the path's \gls{bbox}. These controls can be moved to
distort the path.  Note that this modifies the location of the
path's \glspl{controlpt}. The stroke attributes, such as the line
width, aren't modified. Once you have finished distorting the shape,
uncheck the \menu{transform.distort} menu item (or the
button on the horizontal \gls{toolbar}) or click
anywhere outside the (original) bounding box.

For example, \figureref{fig:distortpatha} shows a circle in 
\editpathmode. In \figureref{fig:distortpathb}, the \editpathmode\ is now off and
the circle is selected. In \figureref{fig:distortpathc}, the
distortion mode has now been switched on. The round distortion
controls are initialised to each corner of the \gls{bbox}. The
controls that define the path aren't visible. (Note that the
\hyperref[sec:graphics]{rendering hints} have been set to speed
and no anti-aliasing while in distortion mode.)

In \figureref{fig:distortpathd}, three of the round distortion
controls have been moved. The original bounding box is still
visible, even though it no longer encompasses the distorted shape.
In \figureref{fig:distortpathe}, the distortion mode has been
switched off and normal select mode has resumed. The modified path
is now showing with its new \gls{bbox}. The distortion moved the
path's \glspl{controlpt}, as can be seen when the path is put back into
\editpathmode\ in \figureref{fig:distortpathf}. If this path is
returned to distortion mode, the distortion controls will be reinitialised to
the corners of the new bounding box. You can't invert the distortion
to restore the path to its original shape.

\FloatSubFigs{fig:distortpath}
{
  {fig:distortpatha}{\includeimg{distortpatha}}{},
  {fig:distortpathb}{\includeimg{distortpathb}}{},
  {fig:distortpathc}{\includeimg{distortpathc}}{},
  {fig:distortpathd}{\includeimg{distortpathd}}{},
  {fig:distortpathe}{\includeimg{distortpathe}}{},
  {fig:distortpathf}{\includeimg{distortpathf}}{}
}
[Distorting a Path]
{Distorting a Path: 
\subfigref{fig:distortpatha} original path in edit mode;
\subfigref{fig:distortpathb} original path selected;
\subfigref{fig:distortpathc} distortion mode on;
\subfigref{fig:distortpathd} distortion in progress;
\subfigref{fig:distortpathe} distorted shape (distortion mode off);
\subfigref{fig:distortpathf} distorted shape in edit mode.}

\section{Separating Compound Shapes}\label{sec:separate}

\menudef{menu.transform.separate}

TODO


\section{Converting a Path or Text-Path into a Pattern}\label{sec:patterns}

\menudef{menu.transform.pattern}

\menudef{menu.transform.pattern.set}

\menudef{menu.transform.pattern.edit}

\menudef{menu.transform.pattern.remove}

TODO


\section{Converting to a Path}\label{sec:converttopath}

\menudef{menu.transform.convert}

\menudef{menu.transform.convert_to_full}

TODO


\subsection{Converting a Text Area, Text-Path or Pattern to a Path}\label{sec:texttopath}

TODO


\subsection{Converting an Outline to a Path}\label{sec:outlinetopath}

TODO


\chapter{Path and Text Styles}\label{sec:styles}

TODO


\section{Line Paint}\label{sec:linepaint}

\inlineglsdef{index.path-attribute.line-paint}

TODO


\section{Fill Paint}\label{sec:fillpaint}

\inlineglsdef{index.path-attribute.fill-paint}

TODO


\section{Line Style}\label{sec:pathstyle}

TODO


\subsection{Line Thickness (or Pen Width)}\label{sec:penwidth}

\inlineglsdef{index.path-attribute.line-width}

TODO


\subsection{Dash Pattern}\label{sec:dashpattern}

\inlineglsdef{index.path-attribute.dash-pattern}

TODO


\subsection{Cap Style}\label{sec:capstyle}

\inlineglsdef{index.path-attribute.cap-style}

TODO


\subsection{Join Style}\label{sec:joinstyle}

\inlineglsdef{index.path-attribute.join-style}

TODO


\subsection{Markers}\label{sec:markers}

\inlineglsdef{index.marker}
TODO


\subsubsection{Enabling or Disabling a Marker}\label{sec:enablingmarkers}

TODO


\subsubsection{Marker Types}\label{sec:markertypes}

\inlineglsdef{index.marker.type}

TODO


\subsubsection{Marker Size}\label{sec:markersize}

\inlineglsdef{index.marker.size}

TODO


\subsubsection{Repeating Markers}\label{sec:repeatingmarkers}

\inlineglsdef{index.marker.repeating}

TODO


\subsubsection{Reversing Markers}\label{sec:reversingmarkers}

\inlineglsdef{index.marker.reversing}

TODO


\subsubsection{Composite Markers}\label{sec:compositemarkers}

\inlineglsdef{index.marker.composite}

TODO


\subsubsection{Marker Orientation}\label{sec:markerorientation}

\inlineglsdef{index.marker.orientation}

TODO


\subsubsection{Marker Offset}\label{sec:markeroffset}

\inlineglsdef{index.marker.offset}

TODO


\subsubsection{Repeat Gap}\label{sec:repeatgap}

\inlineglsdef{index.marker.repeat-gap}

TODO


\subsubsection{Marker \Glsentrytext{manual.colour}}\label{sec:markerpaint}

\inlineglsdef{index.marker.colour}

TODO


\subsection{Winding Rule}\label{sec:winding}

\inlineglsdef{index.path-attribute.winding-rule}

TODO


\section{Text Paint}\label{sec:textpaint}

\inlineglsdef{index.text-attribute.paint}

TODO


\section{Text Style}\label{sec:textstyle}

TODO


\subsection{Font Family}\label{sec:fontfamily}

\inlineglsdef{index.text-attribute.font-family}

TODO


\subsection{Font Size}\label{sec:fontsize}

\inlineglsdef{index.text-attribute.font-size}

TODO


\subsection{Font Series}\label{sec:fontseries}

\inlineglsdef{index.text-attribute.font-series}

TODO


\subsection{Font Shape}\label{sec:fontshape}

\inlineglsdef{index.text-attribute.font-shape}

TODO


\subsection{Anchor}\label{sec:fontanchor}

\inlineglsdef{index.text-attribute.anchor}

TODO


\chapter{\TeX/\LaTeX}\label{sec:tex}

TODO


\section{Adding Commands to the Preamble}\label{sec:preamble}

TODO


\section{Computing the Parameters for \glsfmttext{parshape}}\label{sec:parshape}

TODO


\section{Computing the Parameters for \glsfmttext{shapepar} or
\glsfmttext{Shapepar}}
\label{sec:shapepar}

TODO


\section{Creating Frames for Use with the \stytext{flowfram} Package}
\label{sec:flowframe}

TODO


\subsection{The \stytext{flowfram} Package: A Brief Summary}
\label{sec:flowframesummary}

TODO


\subsection{Defining the Typeblock}\label{sec:typeblock}

TODO


\subsection{Defining a Frame}\label{sec:framedef}

TODO


\subsection{The Frame Shape}\label{sec:frameshape}

TODO


\subsection{Scale Object to Fit Typeblock}\label{sec:scaletotypeblock}

TODO


\subsection{Only Displaying Objects Defined on a Given Page}\label{sec:displaypage}

TODO


\chapter{Step-by-Step Examples}\label{sec:tutorials}

TODO


\section{A House}\label{sec:houseexample}

TODO


\section{Lettuce on Toast}\label{sec:toastexample}

TODO


\section{Cheese and Lettuce on Toast}\label{sec:cheeseexample}

TODO


\section{An Artificial Neuron}\label{sec:neuronexample}

TODO


\section{Bus}\label{sec:busexample}

TODO


\section{A Poster}\label{sec:postertutorial}

TODO


\section{A Newspaper}\label{sec:newstutorial}

TODO


\section{A Brochure}\label{sec:brochure}

TODO


\section{A House With No Mouse}\label{sec:accesstutorial}

TODO


\section{A Lute Rose}\label{sec:rosetutorial}

TODO


\chapter{JDR/AJR File Formats}\label{sec:jdrajrformat}

TODO


\chapter{Help Windows}
\label{sec:helpwindows}

\menudef{menu.help.about}

The \menu{help.about} menu item shows the \inlineglsdef{about.title}
dialog with version details.

\menudef{menu.help.license}

The \menu{help.license} menu item shows the \inlineglsdef{license.title}
dialog (see \sectionref{sec:licence}).

% Common file provided with TeXJavaHelpLib
% https://github.com/nlct/texjavahelp

\menudef{menu.help.manual}

The application's manual is available as either a \gls{pdf}
document, which can be viewed outside of the application, or as a
set of \gls{html} files which can be viewed within the application
via the \menu{help.manual} menu item. This will open the primary
help window (\sectionref{sec:primaryhelp}), but some dialog boxes
may also have a \inlineglsdef{button.help} button that will open a secondary help
dialog (\sectionref{sec:secondaryhelp}).

Both the primary help window and the secondary help dialog windows
have a panel that shows a page of the manual (a
\inlineglsdef{index.help-page}).  Note that \qt{page} in this
context refers to the \gls{html} file displayed in the help window, which
typically contains a section, and doesn't relate to the page numbers
in the \gls{pdf}. The \gls{html} index page is obtained from the
same source code as the \gls{pdf} index page, but the locations are
converted from a \gls{pdf} page number to the \gls{html} page title
(preceded by the marker \gls{symbol.location_prefix}).

Although the \dgls{help-page} is not editable, for some versions of
Java, the caret is visible when the page has the focus, and
the caret can be moved around using the arrow keys on your keyboard.

\menudef{index.menu.helppage}

The \gls{index.menu.helppage} (see \figureref{fig:helppagepopup})
can be activated on the current \dgls{help-page} for both the
primary and secondary help windows. The mouse press to show a popup
menu is typically the right mouse button, but this may not be the
case for all operating systems.  The popup menu can also be
activated using the context menu \keys{\keyref{contextmenu}} key if the
\dgls{help-page} has the focus.  The menu has the following items.

\FloatSubFigs
{fig:helppagepopup}
 {
   {fig:helpframepopup}
   {\includeimg
     [alt=
      {Primary help window popup menu}
     ]{helppagepopup}%
   }
   {},
   {fig:helpdialogpopup}
   {\includeimg
     [alt=
      {Secondary help multi-page dialog popup menu}
     ]{helpdialogpopup}%
   }
   {},
   {fig:helpdialogsinglepopup}
   {\includeimg
     [alt=
      {Secondary help single-page dialog popup menu}
     ]{helpdialogsinglepopup}%
   }
   {}
}
[Help Page Popup Menus]
{Help Page Popup Menus: 
 \subfigref{fig:helpframepopup} Primary Help Window;
 \subfigref{fig:helpdialogpopup} Secondary Help Dialog
 with Multiple Topic Pages;
 \subfigref{fig:helpdialogsinglepopup} Secondary Help Dialog
 with Single Topic Pages}

\menudef{menu.helppage.view_image}

If the popup menu is activated over an image, the \menu{helppage.view_image}
item will open the \dialog{imageviewer} window (see
\sectionref{sec:helpimageviewer}) which can be used to enlarge the
image. This item will be disabled if the popup menu wasn't activated
over an image.

Where the popup menu was activated using the context menu
\keys{\keyref{contextmenu}} key, 
the position of the caret will determine
whether or not to enable this menu item.

\menudef{menu.helppage.home}

If the popup menu is activated on the primary help window
(\figureref{fig:helpframepopup}), this will
behave as the \menu{helpframe.navigation.home} menu item (which
switches the current page to the first page of the document).
This menu item is not available on secondary help windows.

\menudef{menu.helppage.reset}

If the popup menu is activated on the secondary help dialog
(\figuresref{fig:helpdialogpopup,fig:helpdialogsinglepopup}), this
will behave as the \menu{helpdialog.navigation.reset} menu item
(which switches the current page back to the relevant page or the
first in the applicable section of the dialog topic).
It will be disabled if the current page is the reset target page.
This menu item is not available on the primary help window.

\menudef{menu.helppage.up}

If the popup menu is activated on the primary help window or on the
secondary help window that has multiple pages
(\figuresref{fig:helpframepopup,fig:helpdialogpopup}), then this will behave
as the primary \menu{helpframe.navigation.up} or
secondary \menu{helpdialog.navigation.up} menu items. (That is, it
will move up a hierarchical level, if available.)
This menu item will be disabled if there is no parent page (or, for
secondary windows, no parent page within the topic set).

\menudef{menu.helppage.previous}

If the popup menu is activated on the primary help window or on the
secondary help window that has multiple pages
(\figuresref{fig:helpframepopup,fig:helpdialogpopup}), then this
will behave as the primary \menu{helpframe.navigation.previous} or
secondary \menu{helpdialog.navigation.previous} menu items.  (That
is, it will move to the previous page, if available.) This menu item
will be disabled if there is no previous page (or, for secondary
windows, no previous page within the topic set).

\menudef{menu.helppage.next}

If the popup menu is activated on the primary help window or on the
secondary help window that has multiple pages
(\figuresref{fig:helpframepopup,fig:helpdialogpopup}), then this will behave
as the primary \menu{helpframe.navigation.next} or
secondary \menu{helpdialog.navigation.next} menu items.
(That is, it will move to the next page, if available.)
This menu item will be disabled if there is no next page (or, for secondary
windows, no next page within the topic set).

\menudef{menu.helppage.historyback}

This will behave as the primary
\menu{helpframe.navigation.historyback} or secondary
\menu{helpdialog.navigation.historyback} menu items.
(That is, it will move back a page in the history list, if available.)
This menu item is in all the help page popup menus but will be
disabled if there is no page to go back to.

For the secondary help windows, it's possible to follow a link in
the current page to a page outside the topic set. The menu item can
take you back to the previously visited page viewed in that
secondary dialog window.

\menudef{menu.helppage.historyforward}

This will behave as the primary
\menu{helpframe.navigation.historyforward} or secondary
\menu{helpdialog.navigation.historyforward} menu items.
(That is, it will move forward a page in the history list, if available.)
This menu item is in all the help page popup menus but will be
disabled if there is no page to go forward to.


\section{The Primary Help Window}
\label{sec:primaryhelp}

The primary help window is the main help frame accessed via
\menu{help.manual}, which has a panel that shows a page of the
manual (a \dgls{help-page}). Links in the page and the \gls{gui}
navigation elements provide a way to switch to a different page.

There is a menu bar with items for navigation actions or adjusting
\gls{gui} settings. Some menu items are replicated as buttons in the
toolbar, which is split into different regions: navigation, lookup,
settings, and history. The forward, up and next navigation actions
can also be implemented by buttons in the lower navigation panel at
the bottom of the window.

\menudef*{menu.helpframe.navigation}

The \menu{helpframe.navigation} menu provides a way to move around
the document.
\Figureref{fig:navbuttons} shows the corresponding four navigation
buttons in the toolbar: \btn{helpframe.navigation.home} (go to the
start of the manual), \btn{helpframe.navigation.previous} (go to the
previous section), \btn{helpframe.navigation.up} (go to parent
section), and \btn{helpframe.navigation.next} (go to the next
section).

\FloatFig
{fig:navbuttons}
{\includeimg
 [alt=
   {
     [\entrytooltip{menu.helpframe.navigation.home} Button]
     [\entrytooltip{menu.helpframe.navigation.previous} Button]
     [\entrytooltip{menu.helpframe.navigation.up} Button]
     [\entrytooltip{menu.helpframe.navigation.next} Button]
   }
 ]{navbuttons}%
}
[Primary Help Window Navigation Buttons]
{Primary Help Window Navigation Buttons (Home, Previous, Up, Next)}

\menudef{menu.helpframe.navigation.home}

The \menu{helpframe.navigation.home} item, which is also available
as a button on the toolbar, will replace the current view with the
first page of the document.

\menudef{menu.helpframe.navigation.up}

The \menu{helpframe.navigation.up} item, which is also available
as a button on the toolbar, will replace the current view with the
parent page of the current hierarchical level. The item and button
will be disabled if there is no parent page (that is, if the current
page is the document's home page). The parent page may
also be the previous page if the current page is the first in its
current hierarchical level.

\menudef{menu.helpframe.navigation.previous}

The \menu{helpframe.navigation.previous} item, which is also available as
a button on the toolbar, will replace the current view with the
previous page. The item and button will be disabled if there is no
previous page. (That is, if the current page is the first
page of the document.)

\menudef{menu.helpframe.navigation.next}

The \menu{helpframe.navigation.next} item, which is also available as
a button on the toolbar, will replace the current view with the
next page. The item and button will be disabled if there is no
next page. (That is, if the current page is the last
page of the document.)

\Figureref{fig:search+index} shows the search and index buttons,
which may be used to lookup relevant pages.

\FloatFig
{fig:search+index}
{\includeimg
 [alt=
   {
     [\entrytooltip{menu.helpframe.navigation.search} Button]
     [\entrytooltip{menu.helpframe.navigation.index} Button]
   }
 ]{search+index}%
}
{Search and Index Buttons}

\menudef{menu.helpframe.navigation.search}

The \menu{helpframe.navigation.search} item, which is also available
as a button on the toolbar, will open the
\dialog{help_page_search} window (see
\sectionref{sec:helpsearch}), from which you can search the document
for a keyword.

\menudef{menu.helpframe.navigation.index}

The \menu{helpframe.navigation.index} item, which is also available
as a button on the toolbar, will open the index page in a separate
window (see \figureref{fig:indexframe}). You can also open the same
page in the help window at the end of the document. The separate index window
provides a way of navigating the document without having to keep
returning to the index page. Additionally, the index window has a
split page with links on the left to scroll the page to a letter
group.

If an indexed item is shown as a hyperlink, then that link will go
to the principle definition of that item. The indexed item may also
be followed by a list of pertinent locations that are preceded by
the symbol \gls{symbol.location_prefix}.

\FloatFig
{fig:indexframe}
{\includeimg
 [alt=
   {image of index window showing part of the document index}
 ]{indexframe}%
}
{Index Window}

\Figureref{fig:historybuttons} shows the history buttons.
Note that the forward button is greyed (disabled) because the
currently viewed page is at the end of the history list, so it's not
possible to go forward.

\FloatFig
 {fig:historybuttons}
 {%
   \includeimg
    [alt=
     {[\entrytooltip{menu.helpframe.navigation.history} Button]
      [\entrytooltip{menu.helpframe.navigation.historyback} Button]
      [\entrytooltip{menu.helpframe.navigation.historyforward} Button]
     }
    ]
    {historybuttons-annote}%
 }
 {History Buttons}

\menudef{menu.helpframe.navigation.history}

The \menu{helpframe.navigation.history} menu item, which is also
available as a button on the toolbar, opens the
\dialog{help.navigation.history} window,
(see \figureref{fig:historywindow}).

The current page has the title shown in bold and is preceded by
the symbol \gls{symbol.help.navigation.history.pointer}.
Select the required page and click on the
\gls{help.navigation.history.go} button.

\FloatFig
 {fig:historywindow}
 {\includeimg
   [alt={image of the help page history window}]
   {historyframe}%
 }
 {The Page History Window}

\menudef{menu.helpframe.navigation.historyback}

The \menu{helpframe.navigation.historyback} menu item, which is
also available as a button on the toolbar, will replace the current
view with the previously viewed page from this history list. The
item and button will be disabled if there is no previously viewed
page. 

\menudef{menu.helpframe.navigation.historyforward}

The \menu{helpframe.navigation.historyforward} menu item, which is
also available as a button on the toolbar, will replace the current view with the
next page in the history list. The item and button will be disabled if the
currently viewed page is at the end of the history list. 

\menudef*{menu.helpframe.settings}

The \menu{helpframe.settings} menu can be used to change the
graphical interface settings. These settings affect the primary and
secondary help windows, as well as some other related windows.
Note that this is separate from the main application settings.

\menudef{menu.helpframe.settings.decrease}

The \menu{helpframe.settings.decrease} item decreases the font
size by 1.

\menudef{menu.helpframe.settings.increase}

The \menu{helpframe.settings.increase} item increases the font
size by 1.

\menudef{menu.helpframe.settings.font}

The \menu{helpframe.settings.font} item opens the
\dialog{help_font_settings} dialog (see
\sectionref{sec:helpfontdialog}).

\menudef{menu.helpframe.settings.nav}

The \menu{helpframe.settings.nav} item opens the
\inlineglsdef{help_settings_nav.title} dialog. This governs the
lower navigation bar (see \figureref{fig:helplowernavbar}) along the
bottom of the primary help window, which has smaller previous, up and next
buttons.  These buttons by default have the corresponding page
titles next to them, but they will be truncated if they exceed the
limit. This limit can be changed with the
\widget{help_settings_nav.label_limit} widget. Alternatively, you
can hide the text by deselecting the
\widget{help_settings_nav.show_label} checkbox.

\FloatFig
{fig:helplowernavbar}
{\includeimg
 [alt=
   {Help page lower navigation bar}
 ]{helplowernavbar}%
}
{Help Page Lower Navigation Bar}

\section{Secondary Help Window}
\label{sec:secondaryhelp}

The secondary help windows are more minimalist and will only show
the relevant \dgls{help-page} or set of pages that are applicable to
the context that was used to open the secondary help window. If only
one page is applicable, there won't be a navigation tree, otherwise
the navigation tree will only show the applicable pages.

The search, history and index windows are unavailable, but it is
possible to move back and forward in the history list for the
current secondary help window. The topic page will be added to the
primary help window history but otherwise the page history lists
aren't shared between the help windows.

The secondary help windows are designed for use with modal dialogs
(that is, a window that blocks the main application window)
to provide help for the particular dialog. The primary help window
can't be accessed while a modal dialog is open so it will
automatically be closed when a secondary help window is opened.
You can re-open the primary help window once you have closed the
modal dialog.

\menudef*{menu.helpdialog.navigation}

The \menu{helpdialog.navigation} menu provides a way to move around
the topic pages.

\menudef{menu.helpdialog.navigation.reset}

The \menu{helpdialog.navigation.reset} item switches the current
page to the first page of the context topic. This menu item
will be disabled if the current page is the reset target page.

\menudef{menu.helpdialog.navigation.historyback}

The \menu{helpdialog.navigation.historyback} goes back to the
previously visited page. Note that the history is specific to the
current secondary help dialog instance and does not include the history
from the primary help window. This menu item will be disabled if
there is no page in the history list to go back to.

\menudef{menu.helpdialog.navigation.historyforward}

The \menu{helpdialog.navigation.historyforward} moves forward in the
history list, if applicable. This menu item will be disabled if
there is no page in the history list to go forward to.

The \btn{menu.helpdialog.navigation.previous}, 
\btn{menu.helpdialog.navigation.up} and
\btn{menu.helpdialog.navigation.next} buttons
(\figureref{fig:secondaryNavbuttons}) are only available if the
topic context contains multiple pages.

\FloatFig
{fig:secondaryNavbuttons}
{\includeimg
 [alt=
   {
     [\entrytooltip{menu.helpdialog.navigation.reset} Button]
     [\entrytooltip{menu.helpdialog.navigation.previous} Button]
     [\entrytooltip{menu.helpdialog.navigation.up} Button]
     [\entrytooltip{menu.helpdialog.navigation.next} Button]
   }
 ]{secondaryNavButtons}%
}
[Secondary Help Window Navigation Buttons for Multi-Page Topics]
{Secondary Help Window Navigation Buttons for Multi-Page Topics
  (Reset, Previous, Up, Next)}

\menudef{menu.helpdialog.navigation.previous}

The \menu{helpdialog.navigation.previous} menu item is only
available if there are multiple pages for the topic context and will
switch the current page with the previous page in the topic set.
This menu item will be disabled if the previous page is not within the
topic set.

\menudef{menu.helpdialog.navigation.up}

The \menu{helpdialog.navigation.up} menu item is only
available if there are multiple pages for the topic context and will
switch the current page with the parent page if it's within the topic set.
This menu item will be disabled if there is no parent page or if the
parent page is not in the topic set.

\menudef{menu.helpdialog.navigation.next}

The \menu{helpdialog.navigation.next} menu item is only
available if there are multiple pages for the topic context and will
switch the current page with the next page in the topic set.
This menu item will be disabled if the next page is not within the
topic set.

\section{Help Font Dialog}
\label{sec:helpfontdialog}

The \menu{helpframe.settings.font} item opens the
\inlineglsdef{help_font_settings.title} dialog (see
\figureref{fig:helpfontdialog}). Use the
\widget{help_font_settings.family} selector for the main body font
family and the \widget{help_font_settings.size} selector for the
main body font size. Icon characters, such as
\gls{symbol.help.navigation.history.pointer}, may not be available
for your preferred font family, so you can specify an alternative
with the \widget{help_font_settings.icon_font_family} selector. This
will only list fonts that support some commonly used icon
characters.

\FloatFig
{fig:helpfontdialog}
{\includeimg
 [alt=
   {Help page font dialog}
 ]{helpfontdialog}%
}
{Help Page Font Dialog}

Use the \widget{help_font_settings.keystroke_font_family} selector to
choose the font to show keystrokes (such as
\keys{\keyref{shift}}) and the
\widget{help_font_settings.mono_font_family} selector to choose the
font to display code fragments (such as \verb|% \ { } #|).

The document hyperlink style can also be changed with the 
\widget{help_font_settings.hyperlinks} 
\btn{help_font_settings.choose_colour} and
\btn{help_font_settings.underline} widgets.

The styles are applied to the primary help window, all secondary
help windows and related windows, such as the \dialog{help.navigation.history} 
or index windows.

\section{Searching the Documentation}
\label{sec:helpsearch}

The \dialog{help_page_search} window (which can be opened from the
primary help window with \menu{helpframe.navigation.search}) provides 
a way to search the documentation. Enter the desired search term or terms into the
\widget{help_page_search.keywords} box. Select the \widget{help_page_search.case}
checkbox for a case-sensitive search and the \widget{help_page_search.exact}
checkbox for an exact match. If the \widget{help_page_search.exact}
checkbox is not selected, the search will be slower and will match
any instances of the keyword appearing as a sub-string of other
words as well as whole-word matches.

\begin{information}
The search is performed by looking up a pre-compiled set of words with
associated locations that was created when the documentation was
built. It's not possible to search for exact phrases. The results are
ordered according to the number of matches found in each block or
paragraph.
\end{information}

Click on the \widget{menu.help_page_search.search_menu.search} button to the
right of the \widget{help_page_search.keywords} box or use the \menu{help_page_search.search_menu.search}
menu item to start searching. Note that small common words, such as \qt{and}, will be
ignored.

\FloatFig
{fig:searchframe}
{\includeimg
 [alt=
   {image of search window showing search term highlighted in a paragraph}
 ]{searchframe}%
}
{Search Window}

If any matches are found, the title of the relevant
page is shown as a hyperlink, which links to the start of the page.
The title is followed by a block of text where the search term (or
terms) was found (which will be highlighted, as shown in 
\figureref{fig:searchframe}). Clicking on the block of text
should scroll to a nearby location in the relevant page.

\menudef*{menu.help_page_search.search_menu}
The \menu{help_page_search.search_menu} menu has the following
menu items.

\menudef{menu.help_page_search.search_menu.search}
The \menu{help_page_search.search_menu.search} menu item starts
searching for the given keywords. An error box will be displayed if
no keywords have been supplied.

\menudef{menu.help_page_search.search_menu.previous}
The \menu{help_page_search.search_menu.previous} menu item
will scroll the result list to the previous result.

\menudef{menu.help_page_search.search_menu.next}
The \menu{help_page_search.search_menu.next} menu item will 
scroll to the result list to the next result.

\menudef{menu.help_page_search.search_menu.reset}
The \menu{help_page_search.search_menu.reset} menu item will
clear the current result list and the keyword search box so that you
can perform a new search.

\menudef{menu.help_page_search.search_menu.stop}
The \menu{help_page_search.search_menu.stop} menu item can be
used to stop a search if it's taking too long to complete.

\section{Image Viewer}
\label{sec:helpimageviewer}

The \menu{helppage.view_image} item in the \gls{index.menu.helppage}
for both the primary and secondary help windows will be enabled if
the \gls{index.menu.helppage} is activated over an image. The
\menu{helppage.view_image} item will open the image in the
\inlineglsdef{imageviewer.title} window.  If the image had alt text
specified, this will be displayed in the area above the image.

Within the \dialog{imageviewer} window, the image can be enlarged
using the \widget{imageviewer.magnify} spinner. The up and down
spinner controls go in steps of 25 (as opposed to the
\btn{menu.imageviewer.increase} and \btn{imageviewer.decrease}
action, which have an increment of 5). Alternatively, press the
shift key \keys{\keyref{shift}} and drag the mouse to select an area to
zoom in on. Be sure to keep the shift key down when you release the
mouse. If you change your mind, release shift before releasing the
mouse button. If the shift key isn't pressed when you initiate the
drag, dragging will scroll the image instead. Double-clicking the
mouse on the image will go back to the previous magnification.

\menudef{index.menu.imageviewer}

The \gls{index.menu.imageviewer} is a popup menu that can
be activated anywhere over the image in the \dialog{imageviewer}
window. The following menu items are available.

\menudef{menu.imageviewer.fit_to_width}

The \menu{imageviewer.fit_to_width} item will scale the image so
that it fits the window width. This action has a corresponding
button on the toolbar.

\menudef{menu.imageviewer.fit_to_height}

The \menu{imageviewer.fit_to_height} item will scale the image so
that it fits the window height. This action has a corresponding
button on the toolbar.

\menudef{menu.imageviewer.fit_to_page}

The \menu{imageviewer.fit_to_page} item will scale the image so
that it fits within the window area. This action has a corresponding
button on the toolbar.

\menudef{menu.imageviewer.increase}

The \menu{imageviewer.increase} item will increase the current
magnification. This action has a corresponding
button on the toolbar.

\menudef{menu.imageviewer.decrease}

The \menu{imageviewer.decrease} item will decrease the current
magnification. This action has a corresponding
button on the toolbar.

\menudef{menu.imageviewer.zoom_1}

The \menu{imageviewer.zoom_1} item will set the magnification factor to
100\%. This action has a corresponding
button on the toolbar.

\menudef{menu.imageviewer.zoom_2}

The \menu{imageviewer.zoom_2} item will set the magnification factor to
200\%.

\menudef{menu.imageviewer.zoom_5}

The \menu{imageviewer.zoom_5} item will set the magnification factor to
500\%.



\chapter{Multilingual Support}\label{sec:multilingualsupport}

TODO


\chapter{Source Code}\label{sec:sourcecode}

TODO


\section{Java Source}\label{sec:javasource}

TODO


\section{Document Source}\label{sec:docsource}

TODO


\chapter{Troubleshooting}\label{sec:troubleshooting}

TODO


\section{Known Bugs}\label{sec:knownbugs}

TODO

\chapter{\MFUsentencecase{\glsentrytext{index.licence}}}
\label{sec:licence}

\glsadd{index.licence}%
\FlowframTk\ is licensed under the terms of the 
\href{https://www.gnu.org/licenses/gpl-3.0.html}{GNU General
Public License version 3 (GPLv3)}.
\FlowframTk\ depends on the following third party library whose
jar file is in the \filefmt{lib} directory:
\begin{itemize}
   \item TeX Java Help \filefmt{texjavahelplib.jar}
   (GPL, \url{https://github.com/nlct/texjavahelp}).
\end{itemize}


\listentrydescendents
 [title={Summary of \apptext{flowframtk} Switches}]
 {app.flowframtk}

\printmain
\printindex 

\end{document} 

