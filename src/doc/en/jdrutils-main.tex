
\usepackage{fontspec}
\setmainfont{Linux Libertine O}
\usepackage{verbatim}
\usepackage
 [
    novref,
%   debug=showwrgloss
 ]{texjavahelp}

\hypersetup{colorlinks,linkcolor=blue}
\DeclareGraphicsExtensions{.png,.pdf,.tex}

\title{jdrutils}
\author{Nicola L.C. Talbot\\\href{https://www.dickimaw-books.com/}{\nolinkurl{dickimaw-books.com}}}

\InputIfFileExists{version}{}{}

\GlsXtrLoadResources[src={jdrutils,shared,\langbibsrcs},
 \TeXJavaHelpSymbolResourceOptions
]

\GlsXtrLoadResources[src={jdrutils,shared,\langbibsrcs},
 \TeXJavaHelpGlsResourceOptions
]

\providecommand{\FlowframTk}{FlowframTk}

\begin{document}
\maketitle
\frontmatter
\tableofcontents
\mainmatter

\chapter{Introduction}
\label{sec:intro}

\FlowframTk\ comes with a number of command line applications for
converting \gls+{JDR} and \gls+{AJR} files to other formats. This document
describes those applications.

\begin{information}
The command line converters such as \appfmt{ajr2jdr} are due to 
all be combined into a single command line tool. 
\end{information}

\chapter{jdrinfo}

\appdef{jdrinfo}

The \app{jdrinfo} command line application simply displays the file
format (\gls{JDR}/\gls{AJR}) and version number for the given
file.

TODO (pending update).

%\listentrydescendents
% [title={Summary of \apptext{jdrview} Switches}]
% {app.jdrview}

\printmain
\printindex 

\end{document} 

