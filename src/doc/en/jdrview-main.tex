
\usepackage{fontspec}
\setmainfont{Linux Libertine O}
\usepackage{verbatim}
\usepackage
 [
    novref,
   %debug=showwrgloss
 ]{texjavahelp}

\hypersetup{colorlinks,linkcolor=blue}
\DeclareGraphicsExtensions{.png,.pdf,.tex}


\InputIfFileExists{version}{}{}

\GlsXtrLoadResources[src={jdrview,shared,\langbibsrcs},
 \TeXJavaHelpSymbolResourceOptions
]

\GlsXtrLoadResources[src={jdrview,shared,\langbibsrcs},
 \TeXJavaHelpGlsResourceOptions
]

\providecommand{\FlowframTk}{FlowframTk}
\providecommand{\JDRView}{JDRView}

\title{User Manual for \JDRView}
\author{Nicola L.C. Talbot\\\href{https://www.dickimaw-books.com/}{\nolinkurl{dickimaw-books.com}}}

\begin{document}
\maketitle
\frontmatter
\tableofcontents
\listoffigures
\mainmatter

\chapter{Introduction}
\label{sec:intro}

\JDRView\ is a \gls+{JDR}/\gls+{AJR} image viewer. It is designed for viewing
\gls{JDR} or \gls{AJR} images that have been created by an external
application.  Its main purpose is to assist testing applications
that create \gls{JDR} or \gls{AJR} files, and so has a reload facility, but does
not have any functions that modify the image.

Unlike \FlowframTk, \JDRView\ does not have a \gls{mdi}, and so
requires new instances if additional images need to be viewed
at the same time.

If any \glspl+{object} have been assigned a description, move the
pointer over the \gls{object} and the description will appear.
If the pointer is positioned over an empty area, the image's
description will appear, if it has been given one.

\chapter{Command Line Options}
\label{sec:commandlineopts}

\JDRView\ can be invoked from a command prompt:
\appdef{jdrview}

The \meta{option-list} and \meta{filename} may be omitted.
Only one filename is permitted, and it must be either a
\gls{JDR} or \gls{AJR} file. This script uses the environment
variable \gls{JDR_JVMOPTS} to pass options to the \gls{jvm}.
For example, if you want to run \JDRView\ with a maximum size
of 128Mb for the memory allocation pool, you can set
\gls{JDR_JVMOPTS} to \code{-Xmx128m}:
\begin{verbatim}
setenv JDR_JVMOPTS -Xmx128m
\end{verbatim}
Note that \FlowframTk\ also uses this environment variable.

If you can't use the \app{jdrview} script, you can invoke
\JDRView\ from the command line using:
\begin{terminal}
java -jar jdrview.jar \meta{option-list} \meta{filename}
\end{terminal}
(You may need to include the full pathname to \file{jdrview.jar}.)

The following options are provided:

\switchdef{antialias}
Switch on the anti-aliasing (default).

\switchdef{noantialias}
Switch off the anti-aliasing.

\switchdef{cwd}
Set the current working directory to \meta{path}.

\switchdef{version}
Print the version information and exit.

\switchdef{help}
Print the syntax help and exit.

\chapter{The File Menu}\label{sec:thefilemenu}

\menudef{menu.file}

The \menu{file} menu can be used to load and reload \gls{JDR}
or \gls{AJR} images.

\menudef{menu.file.open}

A new image can be loaded using the \menu{file.open} menu item.
This will replace the display, rather than creating a new window.

\menudef{menu.file.reload}

An image can be reloaded using \menu{file.reload}. This
facility is only available if an image has already been loaded.

\menudef{menu.file.properties}

The image properties can be displayed using
\menu{file.properties}. This will display: the file name,
the file size, the file format (\gls{JDR} or \gls{AJR} and version number), the
image bounding box, the paper size (if one was specified in the
file), the image's description and the last time the image was
modified.

\menudef{menu.file.print}

You can print the current image using the \menu{file.print} menu
item which will open the printer dialog box. This behaves as for
\FlowframTk.

\menudef{menu.file.quit}

To quit \JDRView\ use \menu{file.quit}.


\chapter{Settings}
\label{sec:settings}

\menudef{menu.settings.antialias}

The \menu{settings.antialias} menu item can be used to switch
the anti-aliasing on or off.

\section{The \glsentrytext{menu.settings.zoom} Sub-Menu}
\label{sec:zoommenu}

\menudef{menu.settings.zoom}

The \menu{settings.zoom} sub-menu allows you to change
the magnification. You can choose one of the predefined
settings%
\glsaddeach{%
menu.settings.zoom.width,%
menu.settings.zoom.height,%
menu.settings.zoom.page,%
menu.settings.zoom.25,%
menu.settings.zoom.50,%
menu.settings.zoom.75,%
menu.settings.zoom.100,%
menu.settings.zoom.200,%
menu.settings.zoom.400,%
menu.settings.zoom.800%
} 
or you can specify an arbitrary setting.

\menudef{menu.settings.zoom.user}

The \menu{settings.zoom.user} menu item will open the
\inlineglsdef{zoom.title} dialog box (see \figureref{fig:zoom}) in
which to specify an arbitrary magnification.  The magnification
value may be entered as either a percentage
(\figureref{fig:zoompercent}) or a decimal value
(\figureref{fig:zoomdecimal}). For example, either 150\% or 1.5 to
zoom by a factor of 1.5. 

\FloatSubFigs{fig:zoom}
 {
   {fig:zoompercent}{\includeimg{setMagnificationPercent}}{Percentage},
   {fig:zoomdecimal}{\includeimg{setMagnificationDecimal}}{Decimal}
 }
 {Setting Custom Magnification}


\chapter{Help Windows}
\label{sec:helpwindows}

\menudef{menu.help.about}

The \menu{help.about} menu item shows the
\InlineMsgDef
 [\JDRView]
 {about.title}
dialog with version details.

\menudef{menu.help.license}

The \menu{help.license} menu item shows the
\InlineMsgDef
 {license.title}
dialog (see \sectionref{sec:licence}).

\input{helpinterface}


\chapter{\MFUsentencecase{\glsentrytext{index.licence}}}
\label{sec:licence}

\glsadd{index.licence}%
\JDRView\ is licensed under the terms of the 
\href{https://www.gnu.org/licenses/gpl-3.0.html}{GNU General
Public License version 3 (GPLv3)}.
\app{jdrview} depends on the following third party library whose jar
file is in the \filefmt{lib} directory:
\begin{itemize}
   \item TeX Java Help \filefmt{texjavahelplib.jar}
   (GPLv3, \url{https://github.com/nlct/texjavahelp}).
\end{itemize}


%\listentrydescendents
% [title={Summary of \apptext{jdrview} Switches}]
% {app.jdrview}

\printmain
\printindex 

\end{document} 

