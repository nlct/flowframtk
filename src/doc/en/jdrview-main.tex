
\usepackage{fontspec}
\setmainfont{Linux Libertine O}
\usepackage{verbatim}
\usepackage
 [
    novref,
%   debug=showwrgloss
 ]{texjavahelp}

\hypersetup{colorlinks,linkcolor=blue}
\DeclareGraphicsExtensions{.png,.pdf,.tex}

\newcommand{\JDRView}{JDRView}

\title{User Manual for \JDRView}
\author{Nicola L.C. Talbot\\\href{https://www.dickimaw-books.com/}{\nolinkurl{dickimaw-books.com}}}

\InputIfFileExists{version}{}{}

\GlsXtrLoadResources[src={jdrview,shared,\langbibsrcs},
 \TeXJavaHelpSymbolResourceOptions
]

\GlsXtrLoadResources[src={jdrview,shared,\langbibsrcs},
 \TeXJavaHelpGlsResourceOptions
]

\begin{document}
\maketitle
\frontmatter
\tableofcontents
\listoffigures

\chapter{Introduction}
\label{sec:intro}

TODO

\chapter{Settings}
\label{sec:settings}

\subsection{The \glsfmttext{menu.settings.zoom} Sub-Menu}
\label{sec:zoommenu}

\menudef{menu.settings.zoom}

The \menu{settings.zoom} sub-menu allows the magnification to
be changed.

\menudef{menu.settings.zoom.user}

The \menu{settings.zoom.user} menu item opens the
\InlineMsgDef{zoom.title} dialog where you can specify a custom
magnification.

\chapter{Help Windows}
\label{sec:helpwindows}

\menudef{menu.help.about}

The \menu{help.about} menu item shows the
\InlineMsgDef
 [\JDRView]
 {about.title}
dialog with version details.

\menudef{menu.help.license}

The \menu{help.license} menu item shows the
\InlineMsgDef
 {license.title}
dialog (see \sectionref{sec:licence}).

\input{helpinterface}


\chapter{\MFUsentencecase{\glsentrytext{index.licence}}}
\label{sec:licence}

\glsadd{index.licence}%
\JDRView\ is licensed under the terms of the 
\href{https://www.gnu.org/licenses/gpl-3.0.html}{GNU General
Public License version 3 (GPLv3)}.
\app{jdrview} depends on the following third party library whose jar
file is in the \filefmt{lib} directory:
\begin{itemize}
   \item TeX Java Help \filefmt{texjavahelplib.jar}
   (GPLv3, \url{https://github.com/nlct/texjavahelp}).
\end{itemize}


%\listentrydescendents
% [title={Summary of \apptext{jdrview} Switches}]
% {app.jdrview}

\printmain
\printindex 

\end{document} 

