
\usepackage{fontspec}
\setmainfont{Linux Libertine O}
\usepackage{verbatim}
\usepackage{longtable}
\usepackage{booktabs}
\PassOptionsToPackage{prefix}{glossaries-extra}
\usepackage
 [
    novref,
    fnsymleft,
%   debug=showwrgloss
 ]{texjavahelp}

\hypersetup{colorlinks,linkcolor=blue}

\newcommand{\appname}{FlowframTk}

\title{User Manual for \appname}
\author{Nicola L.C. Talbot\\\href{https://www.dickimaw-books.com/}{\nolinkurl{dickimaw-books.com}}}

\InputIfFileExists{version}{}{}

\GlsXtrLoadResources[src={flowframtk,shared,\langbibsrcs},
 \TeXJavaHelpSymbolResourceOptions
]

\GlsXtrLoadResources[src={flowframtk,shared,\langbibsrcs},
 \TeXJavaHelpGlsResourceOptions
]

\begin{document}
\maketitle
\frontmatter
\tableofcontents
\listoffigures

\chapter*{Notes}

\appname\ is licensed under the terms of the GNU General Public License
(\url{http://gnu.org/licenses/gpl.html"}).
See \sectionref{sec:licence}.

DOCUMENTATION IS PROVIDED \qt{AS IS} AND ALL EXPRESS OR IMPLIED
CONDITIONS, REPRESENTATIONS AND WARRANTIES, INCLUDING ANY
IMPLIED WARRANTY OF MERCHANTABILITY, FITNESS FOR A PARTICULAR
PURPOSE OR NON-INFRINGEMENT, ARE DISCLAIMED, EXCEPT TO THE EXTENT THAT
SUCH DISCLAIMERS ARE HELD TO BE LEGALLY INVALID.

This document is a user manual for \appname. For information about
JDRView or jdrutils, see \filefmt{jdrview.pdf} or
\filefmt{jdrutils.pdf}, respectively.

\IfTeXParserLib
  {%
    This manual is also available as a PDF document, \filefmt{flowframtk.pdf}. 
  }
  {}

The latest version of \appname\ can be downloaded from
\url{https://ctan.org/pkg/flowframtk}.
Older versions can be downloaded from
\url{http://www.dickimaw-books.com/software/flowframtk/}.
The source code is in the GitHub repository
\url{https://github.com/nlct/flowframtk}.

Occasionally the canvas doesn't get redrawn correctly. To force a
redraw, use \accelerator{redraw}.


\mainmatter
\chapter{Introduction}
\label{sec:introduction}

\appname\ (formerly \gls{JpgfDraw})
is a \gls{vectorgraphics} application written in \gls{Java}, with a 
\gls{gui}.  The main purpose of \appname\ is to generate \LaTeX\ packages or 
classes that use the \sty{flowfram} package, and to generate
\sty{pgf} picture drawing code. As a side-effect, it can also be
used to generate \gls{parshape} and (\sty{shapepar} package) \gls{shapepar} or 
\gls{Shapepar} specifications. In order to run the application you 
must have the \gls{jre} installed.

In \appname, you can:

\begin{itemize}
  \item construct shapes using line, move and cubic Bézier segments
(\sectionref{sec:newobjects});

  \item edit paths by modifying the defining \glspl{controlpt}
  (\sectionref{sec:editpath});

  \item incorporate text (\sectionref{sec:newtext}) and 
  bitmap images (\sectionref{sec:insertbitmap})
  for annotating and background effects;

  \item combine text and a path to form a text-along-path
  effect (\sectionref{sec:textpath});

  \item apply replicas to a shape to form patterns
  (\sectionref{sec:patterns});

  \item extract the parameter's for \TeX's \gls{parshape} command and for 
  the \gls{shapepar} (or \gls{Shapepar}) command defined in the
  \sty{shapepar} package;

  \item construct frames for use with the \sty{flowfram} package
  (\sectionref{sec:flowframe}).
\end{itemize}

Pictures can be saved as (\sectionref{sec:saveimage}) or loaded
(\sectionref{sec:loadimage}) from \appname's native \gls{jdr}
(binary) or \gls{ajr} (ASCII) file formats. Additionally, images can be
exported (\sectionref{sec:exportimage}) as:

\begin{itemize}
    \item a \LaTeX\ file containing a \env{pgfpicture} environment
    for inclusion in a \LaTeX\ document;

    \item a single-paged complete \LaTeX\ document containing the image
    (either just encapsulating the image or with the page set to the
    same size as the \gls{canvas});

    \item a \LaTeX\ package or class that loads the \sty{flowfram} package;

    \item a PNG file;

    \item a PostScript file;

    \item \pgls{svg} file.
\end{itemize}

Note that the export to PS/PDF/SVG functions use external
processes, such as \app{latex} and \app{dvips}.
\strong{You can't import back from the files you can export to.}

The low-level \sty{pgf} basic layer commands are used during
exports to \LaTeX\ files. These commands aren't particularly easy to
read but are faster to compile than the higher-level commands that
require extra processing to parse the syntax and perform additional
calculations. The primary purpose of \appname\ is to provide a
graphical interface that can generate complicated code that's hard
to write manually.

\appname\ was tested with version 3.0.0 of the \sty{pgf}
package, version 1.16 of the \sty{flowfram} package and version
2.2 of the \sty{shapepar} package. Files created by \appname\
may not work with earlier versions of those packages. Note that some
DVI viewers may not understand PGF specials. I strongly recommend
that you read the user manuals for those packages.


Notation: a \gls{primaryclick} is a single click with the
\gls{primary-mouse-button}. This is typically the left button for a
right-handed mouse (or left side of a trackpad), but may be the
right button for a left-handed mouse. A \gls{menuclick} is a click
with the context-menu mouse button. This button depends on your
configuration, but is typically the right button for a right-handed
two-button mouse (and vice-versa for a left-handed two-button mouse). 
If the button isn't mentioned, a~\gls{click} can
be assumed to mean a~\gls{primaryclick}.

\chapter{Installation}
\label{install}

Ensure that you have the \gls{jre} installed.
This can be downloaded from \url{http://java.sun.com/j2se/}.
You must ensure that you use at least Java~8, as \appname\
does not work with earlier versions.

To install, download the installer
\metafilefmt{flowframtk-}{appversion}{-installer.jar}
(where \meta{appversion} is the application version) and run it. 
This can be done from a terminal or command prompt using:

\begin{terminal}
java -jar flowframtk-\meta{appversion}-installer.jar
\end{terminal}

Depending on the setup of your operating system, you may also be
able to \gls{doubleclick} on the \ext{jar} file to run it.


\chapter{Accessibility}
\label{access}

Most of \appname's mouse functions can be emulated using the
keyboard, however note that some systems do not allow applications
to move the pointer, so keyboard functions that move the pointer are
not guaranteed to work on every system. Keyboard accelerators and
their menu mnemonic equivalents are listed in \tableref{tab:access}.
These accelerators may be changed using the \dialog{accelerators}
tab in the \dialog{configui} window (see
\sectionref{sec:configureuidialog}).  \accelerator{menubarfocus}
switches the focus to the menu bar. The return \keys{\keyref{return}}
key will usually be equivalent to the \widget{okay} button
except when the focus is with a component that interprets
\keys{\keyref{return}} for some other purpose (such as
a~\gls{dropdown}), in which case you need to use \accelerator{okay}.

Within editable text fields, you can use \keys{\keyref{ctrl}+A} to
select all the text, or shift \keys{\keyref{shift}} followed by
the left \keys{\keyref{left}} or right \keys{\keyref{right}} arrow
key to select a portion of the text. If some of the text has been
selected, you can use \keys{\keyref{ctrl}+C} or
\keys{\keyref{ctrl}+X} to copy or cut the text onto the clipboard,
and you can use \keys{\keyref{ctrl}+V} to paste text from the
clipboard into the text field.

\begin{warning}
Note that a few of the accelerators previously used by \gls{JpgfDraw} have
been changed in \appname, as they caused a conflict. For example,
\keys{\keyref{escape}} cancels a displayed menu, so if you use it to
dismiss a menu while you had a \gls{path} under construction, the
entire path could be discarded, which is undesirable. Therefore the
abandon path accelerator is now \accelerator{tools.abandon}. The
\accelerator{help.manual} key is now only for the main manual help
button. The help buttons in dialog windows and the preamble editor
are activated with \accelerator{help} and the help button in the
status bar is activated with \accelerator{contexthelp} (otherwise
the main help button, preamble editor help button and status bar
help button would conflict).
\end{warning}

\seealsorefs{sec:configureuidialog,sec:selectobjects,sec:newobjects,sec:accesstutorial}

\begin{longtable}{lp{0.5\textwidth}p{0.2\textwidth}}
\caption{Keyboard Accelerators and Menu Mnemonics\label{tab:access}}\\
\bfseries Accelerator & \bfseries Function & \bfseries Mnemonic \\
\endfirsthead
\caption*{Keyboard Accelerators and Menu Mnemonics}\\
\bfseries Accelerator & \bfseries Function & \bfseries Mnemonic \\
\endhead
\multicolumn{3}{l}{\fnsymtext{2}{Functions that move the pointer.}}\\\endfoot
\keys{\keyref{return}} &
Finish current \gls{path}/\gls{textarea} & 
\mnemonictrail{tools.finish}\tabularnewline
& 
\emph{or} select \widget{okay} button in dialog boxes &
\mnemonic{okay}\tabularnewline
%
& 
\emph{or} select \widget{close} button in dialog boxes &
\mnemonic{close}\tabularnewline
%
\accelerator{okay} &
Select \widget{okay} button in dialog boxes &
\mnemonic{okay}\tabularnewline
%
\accelerator{tools.abandon} &
Abandon current \gls{path} &
\mnemonictrail{tools.abandon}\tabularnewline
%
\keys{\keyref{escape}} &
Select \widget{cancel} button in dialog boxes &
\mnemonic{cancel}\tabularnewline
%
 & 
\emph{or} close displayed menu &
\tabularnewline
%
\accelerator{editpath.delete_control} & 
Delete selected \gls{controlpt} &
\accelerator{contextmenu} \mnemonic{editpath.delete_control}
\tabularnewline
%
\accelerator{delete_last} &
Delete last segment (path construction mode) &
\tabularnewline
%
\keys{\keyref{insert}} &
Add \gls{controlpt} &
\accelerator{contextmenu} \mnemonic{editpath.add_control}
\tabularnewline
&
\emph{or} display symbol dialog box &
\accelerator{contextmenu} \mnemonic{textarea.insert_symbol}
\tabularnewline
%
\accelerator{focusnext} &
Move focus to next focusable component &
\tabularnewline
%
\accelerator{focusselect} &
Select component with current focus &
\tabularnewline
\midrule
%
\keys{\keyref{pageup}} &
Scroll up by one screen full &
\tabularnewline
%
\keys{\keyref{pagedown}} &
Scroll down by one screen full &
\tabularnewline
%
\keys{\keyref{ctrl}+\keyref{pagedown}} &
If in a tabbed pane: &
\tabularnewline
&
\leftquadpar{Move to the next tab} &
\tabularnewline
&
Otherwise: &
\tabularnewline
&
\leftquadpar{Scroll right by one screen full} &
\tabularnewline
%
\keys{\keyref{ctrl}+\keyref{pageup}} &
If in a tabbed pane: &
\tabularnewline
&
\leftquadpar{Move to the previous tab} &
\tabularnewline
 &
 Otherwise: &
\tabularnewline
&
\leftquadpar{Scroll left by one screen full} &
\tabularnewline
%
Arrow Keys &
If \gls{primary-mouse-button} is pressed: &
\tabularnewline
 &
 \leftquadpar{\fnsym{2}Move mouse by one pixel in given direction} &
\tabularnewline
 &
Otherwise: &
\tabularnewline
 &
\leftquadpar{Scroll by one tick mark in given direction} &
\tabularnewline
%
\keys{\keyref{home}} &
Scroll to the top of the \gls{canvas} &
\tabularnewline
%
\keys{\keyref{end}} &
Scroll to the bottom of the \gls{canvas} &
\tabularnewline
%
\keys{\keyref{ctrl}+\keyref{home}} &
Scroll leftmost &
\tabularnewline
%
\keys{\keyref{ctrl}+\keyref{end}} &
Scroll rightmost &
\tabularnewline
\midrule
%
\accelerator{help.manual} &
Display Handbook &
\mnemonictrail{help.manual}
\tabularnewline
\accelerator{settings.grid.show} &
Show/hide \gls{grid} &
\mnemonictrail{settings.grid.show}
\tabularnewline
\accelerator{contextmenu} &
Show context \gls{popupmenu} (if available) &
\tabularnewline
%
\accelerator{constructclick} &
Emulate \gls{primaryclick} in \gls{construction} &
\tabularnewline
%
\accelerator{navigate.goto} &
 \fnsym{2}Go to coordinate &
\mnemonictrail{navigate.goto}
\tabularnewline
%
\keys{\actualkey{F6}} &
Select mode: &
\tabularnewline
 &
 \leftquadpar{Deselect the \glsdisp{backobject}{backmost} selected 
  \gls{object}, and select next object in the \gls{stack}} &
\mnemonictrail{navigate.skip}
\tabularnewline
&
Edit mode: &
\tabularnewline
&
\leftquadpar{Select next \gls{controlpt}} &
\accelerator{contextmenu} \mnemonic{editpath.next_control}
\tabularnewline
%
\keys{\actualkey{F7}} &
Select mode: &
\tabularnewline
 &
\leftquadpar{Move selected \gls{object}} &
\mnemonictrail{edit.moveby}
\tabularnewline
 &
 Edit mode: &
\tabularnewline
 &
\leftquadpar{Move selected \gls{controlpt}} &
\accelerator{contextmenu} \mnemonic{editpath.coordinates}
\tabularnewline
%
\accelerator{edit.undo} &
Undo &
\mnemonictrail{edit.undo} \tabularnewline
%
\accelerator{edit.redo} &
Redo &
\mnemonictrail{edit.redo}
\tabularnewline
%
\accelerator{menubarfocus} &
Switch focus to menu bar &
\tabularnewline
%
\accelerator{redraw} &
Repaint canvas &
\tabularnewline
%
\accelerator{contexthelp} &
Display current tool help (status bar help button) &
\tabularnewline
%
\accelerator{save_all} &
Saves all images to the \gls{config_dir} &
\tabularnewline
%
\accelerator{help} &
Display subject help (help buttons in dialog windows or preamble
editor) &
\tabularnewline
\accelerator{settings.grid.lock} &
Lock/unlock grid &
\mnemonictrail{settings.grid.lock}
\tabularnewline
%
\accelerator{navigate.select} &
Select next \gls{object} in the \gls{stack} (from the
\gls{front}), and deselect all others &
\mnemonictrail{navigate.select} \tabularnewline
%
\keys{\actualkey{F6}} &
Select mode: &
\tabularnewline
&
\leftquadpar{Add next \gls{object} in the \gls{stack}
     (from the \gls{front}) to selection} &
\mnemonictrail{navigate.add_next}
\tabularnewline
 &
Edit mode: &
\tabularnewline
%
 &
\leftquadpar{Select previous \gls{controlpt}} &
\accelerator{contextmenu} \mnemonic{editpath.prev_control}
\tabularnewline
%
\accelerator{navigate.find} &
\fnsym{2}Find selected \gls{object} &
\mnemonictrail{navigate.find}
\tabularnewline
%
\accelerator{debug_show} &
Displays debugging information &
\tabularnewline
%
\accelerator{debug_log} &
Writes log file in the \gls{config_dir} &
\tabularnewline
\accelerator{close_window} &
Quit &
\mnemonictrail{file.quit}
\tabularnewline
\midrule
%
\accelerator{edit.select_all} &
Select all \gls{object} &
\mnemonictrail{edit.select_all}
\tabularnewline
%
\accelerator{edit.back} &
Move selected \gls{object} to the \gls{back} &
\mnemonictrail{edit.back}
\tabularnewline
%
\accelerator{edit.copy} &
Copy selected \gls{object} to clipboard &
\mnemonictrail{edit.copy}
\tabularnewline
%
\accelerator{transform.convert} &
Convert outline to a \gls{path} &
\mnemonictrail{transform.convert}
\tabularnewline
%
\accelerator{tools.ellipse} &
Switch to ellipse tool &
\mnemonictrail{tools.ellipse}
\tabularnewline
%
\keys{\keyref{ctrl}+\actualkey{F}} &
Select mode: &
\tabularnewline
 &
\leftquadpar{Move selected \gls{object} to the \gls{front}} &
\mnemonictrail{edit.front}
\tabularnewline
 & \TeX\ editor: &
\tabularnewline
 &
\leftquadpar{Find text} &
\mnemonic{texeditor.search.find}
\tabularnewline
\keys{\keyref{ctrl}+\actualkey{G}} &
Select mode: &
\tabularnewline
 &
\leftquadpar{Group selected \gls{object}} &
\mnemonictrail{transform.group}
\tabularnewline
 &
\TeX\ editor: &
\tabularnewline
 &
\leftquadpar{Find again} &
\mnemonic{texeditor.search.find_again}
\tabularnewline
\keys{\keyref{ctrl}+\actualkey{H}} &
Select mode: &
\tabularnewline
 &
\leftquadpar{Shear selected \gls{object}} &
\mnemonictrail{transform.shear}
\tabularnewline
 &
\TeX\ editor: &
\tabularnewline
 &
\leftquadpar{Find and Replace text} &
\mnemonictrail{texeditor.search.replace}
\tabularnewline
%
\accelerator{edit.path.edit} &
Edit selected \gls{path} &
\mnemonictrail{edit.path.edit}
\tabularnewline
%
\accelerator{transform.merge} &
Merge selected \gls{path} &
\mnemonictrail{transform.merge}
\tabularnewline
%
\accelerator{tools.open_curve} &
Switch to open curve tool &
\mnemonictrail{tools.open_curve}
\tabularnewline
%
\accelerator{tools.open_line} &
Switch to open line tool &
\mnemonictrail{tools.open_line}
\tabularnewline
%
\accelerator{tools.gap} &
Gap function &
\mnemonictrail{tools.gap}
\tabularnewline
%
\accelerator{file.new} &
New canvas &
\mnemonictrail{file.new}
\tabularnewline
%
\accelerator{file.open} &
Open \gls{jdr} or \gls{jdr} file &
\mnemonictrail{file.open}
\tabularnewline
%
\accelerator{tools.select} &
Switch to select tool &
\mnemonictrail{tools.select}
\tabularnewline
%
\accelerator{file.quit} &
Quit &
\mnemonictrail{file.quit}
\tabularnewline
%
\accelerator{tools.rectangle} &
Switch to rectangle tool &
\mnemonictrail{tools.rectangle}
\tabularnewline
%
\accelerator{file.save} &
Save current image &
\mnemonictrail{file.save}
\tabularnewline
%
\accelerator{tools.textarea} &
Switch to text tool &
\mnemonictrail{tools.textarea}
\tabularnewline
%
\accelerator{transform.ungroup} &
Ungroup selected \gls{group} &
\mnemonictrail{transform.ungroup}
\tabularnewline
%
\accelerator{edit.paste} &
Paste \gls{object} from clipboard &
\mnemonictrail{edit.paste}
\tabularnewline
%
\accelerator{transform.rotate} &
Rotate selected \gls{object} &
\mnemonictrail{transform.rotate}
\tabularnewline
%
\accelerator{edit.cut} &
Cut selected \gls{object} &
\mnemonictrail{edit.cut}
\tabularnewline
%
\accelerator{edit.path.style.all} &
Edit the selected paths' line styles &
\mnemonictrail{edit.path.style.all}
\tabularnewline
%
\accelerator{transform.scale} &
Scale selected \gls{object} &
\mnemonictrail{transform.scale}
\tabularnewline
%
\accelerator{edit.deselect_all} &
Deselect all &
\mnemonictrail{edit.deselect_all}
\tabularnewline
%
\accelerator{edit.movedown} &
Move selected \gls{object} down the \gls{stack} &
\mnemonictrail{edit.movedown}
\tabularnewline
%
\accelerator{edit.moveup} &
Move selected \gls{object} up the \gls{stack} &
\mnemonictrail{edit.moveup}
\tabularnewline
%
\accelerator{edit.textarea.edit} &
Edit selected text &
\mnemonictrail{edit.textarea.edit}
\tabularnewline
%
\accelerator{tools.closed_curve} &
Switch to closed curve tool &
\mnemonictrail{tools.closed_curve}
\tabularnewline
%
\accelerator{tools.closed_line} &
Switch to closed line tool &
\mnemonictrail{tools.closed_line}
\tabularnewline
%
\accelerator{tools.math} &
Switch to \dgls{maths-mode} tool &
\mnemonictrail{tools.math}
\tabularnewline
\midrule
%
\keys{\actualkey{1}}\ldots\keys{\actualkey{8}} &
Linear gradient paint direction selectors &
\tabularnewline
%
\keys{\actualkey{1}}\ldots\keys{\actualkey{9}} &
Radial gradient paint start location selectors &
\end{longtable}


\chapter{Settings}\label{sec:settings}

You can customise the appearance of \appname's main window either
using the command line arguments (\sectionref{sec:cmdargs}) or
using the settings menu (\sectionref{sec:settingsmenu}).

\section{Command Line Arguments}\label{sec:cmdargs}

\appdef{flowframtk}

\switchdef{help}
Display help message and exit.

TODO

\section{The Settings Menu}\label{sec:settingsmenu}

TODO

\subsection{Styles}\label{sec:setcurrentstyles}

TODO

\subsection{Show Rulers}\label{sec:showrulers}

TODO

\subsection{Grid}\label{sec:gridmenu}

TODO

\subsection{Zoom}\label{sec:zoommenu}

TODO


\subsection{Paper}\label{sec:papermenu}

TODO


\subsection{Configure Image Settings}\label{sec:configuredialog}

TODO


\subsubsection{Control Points}\label{sec:controlsettings}

TODO


\subsubsection{Startup Directory}\label{sec:startdir}

TODO


\subsubsection{JDR/AJR Settings}\label{sec:jdrsettings}

TODO


\subsubsection{Startup Settings}\label{sec:initsettings}

TODO


\subsubsection{Bitmaps}\label{sec:bitmapconfig}

TODO


\subsubsection{Application Paths}\label{sec:processes}

TODO


\subsection{\TeX\ Settings Dialog}\label{sec:texconfig}

TODO


\subsubsection{Setting the Document Class and Normal Font Size}
\label{sec:normalsize}

TODO


\subsubsection{Flowframe Settings}\label{sec:texconfigflf}

TODO


\subsubsection{Text Settings}\label{sec:texconfigtext}

TODO


\subsubsection{Default Preamble}\label{sec:texconfigpreamble}

TODO


\subsection{Configure User Interface}\label{sec:configureuidialog}

TODO


\subsubsection{Graphics Settings}\label{sec:graphics}

TODO


\subsubsection{Annotations Settings}\label{sec:annotations}

TODO


\subsubsection{Language Settings}\label{sec:languages}

TODO


\subsubsection{Accelerator Settings}\label{sec:accelerators}

TODO


\subsubsection{Toolbar, Ruler and Status Bar Settings}\label{sec:rulers}

TODO


\subsubsection{Normalization}\label{sec:normalize}

TODO


\subsubsection{TeX Editor Settings}\label{sec:texeditorui}

TODO


\subsubsection{Look and Feel Settings}\label{sec:lookandfeel}

TODO


\section{Configuration Directory}\label{sec:configdir}

\inlineglsdef{config_dir}

TODO

\chapter{The Basics}\label{sec:thebasics}

TODO


\section{The Canvas}\label{sec:thecanvas}

The \inlineglsdef{canvas}
TODO


\section{The Toolbars}\label{sec:thetoolbars}

The \inlineglspluraldef{toolbar}
TODO


\section{The Rulers}\label{sec:therulers}

The \inlineglspluraldef{ruler}
TODO


\section{The Status Bar}\label{sec:thestatusbar}

The \inlineglsdef{statusbar}
TODO


\chapter{The File Menu}\label{sec:filemenu}

TODO


\section{New}\label{sec:newimage}

TODO


\section{Open}\label{sec:loadimage}

TODO

\section{Recent Files}\label{sec:recentfiles}

TODO


\section{Image Description}\label{sec:imagedescription}

TODO


\section{Save and Save As}\label{sec:saveimage}

TODO

\section{Export}\label{sec:exportimage}

TODO


\subsection{Export to PNG}\label{sec:exportpng}

TODO


\subsection{Export to a Class or Package}\label{sec:exportsty}

TODO


\subsection{Export to PGF}\label{sec:exportpgf}

TODO


\subsection{Export to Single-Paged Document}\label{sec:exportdoc}

TODO


\section{Print}\label{sec:print}

TODO


\section{Message Window}\label{sec:messages}

TODO


\section{Close}\label{sec:closeimage}

TODO


\section{Quit}\label{sec:quit}

TODO


\chapter{Creating New Objects}\label{sec:newobjects}

TODO

\section{Line Paths}\label{sec:newlinepath}

TODO

\section{Curve Paths}\label{sec:newcurvepath}

TODO


\section{Rectangles}\label{sec:rectangles}

TODO


\section{Ellipses}\label{sec:ellipses}

TODO


\section{Text}\label{sec:newtext}

TODO


\chapter{Bitmaps}\label{sec:insertbitmap}

TODO


\section{Properties}\label{sec:bitmapprops}

TODO


\section{Vectorizing}\label{sec:vectorize}

TODO


\chapter{Selecting and Editing Objects}\label{sec:selectobjects}

TODO


\section{Moving an Object}\label{sec:moveobjects}

TODO


\section{Cut}\label{sec:cutobjects}

TODO


\section{Copy}\label{sec:copyobjects}

TODO


\section{Paste}\label{sec:pasteobjects}

TODO


\section{Object Description}\label{sec:objectdescription}

TODO


\section{Editing Control Points}\label{sec:editpath}

TODO


\section{Symmetric Shapes}\label{sec:symmetric}

TODO


\section{Editing Text Areas}\label{sec:edittext}

TODO


\section{Combining a Text Area and Path to Form a Text-Path}\label{sec:textpath}

TODO


\section{Text Outlines}\label{sec:textoutline}

TODO


\section{Reducing to Gray Scale}\label{sec:reduce}

TODO


\section{Color Space Conversions}\label{sec:convertcolspace}

TODO


\section{Fade}\label{sec:fade}

TODO


\section{Removing Translucency}\label{sec:removetrans}

TODO


\section{Moving an Object Up or Down the Stacking Order}\label{sec:moveupordown}

TODO


\section{Rotating Objects}\label{sec:rotateobjects}

TODO


\section{Scaling Objects}\label{sec:scaleobjects}

TODO


\section{Shearing Objects}\label{sec:shearobjects}

TODO


\section{Distorting Shapes}\label{sec:distort}

TODO


\section{Grouping and Ungrouping Objects}\label{sec:grouping}

TODO


\section{Aligning Objects}\label{sec:alignobjects}

TODO


\section{Reversing a Path's Direction}\label{sec:reversing}

TODO


\section{Merging Paths}\label{sec:mergepaths}

TODO


\section{Path Union}\label{sec:pathunion}

TODO


\section{Exclusive Or Function}\label{sec:xorpath}

TODO


\section{Path Intersection}\label{sec:pathintersect}

TODO


\section{Path Subtraction}\label{sec:pathsubtract}

TODO


\section{Separating a Text-Path into a Text Area and Path}\label{sec:separate}

TODO


\section{Converting a Path or Text-Path into a Pattern}\label{sec:patterns}

TODO


\section{Converting to a Path}\label{sec:converttopath}

TODO


\section{Converting a Text Area, Text-Path or Pattern to a Path}\label{sec:texttopath}

TODO


\section{Converting an Outline to a Path}\label{sec:outlinetopath}

TODO


\section{Splitting Text Areas}\label{sec:splittext}

TODO


\chapter{Path and Text Styles}\label{sec:styles}

TODO


\section{Line Color}\label{sec:linepaint}

TODO


\section{Fill Color}\label{sec:fillpaint}

TODO


\section{Line Style}\label{sec:pathstyle}

TODO


\subsection{Line Thickness (or Pen Width)}\label{sec:penwidth}

TODO


\subsection{Dash Pattern}\label{sec:dashpattern}

TODO


\subsection{Cap Style}\label{sec:capstyle}

TODO


\subsection{Join Style}\label{sec:joinstyle}

TODO


\subsection{Markers}\label{sec:markers}

TODO


\subsubsection{Enabling or Disabling a Marker}\label{sec:enablingmarkers}

TODO


\subsubsection{Marker Types}\label{sec:markertypes}

TODO


\subsubsection{Marker Size}\label{sec:markersize}

TODO


\subsubsection{Repeating Markers}\label{sec:repeatingmarkers}

TODO


\subsubsection{Reversing Markers}\label{sec:reversingmarkers}

TODO


\subsubsection{Composite Markers}\label{sec:compositemarkers}

TODO


\subsubsection{Marker Orientation}\label{sec:markerorientation}

TODO


\subsubsection{Marker Offset}\label{sec:markeroffset}

TODO


\subsubsection{Repeat Gap}\label{sec:repeatgap}

TODO


\subsubsection{Marker Color}\label{sec:markerpaint}

TODO


\subsection{Winding Rule}\label{sec:winding}

TODO


\section{Text Color}\label{sec:textpaint}

TODO


\section{Text Style}\label{sec:textstyle}

TODO


\subsection{Font Family}\label{sec:fontfamily}

TODO


\subsection{Font Size}\label{sec:fontsize}

TODO


\subsection{Font Series}\label{sec:fontseries}

TODO


\subsection{Font Shape}\label{sec:fontshape}

TODO


\subsection{Text Transformation Matrix}\label{sec:textmatrix}

TODO


\subsection{Anchor}\label{sec:fontanchor}

TODO


\chapter{\TeX/\LaTeX}\label{sec:tex}

TODO


\section{Adding Commands to the Preamble}\label{sec:setpreamble}

TODO


\section{Computing the Parameters for \glsfmttext{parshape}}\label{sec:parshape}

TODO


\section{Computing the Parameters for \glsfmttext{shapepar} or
\glsfmttext{Shapepar}}
\label{sec:shapepar}

TODO


\section{Creating Frames for Use with the \stytext{flowfram} Package}
\label{sec:flowframe}

TODO


\subsection{The \stytext{flowfram} Package: A Brief Summary}
\label{sec:flowframesummary}

TODO


\subsection{Defining the Typeblock}\label{sec:typeblock}

TODO


\subsection{Defining a Frame}\label{sec:framedef}

TODO


\subsection{The Frame Shape}\label{sec:frameshape}

TODO


\subsection{Scale Object to Fit Typeblock}\label{sec:scaletotypeblock}

TODO


\subsection{Only Displaying Objects Defined on a Given Page}\label{sec:displaypage}

TODO


\chapter{Step-by-Step Examples}\label{sec:tutorials}

TODO


\section{A House}\label{sec:houseexample}

TODO


\section{Lettuce on Toast}\label{sec:toastexample}

TODO


\section{Cheese and Lettuce on Toast}\label{sec:cheeseexample}

TODO


\section{An Artificial Neuron}\label{sec:neuronexample}

TODO


\section{Bus}\label{sec:busexample}

TODO


\section{A Poster}\label{sec:postertutorial}

TODO


\section{A Newspaper}\label{sec:newstutorial}

TODO


\section{A Brochure}\label{sec:brochure}

TODO


\section{A House With No Mouse}\label{sec:accesstutorial}

TODO


\section{A Lute Rose}\label{sec:rosetutorial}

TODO


\chapter{JDR/AJR File Formats}\label{sec:jdrajrformat}

TODO


\chapter{Multilingual Support}\label{sec:multilingualsupport}

TODO


\chapter{Source Code}\label{sec:sourcecode}

TODO


\section{Java Source}\label{sec:javasource}

TODO


\section{Document Source}\label{sec:docsource}

TODO


\chapter{Troubleshooting}\label{sec:troubleshooting}

TODO


\section{Known Bugs}\label{sec:knownbugs}

TODO

\chapter{\MFUsentencecase{\glsentrytext{index.licence}}}
\label{sec:licence}

\glsadd{index.licence}%
\appname\ is licensed under the terms of the 
\href{https://www.gnu.org/licenses/gpl-3.0.html}{GNU General
Public License version 3 (GPLv3)}.
\appname\ depends on the following third party library whose
jar file is in the \filefmt{lib} directory:
\begin{itemize}
   \item TeX Java Help \filefmt{texjavahelplib.jar}
   (GPL, \url{https://github.com/nlct/texjavahelp}).
\end{itemize}


\listentrydescendents
 [title={Summary of \apptext{flowframtk} Switches}]
 {app.flowframtk}

\printmain
\printindex 

\end{document} 

