\documentclass{poster}% new class created in this example

\begin{document}

This is the main body of the poster. This text will
appear in the first of the two flow frames. Once it
has reached the end of the first flow frame, it will then
continue in the second flow frame.

The class, poster.cls, was created by flowframtk, and
uses the flowfram package. All the package options 
defined by the flowfram package are also available for
this new class. If used, they are simply passed on to the
flowfram package, so you can do, say
\begin{verbatim}
\documentclass[draft]{poster}
\end{verbatim}
and you will see the outline of all the defined frames.
Similarly if a package instead of a class had been exported.

The new class also uses the geometry package to 
set the page layout, and the pgf package to draw in the
frame's borders.

Note that flowframtk will only define the frame, and set the
frame's border (and offset). It will not set any of the
other keys, so if, for example, you want to rotate the
frame contents, you will have to edit that in here.

The flowfram package isn't perfect, and will sometimes
create an additional unwanted page. It also conflicts
with certain other packages that also change the output
routine.

Remember that you can't use the normal figure and
table environments within a dynamic or static frame,
instead use staticfigure (see Figure~\ref{fig:sample}) 
and statictable (see Table~\ref{tab:sample}).

\end{document}
